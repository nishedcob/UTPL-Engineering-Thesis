% Chapter 7

\chapter{Conclusiones}
\label{capitulo7}

% TODO: Write Resultados previo a conclusiones

% TODO: get input on conclusions (are they okay for the thesis) %% REVIEW
% Minimo 1 Conclusion frente cada Objetivo Especifico
% Minimo 1 Conclusion frente cada Linea Problematica
% Lo que no va en resultados, puede ir a conclusiones
% Conclusiones tienen que ser declaraciones positivas o negativas, no van preguntas
% Cuantas Conclusiones poner? 3-5 es acceptable -> 7 conclusiones suele ser demaciado
% Tipicamente un recomendacion por cada cada conclusion


A lo largo del desarrollo de este trabajo de titulacion, se ha formado los siguientes conclusiones:

\begin{itemize}
  \item Para el desarrollo de sistemas educativos, LTI puede ser una buena solución que se puede tomar en cuenta para facilitar el flujo de autenticación entre sistemas para el aprendizaje y enseñanza.
  \item Muchos usuarios solo se fijan en que tan llamativo es un interfaz de usuario y no en la funcionalidad que esta por detras. Por lo tanto es importante encontrar las herramientas adecuadas para que el editor de código en la plataforma sea tanto llamativa como funcional, porque la primera impresión es lo que más cuenta.
  \item Para datos que no tienen estructuras muy fijas o que tienen campos de longitudes altamente variables, como es el caso del codigo, MongoDB ofrece una buena alternativa a los bases de datos relacionales y simplifica el trabajo a ser realizado.
  \item Kubernetes y Docker son herramientas muy potentes para generar entornos confiables, entregar sistemas de producción valiosas y garantizar escalabilidad y portabilidad de código realizado debido a que su curva de aprendizaje es minima para el nivel de potencia que los mismos ofrecen.
  \item Cualquier servidor de Git es altamente pesada, especialmente GitLab que en su estado pasivo consume muchos recursos, y tambien GitWeb que tiene inconvenientes cada vez que recibe nuevos objetos de Git sobre HTTP.
  % TODO: Revisar Conclusion General %% REVIEW
  \item La combinacion de Python 3 con librerias selecionadas para realizar los respectivos backends, es decir Django, PyModm, Bcrypt, PyJWT, Requests y ipython forman una combinacion potente para el desarollo rapido de servicios de produccion siempre y cuando se maneja de forma adecuada el diseño de los mismos y el manejo de distintos tipos de datos.
\end{itemize}
