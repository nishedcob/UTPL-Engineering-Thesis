% Chapter 8

\chapter{Recommendaciones}
\label{capitulo8}
% TODO: revisar si recomendations son adecuadas %% REVIEW
%Tipicamente un recomendacion por cada cada conclusion

Se recomienda reutilizar una implementacion de LTI para evitar problemas de integración con lo que se está realizando.

Para realizar una interfaz web, si es que uno no es experto en diseñar páginas webs bonitas, se recomienda siempre apoyarse en el trabajo de terceros... Existe un montón de plantillas y librerías para cualquier tipo de funcionalidad a nivel de cliente, lo cual permite dar mayor enfoque a la funcionalidad del backend.

Para trabajar con cualquier base de datos, sea relacional o no relacional, lo más recomendable es trabajar con un ORM que permite abstraer la base de datos y facilitar el desarrollo y persistencia de datos en la misma.

Hay que no tener miedo de utilizar tecnología de punta, el cual puede ser muy útil para resolver problemas actuales ya que ofrece nuevas perspectivas y soluciones de los cuales no se los ha podido considerar antes.

Donde hay posibilidad, no se recomienda trabajar con GitLab, a menos de que dispone de los recursos necesarios para ello y tambien que requiere las características avanzadas del mismo. En entornos donde sea posible, lo más recomendable es trabajar con Git sobre SSH, que además de ser más seguro, tiene mayor soporte por parte de Git y mayor inteligencia para la sincronización cuando se trabaja con este protocolo que puede dar como resultado mayor eficiencia en uso de red.

% TODO: escribir recomendacion general?

\section{Trabajos Futuros}
% TODO: what was left out or could be improved?
En cuanto se 
% bug fixes on current functionality (for example:
%	always exec old version of exec-repo
%	strange exec-id generation
% )
% Auth LDAP
% realtime code sync
% socialization of editing, permisions system
% grading
%	manual
%	automatic based on unit tests
%	hybrid
% LTI Grades Sync
% more git backends, ie gitlab
% more code persistence backends like redis or gitlab
% more virtualization backends like openstack
% versions of apis
% full security audit -> how secure is jwt+tls for stateless apis?
% greater configuration management via settings, ie the option for content/functionality to change dynamically based on connected services
% services in docker on a kubernetes cluster as a high availability/scalability study
% GitEDU templating system
% update to Django 2.0 (breaks URI namespaces)
% Automate Testing, more complete tests
