% Chapter 8

\chapter{Recommendaciones}
\label{capitulo8}
% TODO: revisar si recomendations son adecuadas %% REVIEW
%Tipicamente un recomendacion por cada cada conclusion

Se recomienda reutilizar una implementacion de LTI para evitar problemas de integración con lo que se está realizando.

Para realizar una interfaz web, si es que uno no es experto en diseñar páginas webs bonitas, se recomienda siempre apoyarse en el trabajo de terceros... Existe un montón de plantillas y librerías para cualquier tipo de funcionalidad a nivel de cliente, lo cual permite dar mayor enfoque a la funcionalidad del backend.

Para trabajar con cualquier base de datos, sea relacional o no relacional, lo más recomendable es trabajar con un ORM que permite abstraer la base de datos y facilitar el desarrollo y persistencia de datos en la misma.

Hay que no tener miedo de utilizar tecnología de punta, el cual puede ser muy útil para resolver problemas actuales ya que ofrece nuevas perspectivas y soluciones de los cuales no se los ha podido considerar antes.

Donde hay posibilidad, no se recomienda trabajar con GitLab, a menos de que dispone de los recursos necesarios para ello y tambien que requiere las características avanzadas del mismo. En entornos donde sea posible, lo más recomendable es trabajar con Git sobre SSH, que además de ser más seguro, tiene mayor soporte por parte de Git y mayor inteligencia para la sincronización cuando se trabaja con este protocolo que puede dar como resultado mayor eficiencia en uso de red.

% TODO: escribir recomendacion general?

\section{Trabajos Futuros}
Se considera los siguientes aspectos que se podrian y/o se deben trabajar a futuro, los cuales no esta en ningun orden especifico:
\begin{itemize}
	\item Arreglar Errores en la funcionalidad existente, por ejemplo:
    \begin{itemize}
    	\item EduNube no actualiza de forma adecuada repositorios de ejecuccion, lo cual resulta muchas veces en la ejecuccion de una version antigua del mismo.
        \item EduNube no genera IDs de ejecuccion de forma adecuada.
    \end{itemize}
    \item Autenticacion en GitEDU y/o otros servicios mediante LDAP.
    \item Sincronizacion en tiempo real de codigo editado en GitEDU y sus respectivo backends de codigo, tal ves mediante websockets.
    \item Socializacion de vistas para editar codigo (GitEDU) con la finalidad de promover interaccion y collaboracion entre usuarios sobre los mismos.
    \item Un sistema de permisos para el editor de codigo (GitEDU).
    \item Distintas formas de calificacion, incluyendo:
    \begin{itemize}
    	\item Calificacion Manual por parte de professores.
        \item Calificacion Automatizado por parte del sistema (tal vez en forma de un servicio nuevo?) en base a pruebas unitarias definidos por professores.
        \item Calificacion Hibrida que combina los anteriores.
    \end{itemize}
    \item Sincronizacion de Notas (generados por la(s) modalidad(es) de calificacion anteriores) por LTI con los sistemas adecuadas para el manejo de los mismos, por ejemplo los LMS.
    \item Mas backends de Git para el servicio GitServerHTTPEndpoint, como por ejemplo GitLab o Djacket.
    \item Mas backends de persistencia de codigo para el servicio GitEDU como Redis o GitLab.
    \item Mas backends de virtualizacion para EduNube como OpenStack, Docker, etc.
    \item Versionamiento de APIs en todos los servicios.
    \item Auditoria de Seguridad del Sistema Desarrollado, especialmente en el caso de los APIs que no manejan estados y solo se protegen con API tokens JWT y a lo mucho TLS.
    \item Mejor gestion de la configuracion, actualmente hay componentes de algunos servicios que dejen de funcionar o que funcionan de forma inadecuada cuando no disponen de sus servicios dependientes. Debe ser configurable cuales servicios existen o no, dinamico la manera en que se encuentran y cada servicio tolerante a fallos en los demas servicios.
    \item Convertir los servicios en imagenes de Docker para montarlos mismos en un cluster de Kubernetes y realizar un estudio de alta disponabilidad/escalabilidad.
    \item Implementar el sistema de plantillas de GitEDU.
    \item Migrar servicios a Django 2.x (la proxima version de soporte a largo plazo de Django esta planificado como el 2.3) ya que en realizar esta migracion de version, actualmente se rompe la forma en que se llevan los URIs con un namespace para cada app.
    \item Automatizar y aumentar las pruebas unitarias de la aplicacion de acuerdo con el plan de pruebas original.
\end{itemize}
