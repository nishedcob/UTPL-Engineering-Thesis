% Appendix F - H - Pruebas

\chapter{Pruebas de GitEDU}
\label{AnexoF}

\section{Prueba de CodePersistenceBackend}
Con respeto al sistema GitEDU, el componente mas importante de probar fue el backend de persistencia de codigo, automatizado de la siguiente manera:
\lstset{language=Python}
\begin{lstlisting}[breaklines]
from pymongo import ASCENDING, DESCENDING
from ideApp.CodePersistenceBackends.MongoDB.mongodb_models import ChangeModel, ChangeFileModel, NamespaceModel,\
    RepositoryModel, RepositoryFileModel
from GitEDU.settings import CODE_PERSISTENCE_BACKEND_MANAGER_CLASS, load_code_persistence_backend_manager

manager = load_code_persistence_backend_manager(CODE_PERSISTENCE_BACKEND_MANAGER_CLASS)

print("code persistence backend manager: <%s>" % manager)

print("Changes in Ascending Order:")
for change in list(ChangeModel.objects.raw({}).order_by([("timestamp", ASCENDING)])):
    print(change)

print("Changes in Descending Order:")
for change in list(ChangeModel.objects.raw({}).order_by([("timestamp", DESCENDING)])):
    print(change)

namespace = "nishedcob"
repository = "test"
file_path = "folder/hello.py"

manager.sync("NRF")
namespace_objs = manager.get_namespace(namespace=namespace)
print(namespace_objs)
namespace_obj = manager.select_preferred_backend_object(result_set=namespace_objs)
print(namespace_obj)
repository_objs = manager.get_repository(namespace=namespace_obj, repository=repository)
print(repository_objs)
repository_obj = manager.select_preferred_backend_object(result_set=repository_objs)
print(repository_obj)
repository_file_objs = manager.get_file(namespace=namespace_obj, repository=repository_obj, file_path=file_path)
print(repository_file_objs)
repository_file_obj = manager.select_preferred_backend_object(result_set=repository_file_objs)
print(repository_file_obj)
print(repository_file_obj.persistence_object)
print(repository_file_obj.persistence_object.pk)
file_edits = ChangeFileModel.objects.raw({
    'file': repository_file_obj.persistence_object.pk,
    'file_path': file_path
})
print(list(file_edits))
edits = []
for file_edit in file_edits:
    print(file_edit.change.pk)
    change = ChangeModel.objects.raw({'_id': file_edit.change.pk}).first()
    print(change)
    edit = {
        'id': change.change_id,
        'author': change.author,
        'timestamp': change.timestamp.as_datetime().__str__(),
        'namespace': namespace,
        'repository': repository,
        'file_path': file_edit.file_path
    }
    print(edit)
    edits.append(edit)
    print(edits)
print(edits)
edits = edits.sort(key=lambda k: k['timestamp'])
print(edits)
\end{lstlisting}
\lstset{language=Bash}

\chapter{Pruebas de EduNube}
\label{AnexoG}

\section{Expirimento de Extraccion de ID de Ultimo Commit de un Repositorio de Git}
\lstset{language=Python}
\begin{lstlisting}[breaklines]
# coding: utf-8
from apiApp.VirtualizationBackends.Kubernetes import KubernetesVirtualizationBackend
kvb = KubernetesVirtualizationBackend()
kvb.get_id_last_git_commit(repository_path='/home/nyx/GitEDU')
kvb.get_id_last_git_commit(repository_path='/home/nyx/GitEDU-copy')
\end{lstlisting}

\section{Prueba Inicial de Ejecuccion con Kubernetes}
\lstset{language=Python}
\begin{lstlisting}
# coding: utf-8

import time

from apiApp.VirtualizationBackends.Kubernetes import Py3KubernetesVirtualizationBackend

# values for test:
kvb = Py3KubernetesVirtualizationBackend()
job_name = 'py3-pytest-3'
# with minikube
#git_repo = 'http://192.168.99.1/python3-code-executor-template.git'
# with Kubernetes on Xen
git_repo = 'http://10.10.10.1/python3-code-executor-template.git'
kubernetes_working_tmp_dir = '/tmp'

# prepare test manifest
manifest = kvb.build_job_template(job_name=job_name, git_repo=git_repo)
manifest_path = '%s/%s.json' % (kubernetes_working_tmp_dir, job_name)
kvb.write_json_manifest(path=manifest_path, json_data=manifest, overwrite=True)

# create job with manifest
print(kvb.job_status(job_id=job_name))
print(kvb.kubectl_create_from_manifest_file(manifest_path=manifest_path))

# job info
print(kvb.job_describe(job_id=job_name))

# get job status & wait until job completion
print(kvb.job_get(job_id=job_name))
while not kvb.job_finished(job_id=job_name):
    print(kvb.job_status(job_id=job_name))
    time.sleep(1)
print(kvb.job_status(job_id=job_name))
print(kvb.job_get(job_id=job_name))

# job info
print(kvb.job_describe(job_id=job_name))

# job output
print(kvb.job_logs(job_id=job_name))

# delete job
print(kvb.kubectl_delete_from_manifest_file(manifest_path=manifest_path))
\end{lstlisting}
\lstset{language=Bash}

\section{Prueba Inicial de Backend de RepoSpecs}
\lstset{language=Python}
\begin{lstlisting}[breaklines]
# coding: utf-8
from apiApp.Validation import RepoSpec
py3_code_exec_template_repospec = RepoSpec.create(repo="python3-code-executor-template", parent=None)
token = py3_code_exec_template_repospec.token
print(token)
secret_key = py3_code_exec_template_repospec.secret_key
print(secret_key)
py3_code_exec_template_repospec = RepoSpec.RepoSpec.objects.get(repo="python3-code-executor-template")
# token == py3_code_exec_template_repospec.token
# secret_key == py3_code_exec_template_repospec.secret_key
# RepoSpec.validate_repospec(py3_code_exec_template_repospec.token) == True
# RepoSpec.decode_repospec(repospec=py3_code_exec_template_repospec.token,
#                          stored_repospec=py3_code_exec_template_repospec)
#                          == {'repo': "python3-code-executor-template"}
# RepoSpec.decode_repospec(repospec=py3_code_exec_template_repospec.token) == {'repo': "python3-code-executor-template"}
# RepoSpec.decode(stored_repospec=py3_code_exec_template_repospec, repospec=None, decode_stored=True)
#                   == {'repo': "python3-code-executor-template"}
py3_code_exec_template_repospec = RepoSpec.update_repospec(repospec=py3_code_exec_template_repospec,
                                                           parent="http://192.168.1.100/shell.git",
                                                           repo=py3_code_exec_template_repospec.repo,
                                                           regen_secret_key=False)
# token != py3_code_exec_template_repospec.token
# secret_key == py3_code_exec_template_repospec.secret_key
token = py3_code_exec_template_repospec.token
print(token)
secret_key = py3_code_exec_template_repospec.secret_key
print(secret_key)
py3_code_exec_template_repospec = RepoSpec.update_repospec(repospec=py3_code_exec_template_repospec,
                                                           parent="http://192.168.1.100/shell.git",
                                                           repo=py3_code_exec_template_repospec.repo,
                                                           regen_secret_key=True)
# token != py3_code_exec_template_repospec.token
# secret_key != py3_code_exec_template_repospec.secret_key
token = py3_code_exec_template_repospec.token
print(token)
secret_key = py3_code_exec_template_repospec.secret_key
print(secret_key)
# RepoSpec.validate_repospec(repospec=py3_code_exec_template_repospec.token) == True
py3_code_exec_template_repospec = RepoSpec.RepoSpec.objects.get(repo="python3-code-executor-template")
# token == py3_code_exec_template_repospec.token
# secret_key == py3_code_exec_template_repospec.secret_key
token = py3_code_exec_template_repospec.token
print(token)
secret_key = py3_code_exec_template_repospec.secret_key
print(secret_key)
py3_code_exec_template_repospec = RepoSpec.update_repospec(repospec=py3_code_exec_template_repospec, parent=None,
                                                           repo=py3_code_exec_template_repospec.repo,
                                                           regen_secret_key=True)
# token != py3_code_exec_template_repospec.token
# secret_key != py3_code_exec_template_repospec.secret_key
token = py3_code_exec_template_repospec.token
print(token)
secret_key = py3_code_exec_template_repospec.secret_key
print(secret_key)
py3_code_exec_template_repospec = RepoSpec.update_repo(old_repo=py3_code_exec_template_repospec.token)
# token == py3_code_exec_template_repospec.token
# secret key == py3_code_exec_template_repospec.secret_key
py3_code_exec_template_repospec = RepoSpec.update_repo(old_repo=py3_code_exec_template_repospec.token,
                                                       regen_secret_key=True)
# token != py3_code_exec_template_repospec.token
# secret_key != py3_code_exec_template_repospec.secret_key

from apiApp.Validation import RepoSpec
postgresql_code_exec_template_repospec = RepoSpec.create(repo="postgresql-code-executor-template", parent=None)
token = postgresql_code_exec_template_repospec.token
print(token)
secret_key = postgresql_code_exec_template_repospec.secret_key
print(secret_key)
postgresql_code_exec_template_repospec = RepoSpec.RepoSpec.objects.get(repo="postgresql-code-executor-template")
# str(token, 'utf-8') == postgresql_code_exec_template_repospec.token
# str(secret_key, 'utf-8') == postgresql_code_exec_template_repospec.secret_key
# RepoSpec.validate_repospec(postgresql_code_exec_template_repospec.token) == True
# RepoSpec.decode_repospec(repospec=postgresql_code_exec_template_repospec.token,
#                          stored_repospec=postgresql_code_exec_template_repospec)
#                          == {'repo': "postgresql-code-executor-template"}
# RepoSpec.decode_repospec(repospec=postgresql_code_exec_template_repospec.token) == {'repo': "postgresql-executor-template"}
# RepoSpec.decode(stored_repospec=postgresql_code_exec_template_repospec, repospec=None, decode_stored=True)
#                   == {'repo': "postgresql-code-executor-template"}
\end{lstlisting}
\lstset{language=Bash}

\section{Cliente Ejemplar de API de RepoSpecs}
\lstset{language=Python}
\begin{lstlisting}[breaklines]

import requests

# Would normally be imported from settings, but for demonstration purposes:
#from EduNube.settings import EDUNUBE_CONFIG
EDUNUBE_CONFIG = {
    "protocol": "http",
    "host": "10.10.10.1",
    "port": 8010,
    "token": "eyJ0eXAiOiJKV1QiLCJhbGciOiJIUzI1NiJ9.eyJhcHBfbmFtZSI6IkdpdEVEVSIsImV4cGlyZXMiOmZhbHNlLCJjcmVhdGVkX2RhdGUi"
             "OiIyMDE3LTExLTEyIDE3OjMzOjE3LjY0MTY4MSIsImVkaXRfZGF0ZSI6IjIwMTctMTEtMTIgMjA6MzY6NDAuMjc5NDAyIn0.825oh2rZU"
             "lIPZFaP_UbYPDpdsXTE0XCaNsia-3NnGuc"
}


class EduNubeRepoSpecHTTPconsumer:

    create_operation = "create"
    get_operation = "get"
    get_or_create_operation = "get_or_create"
    edit_operation = "edit"

    url_template = "%s://%s:%d/api/repospec/%s"

    def build_url(self, protocol, host, port, operation):
        return self.url_template % (protocol, host, port, operation)

    def build_inicial_payload(self):
        return dict()

    def post_url(self, url, payload):
        return requests.post(url, data=payload)

    def validate_url(self, url):
        return url


class ConfigEduNubeRepoSpecHTTPconsumer(EduNubeRepoSpecHTTPconsumer):

    protocol = None
    host = None
    port = None
    object_type = None
    token = None

    def __init__(self, protocol=None, host=None, port=None, object_type=None, token=None):
        self.protocol = self.protocol if protocol is None else protocol
        self.host = self.host if host is None else host
        self.port = self.port if port is None else port
        self.object_type = self.object_type if object_type is None else object_type
        self.token = self.token if token is None else token

    def build_inicial_payload(self):
        return {
            'token': self.token
        }

    def build_config_url(self, operation):
        return self.build_url(protocol=self.protocol, host=self.host, port=self.port, operation=operation)


class DefaultConfigEduNubeRepoSpecHTTPconsumer(ConfigEduNubeRepoSpecHTTPconsumer):
    protocol = EDUNUBE_CONFIG.get('protocol')
    host = EDUNUBE_CONFIG.get('host')
    port = EDUNUBE_CONFIG.get('port')
    token = EDUNUBE_CONFIG.get('token')


class EduNubeRepoSpecConsumer(DefaultConfigEduNubeRepoSpecHTTPconsumer):

    def _validate_findable(self, repo=None, repospec_token=None):
        if repo is None and repospec_token is None:
            raise ValueError("repo and repospec_token can't both be None")

    def _make_call(self, operation, payload):
        url = self.build_config_url(operation=operation)
        return self.post_url(url=url, payload=payload)

    def _if_not_none_add_to_payload(self, payload, payload_index, value):
        if value is not None:
            payload[payload_index] = value
        return payload

    def create(self, repo, parent_repo=None):
        payload = self.build_inicial_payload()
        payload['repo'] = repo
        payload = self._if_not_none_add_to_payload(payload=payload, payload_index='parent', value=parent_repo)
        return self._make_call(operation=self.create_operation, payload=payload)

    def get(self, repo=None, repospec_token=None):
        self._validate_findable(repo=repo, repospec_token=repospec_token)
        payload = self.build_inicial_payload()
        payload = self._if_not_none_add_to_payload(payload=payload, payload_index='repo', value=repo)
        payload = self._if_not_none_add_to_payload(payload=payload, payload_index='repospec_token', value=repospec_token)
        return self._make_call(operation=self.get_operation, payload=payload)

    def get_or_create(self, repo, parent_repo=None, repospec_token=None):
        payload = self.build_inicial_payload()
        payload['repo'] = repo
        payload = self._if_not_none_add_to_payload(payload=payload, payload_index='parent', value=parent_repo)
        payload = self._if_not_none_add_to_payload(payload=payload, payload_index='repospec_token', value=repospec_token)
        return self._make_call(operation=self.get_or_create_operation, payload=payload)

    def edit(self, repo=None, repospec_token=None, parent_repo=None, new_repo=None, regen_secret_key=False):
        self._validate_findable(repo=repo, repospec_token=repospec_token)
        payload = self.build_inicial_payload()
        payload = self._if_not_none_add_to_payload(payload=payload, payload_index='repo', value=repo)
        payload = self._if_not_none_add_to_payload(payload=payload, payload_index='repospec_token', value=repospec_token)
        payload = self._if_not_none_add_to_payload(payload=payload, payload_index='parent', value=parent_repo)
        payload = self._if_not_none_add_to_payload(payload=payload, payload_index='new_repo', value=new_repo)
        if type(regen_secret_key) == bool:
            payload['regen_secret_key'] = regen_secret_key
        return self._make_call(operation=self.edit_operation, payload=payload)
\end{lstlisting}
\lstset{language=Bash}

\section{Prueba inicial de API de Ejecuccion y Jobs}
\lstset{language=Python}
\begin{lstlisting}[breaklines]
import json

from execute_status_example_client import EduNubePy3ExecuteStatusConsumer

en_py3_es_consumer = EduNubePy3ExecuteStatusConsumer()

namespace = 'nishedcob'
repository = 'test'

inicial_execution_create = en_py3_es_consumer.create(namespace=namespace, repository=repository)

print("Inicial Exec Create Code: %d" % inicial_execution_create.status_code)
print("Inicial Exec Create Reply Data: %s" % inicial_execution_create.text)
inicial_execution_create_reply_data = json.loads(inicial_execution_create.text)
print("Inicial Exec Create Reply Data (JSON Loads): %s" % inicial_execution_create_reply_data)

print()
print("-------------------------------------------------------------------------")
print()

inicial_execution_id = inicial_execution_create_reply_data.get('id')

inicial_execution_status = en_py3_es_consumer.status(id=inicial_execution_id)

print("Inicial Exec Status Code: %d" % inicial_execution_status.status_code)
print("Inicial Exec Status Reply Data: %s" % inicial_execution_status.text)
inicial_execution_status_reply_data = json.loads(inicial_execution_status.text)
print("Inicial Exec Status Reply Data (JSON Loads): %s" % inicial_execution_status_reply_data)

print()
print("-------------------------------------------------------------------------")
print()

inicial_execution_result = en_py3_es_consumer.result(id=inicial_execution_id)
print("Initial Exec Result Code: %d" % inicial_execution_result.status_code)
print("Inicial Exec Result Reply Data: %s" % inicial_execution_result.text)
inicial_execution_result_reply_data = json.loads(inicial_execution_result.text)
print("Inicial Exec Result Reply Data (JSON Loads): %s" % inicial_execution_result_reply_data)
\end{lstlisting}
\lstset{language=Bash}

\section{Cliente de API de Ejecuccion y Jobs}
\lstset{language=Python}
\begin{lstlisting}[breaklines]

import requests
import json

# Would normally be imported from settings, but for demonstration purposes:
#from EduNube.settings import EDUNUBE_CONFIG
EDUNUBE_CONFIG = {
    "protocol": "http",
    "host": "10.10.10.1",
    "port": 8010,
    "token": "eyJ0eXAiOiJKV1QiLCJhbGciOiJIUzI1NiJ9.eyJhcHBfbmFtZSI6IkdpdEVEVSIsImV4cGlyZXMiOmZhbHNlLCJjcmVhdGVkX2RhdGUi"
             "OiIyMDE3LTExLTEyIDE3OjMzOjE3LjY0MTY4MSIsImVkaXRfZGF0ZSI6IjIwMTctMTEtMTIgMjA6MzY6NDAuMjc5NDAyIn0.825oh2rZU"
             "lIPZFaP_UbYPDpdsXTE0XCaNsia-3NnGuc"
}


class EduNubeExecuteStatusHTTPConsumer:

    create_operation = 'create'
    status_operation = 'status'
    result_operation = 'result'

    url_template = "%s://%s:%d/api/execute/%s"

    def build_base_url(self, protocol, host, port, operation):
        return self.url_template % (protocol, host, port, operation)

    def build_inicial_payload(self):
        return dict()

    def post_url(self, url, payload):
        return requests.post(url, data=payload)

    def validate_url(self, url):
        return url


class ConfigEduNubeExecuteStatusHTTPConsumer(EduNubeExecuteStatusHTTPConsumer):

    protocol = None
    host = None
    port = None
    object_type = None
    token = None

    def __init__(self, protocol=None, host=None, port=None, object_type=None, token=None):
        self.protocol = self.protocol if protocol is None else protocol
        self.host = self.host if host is None else host
        self.port = self.port if port is None else port
        self.object_type = self.object_type if object_type is None else object_type
        self.token = self.token if token is None else token

    def build_inicial_payload(self):
        return {
            'token': self.token
        }

    def build_base_config_url(self, operation):
        return self.build_base_url(protocol=self.protocol, host=self.host, port=self.port, operation=operation)


class DefaultConfigEduNubeExecuteStatusHTTPConsumer(ConfigEduNubeExecuteStatusHTTPConsumer):
    protocol = EDUNUBE_CONFIG.get('protocol')
    host = EDUNUBE_CONFIG.get('host')
    port = EDUNUBE_CONFIG.get('port')
    token = EDUNUBE_CONFIG.get('token')


class EduNubeExecuteStatusConsumer(DefaultConfigEduNubeExecuteStatusHTTPConsumer):

    def _prepare_repo_suffix(self, language, namespace, repository):
        return "%s/%s/%s/" % (language, namespace, repository)

    def _prepare_id_suffix(self, language, id):
        return "%s/%s/" % (language, id)

    def _make_call(self, operation, suffix, payload=None):
        if payload is None:
            payload = self.build_inicial_payload()
        url = "%s/%s" % (self.build_base_config_url(operation=operation), suffix)
        return self.post_url(url=url, payload=payload)

    def create(self, language, namespace, repository):
        return self._make_call(
            operation=self.create_operation,
            suffix=self._prepare_repo_suffix(language=language, namespace=namespace, repository=repository)
        )

    def status(self, language, id):
        return self._make_call(
            operation=self.status_operation,
            suffix=self._prepare_id_suffix(language=language, id=id)
        )

    def result(self, language, id):
        return self._make_call(
            operation=self.result_operation,
            suffix=self._prepare_id_suffix(language=language, id=id)
        )


class EduNubeLanguageExecuteStatusConsumer(EduNubeExecuteStatusConsumer):

    language = None

    def create(self, namespace, repository, language=None):
        if language is None:
            language = self.language
        return super().create(language=language, namespace=namespace, repository=repository)

    def status(self, id, language=None):
        if language is None:
            language = self.language
        return super().status(language=language, id=id)

    def result(self, id, language=None):
        if language is None:
            language = self.language
        return super().result(language=language, id=id)


class EduNubeShellExecuteStatusConsumer(EduNubeLanguageExecuteStatusConsumer):
    language = 'shell'


class EduNubePy3ExecuteStatusConsumer(EduNubeLanguageExecuteStatusConsumer):
    language = 'python3'


class EduNubePGSQLExecuteStatusConsumer(EduNubeLanguageExecuteStatusConsumer):
    language = 'postgresql'
\end{lstlisting}
\lstset{language=Bash}

\chapter{Pruebas de GitServerHTTPEndpoint}
\label{AnexoH}

\section{Cliente Ejemplar de API con pruebas integradas}
\lstset{language=Python}
\begin{lstlisting}[breaklines]

# import config from somewhere, for example with Django I would use a settings file
#from GitEDU.settings import GIT_SERVER_HTTP_ENDPOINT_CONFIG

# In this case, I will just place an example config here, extracted from GitEDU:
GIT_SERVER_HTTP_ENDPOINT_CONFIG = {
    "protocol": "http",
    "host": "127.0.0.1",
    "port": 8020,
    "token": "eyJhbGciOiJIUzI1NiIsInR5cCI6IkpXVCJ9.eyJleHBpcmVzIjpmYWxzZSwiYXBwX25hbWUiOiJHaXRFRFUiLCJlZGl0X2RhdGUiOiIy"
             "MDE3LTExLTI1IDIwOjI2OjIzLjQ3MDgxNyswMDowMCIsImNyZWF0ZWRfZGF0ZSI6IjIwMTctMTEtMjUgMjA6MjY6MjMuNDcwNzUwKzAwO"
             "jAwIn0.jVHEmUAgJcQy7sU-qyULAnAiIrBAPNbeDjOiwjk5EEk"
}

import requests


class GitServerHTTPEndpointConsumer:

    create_operation = 'create'
    edit_operation = 'edit'
    edit_mv_operation = "%s/mv" % edit_operation
    edit_contents_operation = "%s/contents" % edit_operation

    url_template = "%s://%s:%d/api/%s/%s/%s"

    def build_url(self, protocol, host, port, object_type, operation, object_path):
        return self.url_template % (protocol, host, port, object_type, operation, object_path)

    def build_inicial_payload(self):
        return dict()

    def post_url(self, url, payload):
        return requests.post(url, data=payload)

    def validate_url(self, url):
        if not url.endswith("/"):
            url += "/"
        return url


class ConfigGitSrvHTTPEpConsumer(GitServerHTTPEndpointConsumer):

    protocol = None
    host = None
    port = None
    object_type = None
    token = None

    def __init__(self, protocol=None, host=None, port=None, object_type=None, token=None):
        self.protocol = self.protocol if protocol is None else protocol
        self.host = self.host if host is None else host
        self.port = self.port if port is None else port
        self.object_type = self.object_type if object_type is None else object_type
        self.token = self.token if token is None else token

    def build_inicial_payload(self):
        return {
            'token': self.token
        }

    def build_config_url(self, operation, object_path):
        return self.build_url(protocol=self.protocol, host=self.host, port=self.port, object_type=self.object_type,
                              operation=operation, object_path=object_path)


class DefaultConfigGitSrvHTTPEpConsumer(ConfigGitSrvHTTPEpConsumer):
    protocol = GIT_SERVER_HTTP_ENDPOINT_CONFIG.get('protocol')
    host = GIT_SERVER_HTTP_ENDPOINT_CONFIG.get('host')
    port = GIT_SERVER_HTTP_ENDPOINT_CONFIG.get('port')
    token = GIT_SERVER_HTTP_ENDPOINT_CONFIG.get('token')


class NamespaceGitSrvHTTPEpConsumer(DefaultConfigGitSrvHTTPEpConsumer):
    object_type = 'ns'

    def build_call_url(self, operation, namespace):
        url = self.build_config_url(operation=operation, object_path=namespace)
        if not url.endswith("/"):
            url += "/"
        return url

    def create_call(self, namespace):
        payload = self.build_inicial_payload()
        url = self.build_call_url(operation=self.create_operation, namespace=namespace)
        return self.post_url(url=url, payload=payload)

    def edit_call(self, namespace, new_namespace):
        payload = self.build_inicial_payload()
        payload['new_namespace'] = new_namespace
        url = self.build_call_url(operation=self.edit_operation, namespace=namespace)
        return self.post_url(url=url, payload=payload)


class RepositoryGitSrvHTTPEpConsumer(DefaultConfigGitSrvHTTPEpConsumer):
    object_type = 'repo'

    def build_object_path(self, namespace, repository):
        return "%s/%s" % (namespace, repository)

    def build_call_url(self, operation, namespace, repository):
        object_path = self.build_object_path(namespace=namespace, repository=repository)
        url = self.build_config_url(operation=operation, object_path=object_path)
        if not url.endswith("/"):
            url += "/"
        return url

    def create_call(self, namespace, repository):
        payload = self.build_inicial_payload()
        url = self.build_call_url(operation=self.create_operation, namespace=namespace, repository=repository)
        return self.post_url(url=url, payload=payload)

    def edit_call(self, namespace, repository, new_repository, new_namespace=None):
        payload = self.build_inicial_payload()
        if new_namespace is not None:
            payload['new_namespace'] = new_namespace
        payload['new_repository'] = new_repository
        url = self.build_call_url(operation=self.edit_operation, namespace=namespace, repository=repository)
        return self.post_url(url=url, payload=payload)


class FileGitSrvHTTPEpConsumer(DefaultConfigGitSrvHTTPEpConsumer):
    object_type = 'file'

    def build_object_path(self, namespace, repository, file_path):
        return "%s/%s/%s" % (namespace, repository, file_path)

    def build_call_url(self, operation, namespace, repository, file_path):
        object_path = self.build_object_path(namespace=namespace, repository=repository, file_path=file_path)
        return self.build_config_url(operation=operation, object_path=object_path)

    def create_call(self, namespace, repository, file_path, commit=True):
        payload = self.build_inicial_payload()
        payload['commit'] = commit
        url = self.build_call_url(operation=self.create_operation, namespace=namespace, repository=repository,
                                  file_path=file_path)
        return self.post_url(url=url, payload=payload)

    def edit_mv_call(self, namespace, repository, file_path, new_file_path, new_repository=None, new_namespace=None):
        payload = self.build_inicial_payload()
        if new_namespace is not None:
            payload['new_namespace'] = new_namespace
        if new_repository is not None:
            payload['new_repository'] = new_repository
        payload['new_file_path'] = new_file_path
        url = self.build_call_url(operation=self.edit_mv_operation, namespace=namespace, repository=repository,
                                  file_path=file_path)
        return self.post_url(url=url, payload=payload)

    def edit_contents_call(self, namespace, repository, file_path, contents):
        payload = self.build_inicial_payload()
        payload['contents'] = contents
        url = self.build_call_url(operation=self.edit_contents_operation, namespace=namespace, repository=repository,
                                  file_path=file_path)
        return self.post_url(url=url, payload=payload)

    def create_and_edit_contents_call(self, namespace, repository, file_path, contents):
        return [
            self.create_call(namespace=namespace, repository=repository, file_path=file_path, commit=False),
            self.edit_contents_call(namespace=namespace, repository=repository, file_path=file_path, contents=contents)
        ]


# to test: set test to True and run "python manage.py -c "import ideApp.git_server_http_endpoint"
test = False
if test:
    namespace = 'nishedcob3'
    repository = 'test2'
    file_path = 'folder/test2.py'

    # Example File Creation Call:
    file_consumer = FileGitSrvHTTPEpConsumer()
    file_create = file_consumer.create_call(namespace=namespace, repository=repository, file_path=file_path,
                                            commit=False)

    print('url: %s' % file_create.request.url)
    print('data: %s' % file_create.request.body)

    contents = "print('hello world')\n"

    # Example File Edit Call:
    file_edit = file_consumer.edit_contents_call(namespace=namespace, repository=repository, file_path=file_path,
                                                 contents=contents)

    print('url: %s' % file_edit.request.url)
    print('data: %s' % file_edit.request.body)

    file_path = 'folder/test3.py'

    file_create_and_edit = file_consumer.create_and_edit_contents_call(namespace=namespace, repository=repository,
                                                                       file_path=file_path, contents=contents)
    print("FILE CREATE")
    file_create = file_create_and_edit[0]
    print('url: %s' % file_create.request.url)
    print('data: %s' % file_create.request.body)

    print("FILE EDIT")
    file_edit = file_create_and_edit[1]
    print('url: %s' % file_edit.request.url)
    print('data: %s' % file_edit.request.body)
\end{lstlisting}
\lstset{language=Bash}
