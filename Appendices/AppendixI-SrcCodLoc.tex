% Appendix I - Source Code Locations

\chapter{Ubicacion de Repositorios de Codigo y Sus Licencias}
\label{AnexoI} 

\section{Ubicacion de Codigo Fuente}
Se puede encontrar el codigo fuente, divido en repositorios de este trabajo de titulacion en la siguiente direcciones con las licencias respectivas:
\begin{description}
	\item[GitEDU y EduNube] Servicios Principales para Editar y Ejecutar codigo en Linea.\\
    Repositorios y Espejos:
    \begin{description}
    	\item[GitLab] \url{https://gitlab.com/nishedcob/GitEDU}
        \item[GitHub] \url{https://gitlab.com/nishedcob/GitEDU}
    \end{description}
    Submodules y Directorios:
    \begin{description}
    	\item[3rd\_party/] Directorio con Librerias de Terceros ubicado de tal forma que GitHub no lo cuenta en los estadisticas del repositorio.
        \item[EduNube/] Directorio que forma la raiz del proyecto de Django que es el servicio EduNube.
        \item[GitEDU/] Directorio que forma la raiz del proyecto de Django que es el servicio GitEDU.
        \item[GitEDU-forked-libs] Submodulo que agrupa librerias forkeadas. \\
        Repositorios y Espejos:
        \begin{description}
        	\item[GitLab] \sloppy \url{https://gitlab.com/nishedcob/Thesis-GE-Libraries}
            \item[GitHub] \sloppy \url{https://github.com/nishedcob/Thesis-GE-Libraries}
        \end{description}
    	Submodulos:
        \begin{description}
        	\item[django-app-lti] Liberia Base para LTI en Django, forkeado y actualizado para funcionar con Python 3.\\
            Licencia: \textbf{Harvard Copyright License}. \\
            Repositorios y Espejos:
            \begin{description}
                \item[GitLab] \sloppy \url{https://gitlab.com/nishedcob/django-app-lti}
                \item[GitHub] \sloppy \url{https://github.com/nishedcob/django-app-lti}
            \end{description}
            \item[django-auth-lti] Backend de Autenticacion LTI en Django, forkeado y actualizado para funcionar con Python 3.\\
            Licencia: \textbf{Apache 2.0}. \\
            Repositorios y Espejos:
            \begin{description}
                \item[GitLab] \sloppy \url{https://gitlab.com/nishedcob/django-auth-lti}
                \item[GitHub] \sloppy \url{https://github.com/nishedcob/django-auth-lti}
            \end{description}
            \item[ims\_lti\_py] Libreria de LTI para Python, forkeado y actualizado para funcionar con Python 3.\\
            Licencia: \textbf{Top Hat Monocle Corp. Copyright License}. \\
            Repositorios y Espejos:
            \begin{description}
                \item[GitLab] \sloppy \url{https://gitlab.com/nishedcob/ims_lti_py}
                \item[GitHub] \sloppy \url{https://github.com/nishedcob/ims_lti_py}
            \end{description}
            \item[python-oauth2] Libreria de Python para cumplir con los estandares de OAuth, forkeado y actualizado para funcionar con Python 3.
            \\
            Licencia: \textbf{MIT}. \\
            Repositorios y Espejos:
            \begin{description}
                \item[GitLab] \sloppy \url{https://gitlab.com/nishedcob/python-oauth2}
                \item[GitHub] \sloppy \url{https://github.com/nishedcob/python-oauth2}
            \end{description}
        \end{description}
        \item[GitServerHTTPEndpoint] Submodule para el Servicio GitServerHTTPEndpoint. \\
        Licencia: \textbf{GNU General Public License v3.0}. \\
        Repositorios y Espejos:
        \begin{description}
        	\item[GitLab] \sloppy \url{https://gitlab.com/nishedcob/GitServerHTTPEndpoint}
            \item[GitHub] \sloppy \url{https://github.com/nishedcob/GitServerHTTPEndpoint}
        \end{description}
        \item[UTPL-Engineering-Thesis] el codigo/proyecto de LaTeX para generar este documento. \\
        Licencia: \textbf{Creative Commons Attribution Non Commercial Share Alike 4.0}. \\
        Repositorios y Espejos:
        \begin{description}
        	\item[GitLab] \sloppy \url{https://gitlab.com/nishedcob/UTPL-Engineering-Thesis}
            \item[GitHub] \sloppy \url{https://github.com/nishedcob/UTPL-Engineering-Thesis}
        \end{description}
        \item[db/] Directorio desactualizado con algunos archivos relacionados a la base de datos tanto relacional como no relacional.
        \item[docker/] Directorio con todos los archivos relacionados a la creacion de los seis imagenes de Docker utilizados en el proyecto.
        \item[kubernetes/] Directorio con los expirimentos realizados con Kubernetes. Tambien contiene los submodulos detallados a continuacion:
        \begin{description}
        	\item[postgresql-code-executor-template] Plantilla de Ejecuccion para uso con imagenes de Docker \texttt{postgresql-executor} dentro de su volumen \texttt{/code}.\\
            Licencia: \textbf{GNU General Public License v3.0}. \\
            Repositorios y Espejos:
            \begin{description}
            	\item[GitLab] \sloppy \url{https://gitlab.com/nishedcob/postgresql-code-executor-template}
                \item[GitHub] \sloppy \url{https://github.com/nishedcob/postgresql-code-executor-template}
            \end{description}
        	\item[python3-code-executor-template] Plantilla de Ejecuccion para uso con imagenes de Docker \texttt{python3-executor} dentro de su volumen \texttt{/code}.\\
            Licencia: \textbf{GNU General Public License v3.0}. \\
            Repositorios y Espejos:
            \begin{description}
            	\item[GitLab] \sloppy \url{https://gitlab.com/nishedcob/python3-code-executor-template}
                \item[GitHub] \sloppy \url{https://github.com/nishedcob/python3-code-executor-template}
            \end{description}
        	\item[shell-code-executor-template] Plantilla de Ejecuccion para uso con imagenes de Docker \texttt{shell-executor} dentro de su volumen \texttt{/code}.\\
            Licencia: \textbf{GNU General Public License v3.0}. \\
            Repositorios y Espejos:
            \begin{description}
            	\item[GitLab] \sloppy \url{https://gitlab.com/nishedcob/shell-code-executor-template}
                \item[GitHub] \sloppy \url{https://github.com/nishedcob/shell-code-executor-template}
            \end{description}
        \end{description}
    \end{description}
    Se puede clonar todo el repositorio y submodules con el commando:
    \begin{lstlisting}[breaklines]
git clone --recursive https://gitlab.com/nishedcob/GitEDU.git
    \end{lstlisting}
    O en repositorios ya existentes, siempre se puede actualizar submodulos con el comando:
    \begin{lstlisting}
git submodule update --init --recursive --remote
    \end{lstlisting}
\end{description}
