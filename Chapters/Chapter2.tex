% Chapter 2

\chapter{Estado del Arte}
\label{capitulo2}

\section{Trabajos Relacionados}
Dentro de la temática de este trabajo de titulación es importante entender trabajos relacionados los mismos que han sido realizados para en base a ellos entender investigaciones que ya se han realizado y de esta forma aprender de ellos.

\subsection{GitEduERP}
En un trabajo reciente del autor con sus colegas, frente el problema de necesitar ofrecer una ambiente de programación a estudiantes en línea, se realizó un editor de código en línea con sistema de permisos y la capacidad de compartir entre usuarios en un backend de Django y utilizando una librería ACE liberado por Cloud9 IDE que guardaba código editado en una instancia de GitLab CE. Además ofrece chat en línea con una liberia TogetherJS [UTPL-GitEduERP].

\subsection{Sistema de Encuestas Online}
Para mejorar temas de business analytics, se trató de diseñar procesos de negocio para realizar colección de datos en tiempo real a través de encuestas y analizar las mismas con un fin de ayudar la toma de decisiones estratégicas de negocio en tiempo real. Plantearon soluciones con un fin de optimizar tiempos y recursos a través del uso de soluciones tecnológicas [UTPL-Thesis-Encuestas-Online].

\subsection{Metodología de Enseñanza con la Web 2.0}
Se buscó utilizar la manera en que la moda de lectura-escritura en la Web 2.0 se podría crear cursos interactivos con estudiantes online como base de una nueva metodología de enseñanza. Destaca esos temas desde el punto de vista de ingeniería en sistemas para proporcionar soluciones netamente técnicas y estratégicas para dar el mejor aporte posible a quienes deseen implementar un sistema de este tipo [UTPL-Thesis-Edu-Web-2.0].

\subsection{Xen Web-based Terminal for Learning Virtualization and Cloud Computing Management}
Los autores, Abdullah Almurayh y Sudhanshu Semwal, propusieron e implementaron una arquitectura de cliente-servidor para la distribución de recursos educativos y en el proceso enseñar a estudiantes de Linux, programación orientada a la nube y gestión de nubes/servidores. Su aplicación web ofrece terminales SSH donde cada estudiante y docente disponía de una máquina virtual de tal forma que tenían su propio ambiente con permisos de superusuario y al mismo tiempo eran aislados de la infraestructura real y de los demás usuarios, motivo por el cual se podía dar una solución flexible y a su vez segura. Como hipervisor ocuparon un sistema de Xen que controlan remotamente con su servidor de aplicación (servidor web) [almurayh2014xen].

\subsection{Comparación de Trabajos Relacionados}
\begin{tabular}{|p|p|p|p|p|p|}
	\hline
		& Editar Código en Línea & Persistir Código en Línea & Recolección y Análisis de Datos para decisiones estratégicas en tiempo real & Metodología de Enseñanza Online & Ambientes Virtualizados \\
	\hline
    GitEduERP & x & x & & & \\
    \hline
    Sistema de Encuestas Online & & & x & & \\
    \hline
    Metodología de Enseñanza con la Web 2.0 & & & & x & \\
    \hline
    Xen Web-based Terminal for Learning Virtualization & x & x &  & x & x \hline
\end{tabular}

\section{Sistemas Similares}
\subsection{Repl.it}
\subsection{io.livecode.ch}
\subsection{Cloud9 IDE}
\subsection{GitLab}
\subsection{OverLeaf}
\subsection{Google Drive}
\subsection{Comparación de Sistemas Similares}

\section{Marco Teorico}

\subsection{Aspectos de Propiedad Intelectual}

\subsubsection{Software Libre}

\subsection{Aspectos Ambientales del Entorno de Desarrollo}

\subsubsection{Tipos de Recurso de Aprendizaje}

\subsubsection{MOOC}
\subsubsection{CMS}
\subsubsection{LMS}
\subsubsection{LTI}

\subsubsection{Sistema de Control de Versionamiento}
\subsubsection{Git}

\subsubsection{Virtualización}
\subsubsection{Hipervisores}

\subsubsection{Protocolo de Túnel}

\subsubsection{Criptografía}

\subsubsection{SSH}

\subsubsection{SSL}

\subsubsection{URI}

\subsubsection{HTTP}
