% Chapter 6

\chapter{Despliegue}
\label{capitulo6}

\section{Plan de Despliegue}
En el curso del desarrollo, depuracion y ejecuccion de pruebas llegó a ser evidente en enorme consumo de recursos necesarios para levantar todos los servicios desarrollados y auxiliares a la vez, razon por el cual, se ha visto la necesidad de un rediseño de forma mas liviana de la manera en que se despliegue todos los componentes integrados a la aplicacion para con ello terminar las fases mencionadas anteriormente y dar paso a ejecuccion en ambientes de pocos recursos, tanto en el ambito de desarrollo como de produccion. En la implementacion original de la arquitectura fisica para el desarrollo, se habia planteado una colleccion de maquinas virtuales quienes representarian servidores distintas dentro de una red, pero para temas de optimizar recursos, se ha realizado los siguientes cambios los cuales se documentan a lo largo de este capitulo:
\begin{itemize}
	\item Kubernetes (backend de virtualizacion) se ubica en una maquina virtual de Xen (con paravirtualizacion activa para aumentar el rendimiento del sistema virtualizado) dado que el mismo, por temas de seguridad, debe seguir de una forma aislada. De esta forma se simula que esta en un servidor aparte, sea fisico o virtual (se recomienda mejor virtualizar todo el cluster final de Kubernetes en algun hipervisor de tipo 1 para garantizar mayor seguridad y aislamiento) pero sin la mayor parte de las perdidas de rendimiento que se dio con MiniKube (alojado en VirtualBox).
    \item Los servicios auxiliares en lugar de ser maquinas virtuales de Xen ahora son contenedores de Docker.
    \item Los servicios desarrollados se los han convertido en servicios (Systemd) del sistema operativo, no tanto por temas de rendimiento (aunque se podria mejorar el rendimiento de esta forma, asignandoles a un usuario con mayor prioridad de ejecuccion como el usuario root\footnote{Pero realizarlo esta afuera del alcance de este tesis, tampoco puede cambiar los recursos fisicos de la maquina.}\footnote{Nunca se debe asignar un servicio que se consume en la red externa a algun usuario con privilegios de superusuario, como root, ya que el mismo abre todo el sistema operativo a un nuevo vector de ataque por el mismo servicio. En este caso debe ser un nuevo usuario, preferiblemente uno por cada servicio, con acceso restingido que tiene mayor prioridad de ejecucion para sus procesos.}) si no por temas de facilitar la administracion del mismo.
    \item La integracion de los servicios con NGinX para que el mismo puede protegerlos y operar en la capacidad de proxy inversa, proxy de terminacion SSL/TLS, servidor de archivos estaticos y validador de peticiones sin mayor perdida de rendimiento.
\end{itemize}
% TODO
% diagrama de red/fisica

\section{Preperaciones del Ambiente de Despliegue}
% TODO
% asignacion de dirreciones ip/puertos

\subsubsection{Moodle en Docker}
% TODO
%docker pull mysql
%docker pull jauer/moodle
%docker run -d --name moodledb -p 3306:3306 -v /srv/moodle/mysql:/var/lib/mysql -e MYSQL_DATABASE=moodle -e MYSQL_ROOT_PASSWORD=moodle -e MYSQL_USER=moodle -e MYSQL_PASSWORD=moodle --restart always mysql
%docker run -d -P --name moodle --link moodledb:DB -e MOODLE_URL=http://10.10.10.1:8201 -p 8201:80 -v /srv/moodle/data:/var/moodledata --restart always jhardison/moodle
%visit http://10.10.10.1:8201/

\subsubsection{GitLab en Docker}
% TODO
%https://hub.docker.com/r/gitlab/gitlab-ce/
%https://docs.gitlab.com/omnibus/docker/
%docker pull gitlab/gitlab-ce
%docker run --detach --hostname 10.10.10.1 --publish 8143:443 --publish 8101:80 --publish 8122:22 --name gitlab --restart always --volume /srv/gitlab/config:/etc/gitlab --volume /srv/gitlab/logs:/var/log/gitlab --volume /srv/gitlab/data:/var/opt/gitlab gitlab/gitlab-ce:latest
% visit http://10.10.10.1:8101/
% user: root

\index{Hipervisor} \index{Virtualización} \index{Contenedor}
\paragraph{Máquina Virtual de Xen}
% TODO
\subsubsection{Construción de Servidor Virtualizado para Cluster de Kubernetes}
% TODO
% xen-create-image --hostname=debian-k8s-master --ip=10.10.10.12 --netmask=255.255.255.0 --gateway=10.10.10.1 --memory=2048mb --vcpus=2 --lvm=Xephyr-VG --pygrub --dist=stretch --force --size=10240mb --swap=1024mb
% see Screenshot_2017-12-29_15-21-49.png
\paragraph{Instalación de Docker}
% TODO
% apt update
% apt upgrade
% apt install curl
% curl -fsSL get.docker.com -o get-docker.sh
% sh get-docker.sh
% see Screenshot_2017-12-29_15-31-49.png
\paragraph{Instalación de Cluster de Kubernetes}
% TODO
%https://kubernetes.io/docs/setup/independent/install-kubeadm/
%apt-get update && apt-get install -y apt-transport-https
%curl -s https://packages.cloud.google.com/apt/doc/apt-key.gpg | apt-key add -
%cat <<EOF >/etc/apt/sources.list.d/kubernetes.list
%deb http://apt.kubernetes.io/ kubernetes-xenial main
%EOF
%apt-get update
%apt-get install -y kubelet kubeadm kubectl

%# Basic Init Master
%kubeadm init
%# Tear down cluster
%kubeadm reset
%# Init Master for Flannel
%kubeadm init --pod-network-cidr=10.244.0.0/16
%INIT
% K8s Info
%%%%%%%%%%%%%%%%%%%%%%%%%%%%%%%%%%%%%%%%%%%%%%%%%%%%%%%%%%%%%%%%%%%%%%%%%%%%
%To start using your cluster, you need to run the following as a regular user:

%  mkdir -p $HOME/.kube
%  sudo cp -i /etc/kubernetes/admin.conf $HOME/.kube/config
%  sudo chown $(id -u):$(id -g) $HOME/.kube/config

%You should now deploy a pod network to the cluster.
%Run "kubectl apply -f [podnetwork].yaml" with one of the options listed at:
%  https://kubernetes.io/docs/concepts/cluster-administration/addons/

%You can now join any number of machines by running the following on each node
%as root:

%  kubeadm join --token 655cb5.2275aa7df206fe69 10.10.10.12:6443 --discovery-token-ca-cert-hash sha256:4919df120063c4535fd03e909ce11dfe9e6448f8a767be914e86b16660d267c8
%%%%%%%%%%%%%%%%%%%%%%%%%%%%%%%%%%%%%%%%%%%%%%%%%%%%%%%%%%%%%%%%%%%%%%%%%%%%

%# permit execution on master node
%KUBECONFIG=/etc/kubernetes/admin.conf kubectl taint nodes --all node-role.kubernetes.io/master-

%# save credentials to home folder
%mkdir -p $HOME/.kube
%cp -i /etc/kubernetes/admin.conf $HOME/.kube/config
%chown $(id -u):$(id -g) $HOME/.kube/config

%# test connection
%kubectl version

%#https://kubernetes.io/docs/concepts/cluster-administration/networking/
%# install pod network (Flannel Driver)
%sysctl net.bridge.bridge-nf-call-iptables=1
%cat sysctl.conf
%cat >> /etc/sysctl.conf << EOF
%
%# For Kubectl Flannel
%net.bridge.bridge-nf-call-iptables = 1
%
%EOF
%
%apt install net-tools
%apt install git # necesario para clonar repositorios dentro del cluster, si no se instala, solo job/pi y pod/utility funcionaran
%# check that sshd is running/listening on external interface (port 22 is normal)
%# otherwise apt install openssh-server
%netstat -tupln

%vim.tiny /etc/ssh/sshd_config
%PermitRootLogin yes

%systemctl restart sshd

%# from host:
%scp root@10.10.10.12:/etc/kubernetes/admin.conf .
%kubectl --kubeconfig ./admin.conf get nodes

%mv admin.conf k8s.xen.master.admin.conf
%cp k8s.xen.master.admin.conf ~/.kube/config.xen
%cp ~/.kube/config ~/.kube/config.minikube
%cp ~/.kube/config.xen ~/.kube/config
%kubectl version
\subparagraph{Validación de Cluster de Kubernetes}
%cd kubernetes/
%kubectl create -f debian-pod.yaml
%kubectl get pods/utility
%kubectl describe pods/utility
%kubectl create -f debian-pod-2.yaml
%for manifest in `ls *.json`; do
%    kubectl create -f $manifest;
%done
%kubectl get jobs
%# not all will succeed, some point to git repos (http://192.168.99.1) that only minikube has access to
%watch -n 15 "kubectl get jobs"
%# First to finish:
%kubectl describe jobs/pi
%# Pod Created: pi-kf8zv
%kubectl describe pods/pi-kf8zv
%# see output of job
%kubectl logs jobs/pi
\index{Contenedor} \index{Virtualización} \index{Hipervisor}

\section{Despliegue}
% TODO:
% servicios y su asignacion de puertos
% GitEDU - 8000/HTTP = gitedu.localhost
% EduNube - 8001/HTTP = edunube.localhost
% GitServerHTTPEndpoint - 8002/HTTP = githttp.localhost
% aux services (redir NGinX):
% 

\subsection{GitEDU}

\subsection{EduNube}

\subsection{GitServerHTTPEndpoint}

\section{Pruebas del Despliegue y Resultados}

