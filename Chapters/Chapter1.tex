% Chapter 1

\chapter{Introducción}
\label{capitulo1}

%\chapter{Chapter Title Here} % Main chapter title
%
%\label{Chapter1} % For referencing the chapter elsewhere, use \ref{Chapter1} 

%----------------------------------------------------------------------------------------

% Define some commands to keep the formatting separated from the content 
\newcommand{\keyword}[1]{\textbf{#1}}
\newcommand{\tabhead}[1]{\textbf{#1}}
\newcommand{\code}[1]{\texttt{#1}}
\newcommand{\file}[1]{\texttt{\bfseries#1}}
\newcommand{\option}[1]{\texttt{\itshape#1}}

%----------------------------------------------------------------------------------------

Hoy en día se encuentra una necesidad creciente para programadores, informáticos y personas que pueden leer y entender código debido a una tendencia a tratar de siempre automatizar a un mayor grado las tareas humanas con la tecnología. Actualmente la Universidad Técnica Particular de Loja cuenta con métodos clásicos y manuales para enseñar y calificar código que estudiantes programan lo cual no es beneficioso para los docentes, quienes pierden mucho tiempo realizando tareas que se podría realizar de forma automática, ni estudiantes quienes tengan menos atención de sus docentes debido a que tienen mucho código que calificar.
 
Actualmente, en el mercado existen varios sistemas que permiten escribir, ejecutar y probar código en línea. Uno de estos es Repl.it que puede integrase con sistemas que disponen del protocolo de integración entre sistemas de aprendizaje LTI \index{LTI} \citep{Repl.it-Home} y puede calificar el código de estudiantes en base a pruebas unitarias pero el mismo tiene un alto coste que no se escala bien y no está orientando tanto al aprendizaje o colaboración si no a evaluación \citep{Lopez-Jorge}. Por otro lado, Io.livecode.ch es un sistema libre que utiliza contenedores de Docker para permitir la ejecución de código en línea, pero el problema de este sistema es que solo está orientado a enseñanza de programación mas no a realizar pruebas en línea y además no implementa nada de seguridad para proteger el servidor contra usuarios maliciosos \citep{io.livecode.ch}. Cloud9 IDE es otra herramienta popular para programar en línea, éste integra máquinas virtuales de Ubuntu con el mismo para permitir hacer pruebas en tiempo real \citep{Cloud9-Home}, pero el mismo tampoco es factible por su falta de interoperabilidad con LTI \index{LTI} y falta de enfoque ni en enseñanza ni en evaluación y calificación. 
 
Como cualquier otro tipo de institución con fines comerciales, las universidades e instituciones educativas siempre deben estar buscando la manera en que pueden mantener su competitividad para dar la mejor educación posible y al menor costo. La tecnología de hoy en día ha alcanzado una madurez en donde puede apoyar en temas de automatizar algunos procesos dentro de la enseñanza, evaluación y calificación de estudiantes y de la misma forma liberar el tiempo de professores para que estos puedan ayudar a aquellos estudiantes que realmente necesitan un poco más de apoyo.
 
Frente a este problema, se propone un sistema para editar código en línea que a su vez integra LMS \index{LMS} externos (para autenticación y notas), un servidor de control de versiones externo (para la persistencia de código), y un servicio web de ejecución de código en línea de una forma segura, eficaz y eficiente (para dar un ambiente de ejecución y pruebas tanto para los usuarios del sistema como para calificar de forma automática). A continuación se presenta la problemática para definir el contexto del tema frente al cual se espera resolver dentro de este trabajo de titulación. 

\section{Problemática}
Dentro de la universidad, existe la necesidad de contar con una mejor forma de enseñar y evaluar conocimientos de programación en los estudiantes de las titulaciones de Sistemas Informáticos y Computación y Electrónica, especialmente quienes estudian a distancia. Para solucionar este problema, existen sistemas alternativos que se analizan en el capítulo 2, pero estos resultan demasiado costosos para su implementación completa con los recursos actualmente disponibles.
 
Se considera que cualquier sistema implementado como solución debe ser capaz de captar información acerca de su propio uso con un fin de ayudar con la administración del mismo, debe guardar de manera segura y confiable el código escrito dentro del mismo. Así mismo, debe proporcionar de las herramientas que requieren los estudiantes y sus profesores para desarrollar, probar y calificar de manera eficaz, segura y eficiente el código escrito en línea. Además como plataforma en línea se considera que se podría aprovechar para temas de colaboración como: programación en pares y otras técnicas que apliquen los equipos de desarrollo para cumplir con sus responsabilidades de una forma paralela.

\section{Metodología}
La fase de investigación consiste en realizar una breve recopilación de lo que hay dentro del entorno de despliegue para realizar un estudio del mismo. Además se realiza un estudio de aquellas tecnologías que se utiliza dentro de la solución planteada. Para la investigación de todos los puntos anteriores, se basa en fuentes confiables como: fuentes académicos o comerciales con un fin de explicar alguna tecnología para vender la misma.
 
Después de la fase de investigación, donde se formula buenas bases y entendimiento del contexto del ambiente en el que se encuentra el presente trabajo, se empieza una fase de análisis el cual se basa en un documento de visión que desglosa las necesidades existentes para el sistema, seguidamente un ERS o especificación de requerimientos demuestra de forma más exacta las capacidades que debe tener el sistema.
 
Una vez que se tiene entendidas las capacidades que debe tener el sistema, se procede a una fase de diseño donde se realiza diagramas para definir la arquitectura, despliegue e interacción que el sistema tendrá con los usuarios.
 
Con la definición de los requerimientos y el diseño del sistema como tal, lo que sigue es una fase de desarrollo la misma que se plantea llevar bajo una metodología de desarrollo iterativo incremental que se caracteriza por la revisión contínua con los stakeholders principales del proyecto. De esta forma se puede mantener un poco de agilidad en el desarrollo y consecuentemente desarrollar un producto que cumpla con las necesidades de los usuarios mientras se mantiene dentro de los límites de tiempo y recursos preestablecidos.
 
Además se propone tener solo iteraciones (basados en entregables) que duren máximo un mes, esto se lo propone para los dos entregables más grandes; el subsistema de ejecución de código por su complejidad y el capítulo seis que se trata del despliegue de la aplicación final. Las otras iteraciones son de dos semanas para tratar de mantener el proyecto en un estado fluido de actividad y presentación de entregables de forma contínua.
 
Las pruebas se llevaran igual en fases, donde primero en una prueba alfa se integra con sistemas reales para probar temas de integración. En base a los resultados de estas pruebas se realiza recomendaciones y mejoras previo a la próxima fase de pruebas que son constituidas por pruebas beta donde se prueba la aplicación en un entorno real con usuarios finales, tanto estudiantes como profesores para que puedan dar retroalimentación previo al despliegue de una versión final.
 
Finalmente se trabaja juntamente con los que serán encargados de mantener la aplicación para el despliegue del mismo. Esto se lleva en fases en donde se levantan primero las dependencias necesarias, como bases de datos y servicios externos entre otros previo al levantamiento y configuración del sistema. La última fase del despliegue consiste en levantar los proxies y servicios auxiliares que ayuden a mejorar temas de seguridad y rendimiento a la aplicación desplegado.

\section{Objetivos}
El objetivo principal es mejorar la enseñanza y evaluación de estudiantes de programación de la Universidad Técnica Particular de Loja. Para cumplir este objetivo se propone un sistema de edicción y prueba de código en línea que ayude a los docentes a calificar código que escriben sus estudiantes en base a pruebas unitarias e integra funcionalidades de sistemas LMS \index{LMS} institucionales en adición a servidores de control de versiones institucionales. Para alcanzar este fin, también se plantea cumplir con algunos objetivos específicos en vía al principal. \index{Editar Código|textit}

\begin{itemize}
	\item Facilitar la autenticación de los estudiantes para que puedan entrar directamente desde sistemas LMS \index{LMS} institucionales a través del protocolo LTI\index{LTI}. \index{Autenticación LTI}
	\item \index{Ejecutar Código|(} Proveer la capacidad de poder ejecutar código en línea. Con esta funcionalidad, el sistema se acerca más a un entorno completo de desarrollo donde estudiantes requieren un mínimo de programas instalados en sus equipos personales. \index{Ejecutar Código|)}
	\item \index{Calificar Código|(} Autocalificar código en base a pruebas unitarias. El fin de esta característica es que el sistema sea un apoyo a los docentes en ayudarles automatizar y por lo tanto optimizar su flujo de calificar para que tengan más tiempo para estar enseñando y ayudando sus alumnos. Las mismas pruebas unitarias, por la forma en que automatizan la forma en que se pueden usar para calificar código en base a las funcionalidades que tiene y cómo reacciona frente ciertas entradas, también hace el proceso completamente objetivo debido a que elimina el factor humano del proceso. \index{Calificar Código|)}
	\item \index{Controlar Versiones de Código|(} Utilizar servidores de sistemas de control de versiones de la institución para recolectar código en los mismos. Eso permitirá, de una forma transparente y sin mayor interacción del usuario, ver un historial de código escrito en la plataforma y también servir de respaldo continuo del trabajos que realizan los estudiantes. \index{Persistir Código|textit} \index{Controlar Versiones de Código|)}
	\item \index{Recolección de Datos|(} \index{Toma de Decisiones|(} Recolectar datos para la toma de decisiones estratégicas. Esta característica permite a la administración de la aplicación sea la universidad o equipo de mantenimiento sacar datos y estadísticas de uso de la aplicación para en base a ello realizar análisis de cómo se está siendo usado, que son los comportamientos de los estudiantes y otras decisiones estratégicas que se pueden tomar en base a los datos de uso del sistema. \index{Recolección de Datos|)} \index{Toma de Decisiones|)}
\end{itemize}

\section{Resultados Esperados}
En el presente trabajo de titulación pretende obtener los siguientes resultados:
\begin{description}
	\item[Sistema para Editar Código en Línea] que permita a los estudiantes y sus docentes tener una herramienta para mejorar el proceso de enseñanza, aprendizaje y evaluación de la programación. \index{Editar Código}  
 \item[Sistema para Ejecutar Código en Línea que se integró con lo anterior] y facilita las pruebas que requieren los estudiantes o profesores con respeto a algún código al cual tengan acceso. \index{Ejecutar Código}
 \item[Repositorio de datos recolectados para la toma de decisiones estratégicos] que permita a la titulación ver cómo se está utilizando el sistema y en base a ello planificar decisiones futuras. \index{Recolección de Datos} \index{Toma de Decisiones}
 \item[Manual de Programador] que sea una ayuda para futuros desarrolladores que deseen extender la funcionalidades del sistema inicial.
 \item[Manual de Usuario] que apoya a estudiantes y docentes que quieren aprender a usar el nuevo sistema.
 \item[Manual de Mantenimiento/Administrador] que ayuda al equipo de mantenimiento poder mantener la aplicación al corto, mediano y largo plazo.
\end{description}

\section{Organización del Documento}
% TODO: Expand with additional content %% AFTER FINISHED WITH CONTENT
Dentro del capítulo uno, se introduce el tema y problemática, se organiza el documento y se plantea la metodología y objetivos que guían el proyecto desde su inicio hasta su fin.
 
En el capítulo dos, se ve el estado del arte que contenga un análisis de trabajos relacionados y sistemas similares a la que se plantea desarrollar. También dentro del capítulo dos se encuentra un subsistema del estado del arte en la forma de marco teórico, el cual desglosa la situación actual y permite ver la teoría de las tecnologías de apoyo para el sistema final.
 
En el capítulo tres, se da un análisis profundo del problema para entenderlo de forma completa previa a las fases siguientes. Dentro del mismo capítulo, en base al análisis, se plantea un diseño para el producto final, el cual se refleja más adelante en el desarrollo durante el curso del capítulo cuatro.
 
A lo largo del capítulo cuatro, se lleva al lector por el proceso de desarrollo. Por lo tanto el mismo se divide en los varios módulos que se necesitan desarrollar para el proyecto; un módulo de autenticación LTI \index{LTI} que permite a los estudiantes autenticarse contra sistemas LMS \index{LMS} ya existentes, un módulo de editar código en línea, un módulo de persistencia de código en servidores de control de versiones institucionales, un subsistema de ejecución de código en línea, un módulo de pruebas unitarias para calificaciones automáticas y un módulo para sincronizar calificaciones con sistemas LMS \index{LMS} institucionales.
 
El quinto capítulo se basa en la fase anterior de desarrollo para planificar y realizar pruebas de funcionamiento e integración. De esta manera se encuentran las fallas existentes para resolver los mismos antes de seguir con el despliegue en el sexto capítulo.
 
A continuación, en el capítulo sexto se planifica y se despliega la aplicación final para que pueda empezar su vida útil dentro de la universidad.
 
Dentro del séptimo capítulo se resume los resultados y se concluye el trabajo realizado en base al cumplimiento de los objetivos planteados dentro del capítulo uno. Además se da recomendaciones e ideas para trabajos futuros en base al proyecto realizado y para la evolución continua del prototipo final.

Al final de todos los capítulos anteriores se encuentran los anexos con documentos de apoyo como: el documento de vision (Anexo \ref{AnexoA}) y especificacion de requisitos (Anexo \ref{AnexoB}), implementacion tecnicas llevadas a lo largo del desarollo (Anexos \ref{AnexoC}, \ref{AnexoD}, \ref{AnexoE}), pruebas realizadas para validar funcionalidades (Anexos \ref{AnexoF}, \ref{AnexoG}, \ref{AnexoH}), y ubicacion y licencias de codigo llevado en el curso de la tesis (Anexo \ref{AnexoI}).
