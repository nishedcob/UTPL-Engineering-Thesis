% Chapter 3

\chapter{Analisis y Diseño}
\label{capitulo3}

\section{Analisis}

\subsection{Vision}

\subsection{Especificacion de Requerimientos}

\section{Diseño}
Previo a la implementación de cualquier sistema es importante analizar su funcionamiento y en base a lo mismo realizar un diseño que provee estas funcionalidades que se requiere. Por lo tanto, a continuación se demuestra las fases del diseño realizado desde la definición de sistemas, subsistemas y módulos y la interacción entre ellos, hasta la arquitectura interna y de despliegue.

\subsection{Sistemas, Subsistemas y Módulos}
Debido a la naturaleza del alcance de este trabajo de titulación, se considera que se debe dividir la plataforma en dos sistemas que sean capaces de interactuar entre sí. El primer sistema es la que ofrece el editor de código en línea, integración con los LMS institucionales y a su vez integración con un servidor de control de versiones. El segundo sistema solo se encarga de ejecución de código en línea a través de algún tipo de virtualización y aislamiento de procesos. De esta forma se puede aislar el sistema que está bajo mayor riesgo de ser atacado (el de ejecución de código arbitrariamente) del sistema con mayor riesgo de impacto negativo (el sistema de escribir código en línea, el cual también se encarga de las calificaciones). A continuación se presenta la interacción entre los sistemas y actores en el Diagrama de Contexto de Sistemas.

% figure 

Los tres tipos de usuarios que se considera para el sistema son administradores, quienes administran y mantienen todo el sistema en su ambiente de despliegue, docentes, quienes generan contenido didactica para sus estudiantes y califican los mismos motivo por el cual que tienen acceso a los LMS, el sistema de editar codigo (que les permite generar nuevas tareas / exámenes / pruebas / talleres para sus estudiantes y ver el progreso y calificaciones de los mismos) y el sistema de control de versiones (donde puede ver y bajar codigo que escriben estudiantes y ver a un nivel más profundo el historial y evolución del mismo), y estudiantes, quienes tengan acceso a los mismos tres sistemas de los docentes. Los estudiantes acceden al LMS donde vean los tareas / exámenes / pruebas / talleres nuevos que hay y al abrirlos se les lleva, identifica y autentica el sistema de editar código en línea mediante LTI. Los estudiantes realizan sus obligaciones de código, mientras que los mismos se respaldan de forma automática en un servidor de control de versiones independiente de tal forma que se lo puede revisar tanto los estudiantes involucrados como su docente a futuro. Aquello control adicional podría llegar a ser una evidencia importante para la resolución de conflictos entre estudiantes y/o docentes a futuro.

El sistema de editar codigo, apartir de aqui conocido simplemente bajo el nombre GitEDU, controla remotamente por medio de comandos, mediante posiblemente una mezcla de protocolos (SSH, HTTP sobre SSH [SHTTP] y HTTPS), dependiendo de cómo se da el caso, un sistema de ejecucion de código, conocido a partir de aquí en adelante como EduNube, para señalar acciones que debe tomar el segundo y/o datos que necesita sincronizar con la primera. Al mismo tiempo, GitEDU comunica con el servidor de control de versiones mediante su API externa (una API REST) sobre HTTP o HTTPS para controlar el mismo y prepararle para flujos continuos de código editado que puede viajar por una variedad de protocolos como SHTTP, HTTPS, HTTP, SFTP, FTP o FTPS como requiere la situación y ambiente de despliegue. Siempre y cuando EduNube requiere de codigo de algun usuario para realizar una calificación o algún otro tipo de interacción con un usuario que requiere probar el mismo, primero se procede a descargar el código involucrado desde el servidor de control de versiones sobre uno de los protocolos mencionado previamente para el mismo. En base a la naturaleza de las operaciones que está haciendo, comunica sus resultados con GitEDU para que puede manejar las mismas de la forma adecuada. En el caso de notas, GitEDU se encarga de sincronizar las mismas con los LMS institucionales con los cuales está integrado mediante el protocolo LTI.

\subsubsection{GitEdu}

% figure

\paragraph{Subsistema de Gestión de Usuarios}
El subsistema de gestión de usuarios de encarga de temas de seguridad y acceso por usuarios a varios componentes de la aplicación. Para lo mismo autentica usuarios mediante dos módulos (uno clásico y el otro mediante el protocolo LTI) y lleva un sistema de permisos orientado a usuarios, grupos y roles.

\subparagraph{Módulo de Autenticación Clásica de Usuarios}
El módulo de autenticación clásica de usuarios permite que los usuarios finales pueden iniciarse session mediante un usuario y contraseña.

\subparagraph{Módulo de Autenticación por LTI}
El módulo de autenticación por LTI permite que usuarios finales puede iniciar su sesión mediante una sesión existente en algún sistema externa que soporta LTI.

\subparagraph{Módulo de Grupos de Usuarios}
El módulo de grupos de usuarios permite definir grupos específicos de usuarios cuyo membresía les cambia el nivel y naturaleza del acceso que tengan a la aplicación.

\subparagraph{Módulo de Permisos de Usuarios y Grupos}
El módulo de permisos de usuarios y grupos define la naturaleza de acceso que tengan usuarios autorizados y los grupos a los cuales pertenecen dentro de la aplicación.

\subparagraph{Módulo de Roles de Usuarios y Grupos}
El módulo de roles de usuarios y grupos define las responsabilidades de varios grupos de usuarios y usuarios específicos y en base a los mismos les asigna permisos adecuados tanto para proteger la aplicación contra uso y accesos indebidos y al mismo tiempo permitir a los usuarios poder realizar sus actividades normalmente sin impedimentos.

\paragraph{Subsistema de Editar Código}
El subsistema de editar codigo forma el núcleo de la funcionalidad del sistema GitEdu. Eso lo realiza a través de funcionalidades de editar codigo, persistirlo y permitir la interacción adecuada y social entre usuarios.

\subparagraph{Modulo de Editar Codigo}
El módulo de editar código en línea ofrece funcionalidades básicas de un editor de código (IDE) en línea.

\subparagraph{Modulo de Persistencia de Codigo}
El módulo de persistencia de código persiste codigo versionado en servidores de control de versiones externas que usuarios escriben y editan, y al momento de necesitar abrir un código previo, recupera el mismo de los servidores de control de versiones externas para que puede seguir editando lo mismo.

\subparagraph{Módulo de Control de Acceso de Código}
El módulo de control de acceso de código permite a usuarios definir y modificar los permisos que tiene su código para controlar la naturaleza del acceso que tienen otros usuarios, grupos y roles.

\subparagraph{Modulo de Compartir}
El módulo de compartir permite compartir el acceso al código con otros usuarios y grupos.

\subparagraph{Módulo de Interacción Social}
El módulo de interacción social permite que usuarios pueden interactuar en tiempo real mientras que editan juntos el código.

\paragraph{Subsistema de Docentes}
El subsistema de docentes ofrece muchas de las funcionalidades que requieren los docentes para llevar exitosamente su materia de programación. Estas funcionalidades incluyen gestión de estudiantes, materias y notas. La gestión de materias y notas también involucra un gestión y calificación de talleres, deberes, pruebas y exámenes. Además por el hecho de que se genera nuevas notas y calificaciones a través de GitEdu, también se encarga de sincronizar notas con otras sistemas externas como los LMS a través de LTI.

\subparagraph{Modulo de Gestion de Estudiantes}
El módulo de gestión de estudiantes permite que docentes pueden ver notas de un estudiante suyo. Además estudiantes pueden revisar sus notas y tanto administradores como docentes pueden asignar estudiantes a aulas o cambiar o eliminar asignaciones previas.

\subparagraph{Módulo de Gestión de Materias}
El módulo de gestión de materias permite a docentes y administradores ver estadísticas de una materia, la creación de grupos de estudiantes dentro de una materia (por ejemplo para un proyecto grupal) tanto por parte de un docente o por parte de un estudiante, modificar grupos de estudiantes (por ejemplo botar un integrante), y asignar talleres, deberes, pruebas o exámenes a un estudiante, grupo o materia. Además dentro de la gestión de materias, docentes pueden establecer el peso de cada taller, deber, prueba y examen para en base a ello calcular los promedio de un estudiante.

\subparagraph{Módulo de Talleres, Deberes, Pruebas y Exámenes}
El módulo de talleres, deberes, pruebas y exámenes permite a docentes definir nuevos talleres, deberes, pruebas, exámenes, etc y los criterios de evaluación de los mismos. Involucrado en eso son pruebas unitarias (si desee calificación automatizada) y documentación que explica lo que hay que hacer al estudiante.

\subparagraph{Módulo de Notas y Sincronización}
El módulo de notas y sincronización gestiona las notas que mantiene GitEdu y se encarga de sincronizar cambios en las mismas. Aunque realiza este trabajo por detrás normalmente, también proporciona una interfaz gráfica para interacción con los docentes respectivos y adecuados. Además por el tema de seguridad y integridad de las notas, se lleva las mismas versionado y con un registro preciso de cuando fueron modificados y por quien.

\paragraph{Subsistema de Ejecución de Código}
El subsistema de ejecución de código se encarga de comunicar con el sistema externa EduNube con un fin de aprovechar las funcionalidades del mismo y proveer funcionalidades de ejecución en línea, aplicación de pruebas unitarias y calificación automática de código sin exponer el sistema EduNube a usuarios finales. Para aumentar la seguridad que logra llevar EduNube, GitEdu communica con lo mismo a través de canales seguros como HTTPS y SSH utilizando métodos de autenticación orientados a una schema de llaves públicas y privadas (criptografía asimétrica) para asegurarse de la identidad de ambos previo al envío de comandos y comunicación de datos y notas.

\subparagraph{Modulo de Comandos}
El módulo de comandos se encarga de controlar el sistema de EduNube a través de su API externa para indicar al mismo cuando debe bajar, ejecutar o calificar algún código.

\subparagraph{Módulo de Comunicación de Datos y Notas}
El módulo de comunicación de datos y notas permite la sincronización de los mismos cuando sea necesario entre EduNube y GitEdu.

\paragraph{Subsistema de Administración}
El subsistema de administración ayuda a administradores de GitEdu recolectar datos acerca del uso del mismo y analizar lo datos en adición a ser capaces de configurar y monitorear el rendimiento del sistema.

\subparagraph{Módulo de Configuración}
El módulo de configuración permite la configuración en tiempo real del sistema GitEdu, por ejemplo conexiones con sistemas externas y gestión de usuarios.

\subparagraph{Modulo de Monitoreo}
El módulo de monitoreo permite ver en tiempo real el consumo del sistema.

\subparagraph{Módulo de Recolección de Datos}
El módulo de recolección de datos lleva un registro continuo del uso del sistema para su revisión y análisis por parte de operadores humanos después tanto para temas de auditoría como toma de decisiones estratégicas.

\subparagraph{Módulo de Análisis de Datos}
El módulo de análisis de datos proporciona las herramientas necesarias para que un operador humano puede interpretar los mismos y sacar sus conclusiones.

\subsubsection{EduNube}

% figure

\paragraph{Subsistema de Gestión de Usuarios}
El subsistema de gestión de usuarios de encarga de temas de seguridad y acceso indirecto por usuarios (a través de GitEdu) y acceso directo por parte de administradores a varios componentes de la aplicación. Para lo mismo autentica usuarios mediante dos módulos (uno clásico solo para administración del mismo y el otro mediante llaves públicas/privadas [criptografía asimétrica] para aplicaciones externas) y lleva un sistema de permisos orientado a usuarios, grupos y roles.

\subparagraph{Módulo de Autenticación Clásica de Usuarios}
El módulo de autenticación clásica de usuarios permite que administradores pueden iniciarse sesión mediante un usuario y contraseña.

\subparagraph{Módulo de Autenticación por Criptografía Asimétrica}
El módulo de autenticación por criptografía asimétrica permite que usuarios finales, a través de GitEdu puede iniciarse sesión mediante para la ejecución controlada y remoto de comandos y código.

\subparagraph{Módulo de Grupos de Usuarios}
El módulo de grupos de usuarios permite definir grupos específicos de usuarios cuyo membresía les cambia el nivel y naturaleza del acceso que tengan a la aplicación.

\subparagraph{Módulo de Permisos de Usuarios y Grupos}
El módulo de permisos de usuarios y grupos define la naturaleza de acceso que tengan usuarios autorizados y los grupos a los cuales pertenecen dentro de la aplicación.

\subparagraph{Módulo de Roles de Usuarios y Grupos}
El módulo de roles de usuarios y grupos define las responsabilidades de varios grupos de usuarios y usuarios específicos y en base a los mismos les asigna permisos adecuados tanto para proteger la aplicación contra uso y accesos indebidos y al mismo tiempo permitir a los usuarios poder realizar sus actividades normalmente sin impedimentos.

\paragraph{Subsistema de Ejecución}
El subsistema de ejecución de código se encarga de recibir y procesar  comandos del sistema externa GitEdu con un fin de aprovechar los hipervisores que tiene a su cargo y proveer funcionalidades de ejecución en línea, aplicación de pruebas unitarias y calificación automática de código sin exponer dichos hipervisores a usuarios finales. Para aumentar la seguridad que logra llevar EduNube, GitEdu communica con lo mismo a través de canales seguros como HTTPS y SSH utilizando métodos de autenticación orientados a una schema de llaves públicas y privadas (criptografía asimétrica) para asegurarse de la identidad de ambos previo al envío de comandos y comunicación de datos y notas.

\subparagraph{Módulo de Recepción de Comandos}
El módulo de recepción de comandos se encarga de recibir comandos externos enviados a través del API externa que provee para que otras aplicaciones pueden indicar al mismo cuando debe bajar, ejecutar o calificar algún código.

\subparagraph{Módulo de Permisos y Restricciones de Comandos}
El módulo de permisos y restricciones de comandos se encarga de limitar o permitir comandos externos recibidos por el sistema EduNube. Se basa en políticas establecidos previamente que definen los horarios de trabajo de ciertos procesos y dependencias de los mismos (por ejemplo no se debe calificar los exámenes de un parallelo hasta que todos terminan). Junto al modulo de control de acceso a codigo, eso permite a docentes y administradores tener control refinado sobre cuando y que está permitido hacer usuarios que están usando EduNube a través de GitEdu.

\subparagraph{Módulo de Control de Hipervisores}
El módulo de control de hipervisores se encarga de cumplir con la funcionalidades críticos del sistema EduNube de compilar, ejecutar y calificar código. Esto lo hace a través de tres respectivos submódulos para cada fase. De manera más general, este módulo también controla hipervisores conectados al sistema EduNube con un fin de poder realizar las operaciones listados previamente.
\begin{description}
	\item[Submódulo de Compilación de Código]
	El submódulo de compilación de código se encarga de compilar codigo que se le entrega en base a procedimientos que define el profesor anteriormente y reporta resultados del mismo, sea exitosa o sea errores de compilación. Para la implementación del mismo submódulo, se aprovecha de las máquinas virtuales que provee el hipervisor con un fin de realizar la compilación en un ambiente controlado y aislado.
    \item[Submódulo de Ejecución de Código]
	El submódulo de ejecucion de codigo recibe de entradas codigo compilado y comandos que se desee ejecutar en el ambiente de ejecución y pruebas con un fin de proveer, a través de las máquinas virtuales controladas y aisladas, un ambiente interactivo donde estudiantes, tanto estudiantes como docentes pueden trabajar de una forma interactiva con código desarrollado dentro de la plataforma.
    \item[Submódulo de Calificación Automatizado de Código]
	El submódulo de calificación automatizado de código se basa en pruebas unitarias escritas previamente por un docente y que van contra la funcionalidad que se pide en el código de los estudiantes con un fin de calificar el código de una forma autónoma. Una vez que termina de calificar un código, reporta los resultados a gestores de los mismos.
\end{description}

\subparagraph{Módulo de Persistencia de Código}
El módulo de persistencia de código recupere codigo versionado en servidores de control de versiones externas que usuarios han escrito y editado con un fin de permitir la compilación, ejecución y calificación del mismo dentro del sistema EduNube.

\subparagraph{Módulo de Control de Acceso a Código}
El módulo de control de acceso a codigo se encarga de restringir y permitir en base a reglas las operaciones de compilación, ejecución y calificación automatizada de códigos que usuarios editan. De la misma forma permite definir flujos de trabajo y dependencias para el procesamiento de proyectos en adición a establecer horarios y restricciones de procesamiento para los mismos.

\subparagraph{Módulo de Comunicación de Datos y Notas}
El módulo de comunicación de datos y notas permite la sincronización de los mismos cuando sea necesario entre EduNube y GitEdu.

\paragraph{Subsistema de Administración}
El subsistema de administración ayuda a administradores de EduNube recolectar datos acerca del uso del mismo y analizar lo datos en adición a ser capaces de configurar y monitorear el rendimiento del sistema.

\subparagraph{Módulo de Configuración}
El módulo de configuración permite la configuración en tiempo real del sistema EduNube, por ejemplo conexiones con sistemas externas, gestión de llaves públicas y privadas, y gestión de usuarios.

\subparagraph{Modulo de Monitoreo}
El módulo de monitoreo permite ver en tiempo real el consumo del sistema y sus hipervisores respectivos.

\subparagraph{Módulo de Recolección de Datos}
El módulo de recolección de datos lleva un registro continuo del uso del sistema para su revisión y análisis por parte de operadores humanos después tanto para temas de auditoría como toma de decisiones estratégicas.

\subparagraph{Módulo de Análisis de Datos}
El módulo de análisis de datos proporciona las herramientas necesarias para que un operador humano puede interpretar los mismos y sacar sus conclusiones.

\subsubsection{Ventajas del Diseño de Sistemas, Subsistemas y Módulos}
La ventaja de un diseño orientado a módulos como se lo presenta aquí es la división de funcionalidad para ayudar en la extensibilidad y mantenibilidad de los sistemas al largo plazo. Por ejemplo este diseño permite agregar funcionalidad adicional a futuro como autenticación por otros medios (por ejemplo LDAP) o la integración de mecanismos anti-plagio. Esta propiedad puede dar una vida útil mayor a los sistemas desarrolladas porque muchas veces no se conoce todo lo que puede requerir un software a futuro para el diseño e implementación inicial. A continuación se presenta la arquitectura de estos dos sistemas, tanto GitEdu como su ambiente de ejecución de código virtualizado, EduNube.

\subsection{Arquitectura}

% figure

Como se puede observar en la figura (diag-arq-gitedu), a un nivel global, el sistema GitEdu consiste en 5 capas que son:
\begin{itemize}
	\item \textbf{Seguridad} auténtica usuarios y controla el acceso que tiene cada uno de ellos. Esto se realiza con los siguientes componentes:
    \begin{itemize}
    	\item \textbf{Autenticación por LTI} usa interfaces LTI de aplicaciones externas como LMS institucionales para autenticar usuarios.
		\item \textbf{Autenticación Clásica} permite el login (autenticación) de usuarios mediante la forma clásica de un usuario y contraseña.
		\item \textbf{Permisos} permite otorgar o revocar partes específicas del sistema a usuarios específicos. Ejemplos de permisos podrían ser acceso a la administración de una materia, por ejemplo el control que se da a un docente o grupo de docentes sobre una materia.
		\item \textbf{Grupos} permite aplicar el sistema de permisos a grupos de usuarios creados bajo la discreción de usuarios suficiente privilegiados dentro del sistema. Ejemplos de grupos podrían ser miembros de una cierta materia o tipo de materia, por ejemplo ''Base de Datos''.
		\item \textbf{Roles} permite definir usuarios por su propósito del sistema con un fin de aplicarles un conjunto global de permisos asociados por el tipo de usuario es. Ejemplos de roles podrían ser Estudiante, Docente y Administrador.
    \end{itemize}
    \item \textbf{Administración} permite la administración, mantenimiento y recolección de datos por usuarios dentro del sistema. Para cumplir con estas responsabilidades se divide en:
    \begin{itemize}
    	\item \textbf{Subsistema de Gestión de Configuraciones} que permite a administradores cambiar de forma rapida y facil las configuraciones internas del sistema.
        \item \textbf{Subsistema de Recolección de Datos de Uso} que collecta automáticamente datos del uso del sistema.
        \item \textbf{Subsistema de Análisis de Datos de Uso} que permite a administradores visualizar y analizar los datos que ha recolectado el Subsistema de Recolección de Datos de Uso.
        \item \textbf{Subsistema de Administración y Mantenimiento} que permite a administradores ver en tiempo real y también registros históricos de consumo de recursos. Este subsistema también proporciona a administradores las herramientas que necesita para gestionar y controlar problemas como usuarios maliciosos, sistemas caídos entre otros problemas detectados.
    \end{itemize}
    \item \textbf{Presentación} que se encarga de presentar interfaces web a varias tipos de usuario, incluyendo:
    \begin{itemize}
    	\item \textbf{Estudiantes}
		\item \textbf{Docentes}
		\item \textbf{Administradores}
    \end{itemize}
    \item \textbf{Lógica de Negocios} que se encarga de llevar a cabo los procesos de negocio por el cual la aplicación tiene propósito. Esto se subdivida en cuatro subsistemas:
    \begin{itemize}
    	\item \textbf{Subsistema de Docentes y Educación} que se lleva a cabo procesos de negocio que tienen que ver con enseñanza y evaluación. Dentro de esta subsistema también se lleva las tareas de administración asociadas con un docente y sus aulas de alumnos.
        \item \textbf{Subsistema de Gestión de Código} lleva el control del código y los procesos asociados en cuanto.
        \item \textbf{Subsistema de Gestión de Notas} maneja los procesos de negocio asociados con la sincronización y auditoría de notas generadas dentro del sistema.
        \item \textbf{Subsistema de Ejecución de Código} que lleva a cabo procesos de negocio que tienen que ver con ejecución arbitraria de código y por lo tanto requiere un cierto nivel de seguridad y aislamiento elevado a través de entornos virtualizados. Dentro de los procesos que se incluyen aquí, hay compilación, ejecución y calificación de código automatizado mediante pruebas unitarias.
    \end{itemize}
    \item \textbf{Persistencia} que se encarga de asegurar que se persisten datos importantes para uso futuro y también la recuperación de los mismos. Esto se subdivide en tres clases de persistencia:
    \begin{itemize}
    	\item Persistencia a través de un \textbf{ORM Relacional} que guarda y recupera datos estructurados de la aplicación en bases de datos relacionales y orienta los mismos datos a objetos dentro de la aplicación para permitir su facil manipulacion.
        \item Persistencia a través de un \textbf{ORM NoSQL} que guarda y recupera datos no estructurados de la aplicación en bases de datos no relacionales y orienta los mismos datos a objetos dentro de la aplicación para permitir su facil manipulacion.
        \item Persistencia de código a través de un \textbf{Gestor de Persistencia de Código Externo} que guarda y recupera codigo de usuarios para su manipulación dentro de la aplicación y persistencia fuera del mismo.
    \end{itemize}
\end{itemize}

% figure

Como se puede observar en la figura (diag-arq-edunube), a un nivel global, el sistema EduNube consiste en 5 capas que son:
\begin{itemize}
	\item \textbf{Seguridad} auténtica administradores y sistemas externos y controla el acceso que tiene cada uno de ellos. Esto se realiza con los siguientes componentes:
    \begin{itemize}
    	\item \textbf{Autenticación por Criptografía Asimétrica} usa pares de llaves públicas y privadas suficientemente grandes para ser difíciles de falsificar matemáticamente para autenticar sistemas externas.
    	\item \textbf{Autenticación Clásica} permite el login (autenticación) de administradores mediante la forma clásica de un usuario y contraseña.
    	\item \textbf{Permisos} permite otorgar o revocar partes específicas del sistema a usuarios específicos. Ejemplos de permisos podrían ser quien puede crear máquinas virtuales o ver el consumo de recursos del sistema.
    	\item \textbf{Grupos} permite aplicar el sistema de permisos a grupos de usuarios creados bajo la discreción de administradores suficiente privilegiados dentro del sistema. Ejemplos de grupos podrían ser ciertos tipos de usuario que representa algun tipo de estudiante en algún tipo de materia, por ejemplo para la materia de ''Base de Datos''.
    	\item \textbf{Roles} permite definir usuarios por su propósito del sistema con un fin de aplicarles un conjunto global de permisos asociados por el tipo de usuario es. Ejemplos de roles podrían ser Sistema Externa y Administrador.
    \end{itemize}
	\item \textbf{Administración} permite la administración, mantenimiento y recolección de datos por usuarios dentro del sistema. Para cumplir con estas responsabilidades se divide en:
    \begin{itemize}
    	\item \textbf{Subsistema de Gestión de Configuraciones} que permite a administradores cambiar de forma rapida y facil las configuraciones internas del sistema.
    	\item \textbf{Subsistema de Recolección de Datos de Uso} que collecta automáticamente datos del uso del sistema.
    	\item \textbf{Subsistema de Análisis de Datos de Uso} que permite a administradores visualizar y analizar los datos que ha recolectado el Subsistema de Recolección de Datos de Uso.
    	\item \textbf{Subsistema de Monitoreo} que permite a administradores ver en tiempo real y también registros históricos de consumo de recursos. Este subsistema también proporciona a administradores las herramientas que necesita para gestionar y controlar problemas como usuarios maliciosos, sistemas caídos entre otros problemas detectados.
    \end{itemize}
	\item \textbf{Presentación} que se encarga de presentar interfaces web a varias tipos de usuario, incluyendo:
    \begin{itemize}
    	\item \textbf{Compilación/Ejecución de Código y Pruebas Unitarias} proporciona las interfaces gráficas utilizadas durante la compilacion y ejecucion de codigo en adición a la aplicación de pruebas unitarias contra un codigo para ayudar en la calificación del mismo.
        \item \textbf{Notas} permite a docentes y sistemas externos ver las notas asignados a un código.
        \item \textbf{Administradores} proporciona las interfaces gráficas necesarias para ayudar un administrador cumplir con sus responsabilidades de monitoreo, control y mantenimiento del sistema en ambientes de producción.
    \end{itemize}
	\item \textbf{Lógica de Negocios} que se encarga de llevar a cabo los procesos de negocio por el cual la aplicación tiene propósito. Esto se subdivida en cuatro subsistemas:
    \begin{itemize}
    	\item \textbf{Subsistema de Recepción de Comandos} que se lleva a cabo procesos de negocio que tienen que ver con recibir comandos de sistemas externas y procesar las mismas.
        \item \textbf{Subsistema de Control de Hipervisores} lleva el control de los hipervisores conectados y les guía en el proceso de virtualizar y ejecutar código dentro de máquinas virtuales de los mismos hipervisores.
        \item \textbf{Subsistema de Calificaciones} maneja los procesos de negocio asociados con la calificación, sincronización y auditoría de notas generadas dentro del sistema.
        \item \textbf{Subsistema de Ejecución de Código y Pruebas Unitarias} que lleva a cabo procesos de negocio que tienen que ver con ejecución arbitraria de código en entornos virtualizados. Dentro de los procesos que se incluyen aquí, hay compilación, ejecución y calificación de código automatizado mediante pruebas unitarias.
    \end{itemize}
	\item \textbf{Persistencia} que se encarga de asegurar que se persisten datos importantes para uso futuro y también la recuperación de los mismos. Esto se subdivide en tres clases de persistencia:
    \begin{itemize}
    	\item Persistencia a través de un \textbf{ORM Relacional} que guarda y recupera datos estructurados de la aplicación en bases de datos relacionales y orienta los mismos datos a objetos dentro de la aplicación para permitir su facil manipulacion.
		\item Persistencia a través de un \textbf{ORM NoSQL} que guarda y recupera datos no estructurados de la aplicación en bases de datos no relacionales y orienta los mismos datos a objetos dentro de la aplicación para permitir su facil manipulacion.
		\item Persistencia de código a través de un \textbf{Gestor de Persistencia de Código Externo} que guarda y recupera codigo de usuarios para su manipulación dentro de la aplicación y persistencia fuera del mismo.
    \end{itemize}
\end{itemize}

\subsubsection{Ventajas del Diseño Arquitectónico}
Se ha considerado para ambos sistemas una arquitectura de n-capas debido a su capacidad de escalabilidad sobre varios niveles físicos. Además la separación lógica entre capas permite mayor mantenibilidad al largo plazo donde solo se necesitan reemplazar los componentes que requieren cambios en lugar de todo el sistema. El nivel de aislamiento entre estas mismas capas brinde mejor seguridad contra usuarios externos sean maliciosos o que quieren dar mal uso (un uso inadecuado) al sistema. A continuación se presenta la arquitectura de despliegue propuesta dentro de este trabajo de titulación.

\subsection{Despliegue}

% figure

Como se puede observar en la figura (diag-despl-gitedu-edunube), el despliegue del sistema GitEdu junto al sistema EduNube, consiste en dos redes, una externa que contiene los usuarios finales (Estudiantes, Docentes, Administradores) en adición a sistemas externas como los dos LMS institucionales (Moodle y OpenEdX) y el servidor de Git (GitLab Community Edition), mientras que la red interna contiene las varias componentes que forman el conjunto de sistemas:
\begin{itemize}
	\item \textbf{Cortafuegos} divide la red interna de la red externa y permite únicamente que pasa tráfico HTTP y HTTPS de manera bidireccional (con posibilidad de llamadas LTI sobre estos protocolos) y la salida de SSH al servidor de Git (para ofrecer mayor seguridad que transferencias sobre HTTP(S)).
    \item \textbf{Proxy de Terminación SSL} accepta HTTP y HTTPS, pero para asegurar la seguridad de usuarios finales y aplicaciones externas redirecciona toda uso del protocolo HTTP a HTTPS. Además descifra conexiones HTTPS entrantes para manejarlos por HTTP dentro de la red interna (reduciendo la carga en otros servidores sin abandonar seguridad en la red externa) y cifra conexiones salientes. Se utiliza NGinX por su bajo consumo de recursos en comparación con sus competidores como Apache.
    \item \textbf{Proxy Inversa / Balanceador de Carga} accepta todo el tráfico HTTPS (ahora por el protocolo HTTP dentro de la red interna) entrante y elige en base a la ruta cual servidor se lo puede responder. A futuro este componente permite que se puede escalar el sistema en lo que permite dividir tráfico entre varios servidores del mismo tipo (por ejemplo dos servidores de aplicación: GitEdu1 y GitEdu2). Se considera que NGinX es la solución más óptima para realizar este rol.
    \item \textbf{Servidor Web de Archivos Estáticos} se responsabiliza por peticiones de archivos estáticos que no conviene enviar a los servidores de aplicación (quienes los procesaría con mayor lentitud). Estos mismos tienen una conexión NFS o SSHFS con los mismos servidores de aplicación para que estos sean capaces de actualizar remotamente los archivos estáticos en caso de que los mismos cambien (por ejemplo cuando un usuario actualiza su foto de perfil). Estudios de rendimiento demuestran que NGinX cumple este rol de ser servidor web de archivos estáticos mejor que Apache.
    \item \textbf{Servidor de Aplicación ''GitEdu''} se aloja el sistema GitEdu, desarrollado bajo el marco Django y el lenguaje de programación Python (version 3.x). Comunica con el servidor externo de Git sobre SSH y dispone tanto de un base de datos relacional como uno no relacional para persistir los distintos tipos de datos que tiene que manejar. Se conecta directamente al Servidor de Aplicación “EduNube” mediante SHTTP (HTTP sobre un túnel SSH) utilizando un par de llaves (pública y privada) como schéma de criptografía asimétrica. Se considera uWSGI como mejor servidor de aplicación para el sistema debido a que, a diferencia de su competencia (Gunicorn), esta suporta Websockets.
    \item \textbf{Motores de Base de Datos Relacional} se alojan datos estructurados para ambos sistemas, tanto GitEdu como NubeEdu.
    \item \textbf{Motores de Base de Datos NoSQL} se alojan datos no estructurados para ambos sistemas, tanto GitEdu como NubeEdu.
    \item \textbf{Proxy Inversa de Control de Acceso} restringe acceso al servidor de aplicaciones de EduNube cuando se accede al mismo desde la red externa (a través del proxy inversa).
    \item \textbf{Servidor de Aplicación ''EduNube''} se aloja el sistema EduNube, desarrollado bajo el marco Django y el lenguaje de programación Python (version 3.x). Comunica con el servidor externo de Git sobre SSH y dispone tanto de un base de datos relacional como uno no relacional para persistir los distintos tipos de datos que tiene que manejar. Utiliza un shell local (dentro del mismo servidor) para manejo de un hipervisor local o en el caso de un hipervisor remoto, maneja el mismo sobre SSH. Las máquinas virtuales que se levanta el hipervisor, también se los controla con SSH. Se considera uWSGI como mejor servidor de aplicación para el sistema debido a que, a diferencia de su competencia (Gunicorn), esta suporta Websockets.
    \item \textbf{Hipervisores} gestionan máquinas virtuales por parte del sistema EduNube quien le envía comandos por SSH o un shell local. A su vez, controla sus máquinas virtuales mediantes SSH. Como tecnologías de hipervisor se considera Xen, KVM y Docker como buenas opciones para diferentes necesidades. Cada sistema de hipervisores soporta el uso de plantillas para la creación masiva de máquinas virtuales iguales para cada estudiante de una materia.
    \item \textbf{Máquinas Virtuales} ejecutan código de usuarios y reportan los resultados.
\end{itemize}

\subsubsection{Ventajas de la Arquitectura de Despliegue}
De la misma moda como el diseño de sistemas, subsistemas y módulos en adición al diseño arquitectónico, el diseño de arquitectura de despliegue está orientado a extensibilidad a través de modularidad, mantenibilidad por la naturaleza de sus capas y módulos, además de escalabilidad y seguridad mantenido a través de mecanismos de aislamiento de componentes que sean libres de escalar independientemente entre ellos, la arquitectura de despliegue se orienta a los mismos principios.

La separación de deberes dentro del sistema permite que cada servicio dentro de la arquitectura de despliegue se puede dedicar a hacer bien su parte de la manera más eficiente posible sin responsabilidades que no puede sostener de forma inadecuada, algo importante para empezar a poder garantizar rendimiento óptimo del sistema. El uso de un cortafuegos y varias proxies inversas adelantes de cada sistema también ayuda aislar las mismas del mundo externo a un nivel de protocolos reduce los superficies de ataque contra las mismas para ofrecer seguridad sin sacrificar rendimiento de una manera similar al punto anterior de separación de trabajo en roles y asignación al actor más adecuado para el mismo.

Este mismo uso de trabajadores, especializado cada uno en sus responsabilidades y posible carga de trabajo también sigue ofreciendo escalabilidad a futuro. Esto se realiza con comunicación entre servicios que está sobre protocolos de red para que las mismas o pueden estar en un solo recurso (sea servidor fisica o virtual) o en varios apartados lo cual permite a futuro la asignación de recursos físicos o virtuales como se lo ve la necesidad. Esta capacidad para escalar se puede ofrecer gracias al hecho de que los servicios semi independientes unos de otros, permitiendo n-capas sobre m-niveles físicos/lógicos.

Donde haya dependencias de, por ejemplo, sistema de ficheros compartidos para dos capas distintas se puede implementar con protocolos que permiten acceso remoto para las operaciones necesarios, en este caso que los servidores de aplicación sean capaces de actualizar los archivos estáticos en el servidor para el mismo (se lo está considerando esta parte con un NFS o SSHFS). Obviamente en el caso de que estén en el mismo servidor, no se da la necesidad de establecer aquello conexión.

En casos donde se considera una necesidad de mayor seguridad en proteger canales de comunicación, se ha considerado que la mejor forma de brindar lo mismo es con tunnels SSH que encapsulan trafico que por si sola ocupa un protocolo adecuado según la necesidad. Por ejemplo para consumir el API de EduNube, GitEdu utiliza HTTP sobre SSH (también conocido como SHTTP). De esta forma se puede aprovechar lo que ya existe y que está comprobada como segura y eficiente (SSH como protocolo, especialmente en su segunda versión, obliga fuertes mecanismos de autenticación y seguridad mientras que al mismo tiempo ofrece menor sacrificio de rendimiento y alta funcionalidad).

A continuación se presenta el alcance del desarrollo, pruebas y despliegue para este trabajo de titulación.

\subsection{Alcance}

Debido al hecho de que se cuenta con tiempo y recursos limitados para realizar y acabar el trabajo de titulación, se considera un alcance limitado en ciertos aspectos con un fin de acabar con los componentes críticos del sistema dentro del tiempo dado.

% figure

Para el sistema GitEdu se considera los siguientes subsistemas y módulos críticos:
\begin{itemize}
	\item \textbf{Todo el subsistema de Gestión de Usuarios.} Todos los módulos de este sistema son críticos porque se encargan de cómo los usuarios se autentican contra el mismo y el control de acceso que tienen cada uno. Por lo tanto el subsistema en sí es una parte importante para garantizar seguridad en el sistema final.
    \item \textbf{Parte del subsistema de Editar Código.} El subsistema de editar codigo se considera crítico los módulos tanto de editar como de persistir codigo. El módulo de editar codigo es crítico porque forma la funcionalidad principal de que se trata el sistema mientras que el modulo de persistir codigo viene a ser critico para sincronizar codigo entre los dos sistemas. No se toma en consideración como críticos los módulos de interacción social, compartir codigo ni control de acceso a codigo porque las mismas se tratan de llevar el sistema más allá de su propósito original y son más adecuados para un futuro trabajo.
    \item \textbf{Parte del subsistema de Docentes.} Dentro del subsistema de docentes, se considera crítico un módulo de crear talleres, pruebas y exámenes en adición a un módulo de notas y su sincronización. Entre estos dos se forma el núcleo de funcionalidad que requieren docentes dentro de la versión inicial. Dentro de trabajos futuros, se podría extender esta funcionalidad crítica con temas más administrativas como los módulos de gestión de materias y gestión de estudiantes que por el momento no se los considera como funcionalidades críticas para una primera versión estable, el producto de este trabajo de titulación.
    \item \textbf{Parte del subsistema de Administración.} Dentro del subsistema de administración, solo se considera que los componentes críticos se tratan de colección de datos y el análisis de los mismos debido a que esos dos módulos respectivamente ayudan en la toma de decisiones estratégicas que tienen que ver con la escuela de Ciencias de Computación y Electrónica y el uso del sistema. Los módulos de monitoreo y configuración, aunque podrían ser muy útiles en un ambiente de producción no caen dentro del alcance inicial de este tesis y por lo tanto no se los considera críticos bajo los parámetros del proyecto. De todas formas su implementación podría ser una funcionalidad adicional interesante para un futuro proyecto.
    \item \textbf{Todo el subsistema de Ejecución.} Todos los módulos de este sistema son críticos porque se encargan de la comunicación con el sistema EduNube. El módulo de comandos prepara comandos para ejecutar a través del módulo de comunicación de datos y notas se los envía en adición a conseguir resultados y sincronizar notas y calificaciones generadas por EduNube.
\end{itemize}

Para el sistema EduNube, se considera los siguientes subsistemas y módulos como críticos:
\begin{itemize}
	\item \textbf{Todo el subsistema de Gestión de Usuarios.} Todos los módulos de este sistema son críticos porque se encargan de cómo los usuarios se autentican contra el mismo y el control de acceso que tienen cada uno. Por lo tanto el subsistema en sí es una parte importante para garantizar seguridad en el sistema final.
	\item \textbf{Parte del subsistema de Ejecución.} Dentro del subsistema de ejecución se consideran críticos los módulos de comandos (para procesar comandos entrantes), de control de hipervisores (para gestionar recursos virtuales), y de persistencia de código (para recuperar código para su compilación, compilación y ejecución).  Además como componentes del módulo de control de hipervisores, se considera crítico los submódulos de compilación de código (para aquellos lenguajes que lo requieren), de ejecución (para ejecutar código), y de calificación automatizado de código (para calificar codigo de estudiantes). Se considera que los módulos de seguridad adicionales dentro de este subsistema no son necesarios por el momento debido al que único sistema que tendrá acceso a esta será la de GitEdu. A futuro, donde más sistemas y servicios empiezan a necesitar consumir el API de EduNube, llegará a ser necesario que tenga aquellos módulos de seguridad para proveer protección adicional.
	\item \textbf{Parte del subsistema de Administración.} Dentro del subsistema de administración, solo se considera que los componentes críticos se tratan de colección de datos y el análisis de los mismos debido a que esos dos módulos respectivamente ayudan en la toma de decisiones estratégicas que tienen que ver con la escuela de Ciencias de Computación y Electrónica y el uso del sistema. Los módulos de monitoreo y configuración, aunque podrían ser muy útiles en un ambiente de producción no caen dentro del alcance inicial de este tesis y por lo tanto no se los considera críticos bajo los parámetros del proyecto. De todas formas su implementación podría ser una funcionalidad adicional interesante para un futuro proyecto.
\end{itemize}

% figure

En cuanto el despliegue, los sistemas externas que se toma en consideración para el alcance son Moodle como LMS, y GitLab Community Edition como servidor de control de versiones. En relación a los ambientes virtuales, solo se considera un hipervisor y un sistema operativo para las máquinas virtuales dentro de la versión final (durante el desarrollo se presenta los candidatos, comparación y selección de los mismos). También se incluye dentro del alcance el desarrollo y despliegue de los dos sistemas GitEdu y EduNube con sus respectivas bases de datos utilizando uWSGI como servidor de aplicación. No se incluye dentro del alcance OpenEdX o ningún otro LMS externa fuera de la mencionada previamente ni las configuraciones del cortafuego y servidores web, estos quedan como responsabilidades institucionales en cuanto como deseen implementar estos partes del despliegue. Lo que se presenta aquí es una sugerencia del autor basado en su experiencia práctica.
