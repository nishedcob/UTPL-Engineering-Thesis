% Chapter 7

\chapter{Conclusiones}
\label{capitulo7}
% TODO: get input on conclusions (are they okay for the thesis)
% Minimo 1 Conclusion frente cada Objetivo Especifico
% Minimo 1 Conclusion frente cada Linea Problematica
% Lo que no va en resultados, puede ir a conclusiones
% Conclusiones tienen que ser declaraciones positivas o negativas, no van preguntas
% Cuantas Conclusiones poner? 3-5 es acceptable -> 7 conclusiones suele ser demaciado
% Tipicamente un recomendacion por cada cada conclusion

LTI puede ser una buena solución que se puede tomar en cuenta para facilitar el flujo de autenticación entre sistemas para el aprendizaje y enseñanza.

Muchos usuarios solo se fijan en que tan llamativo es un interfaz de usuario y no en la funcionalidad que esta por detras. Por lo tanto era importante encontrar la herramientas adecuadas para que el editor de código en la plataforma sea tanto llamativa como funcional, porque la primera impresión es lo que más cuenta.

Para datos que no tienen estructuras muy fijas o que tienen campos que pueden extenderse de larga, MongoDB ofrece una buena alternativa a los bases de datos relacionales y simplifica el trabajo que realizar.

Kubernetes y Docker son herramientas muy potentes para generar entornos confiables, entregar sistemas de producción valiosas y garantizar escalabilidad y portabilidad de código realizado.

Cualquier servidor de Git es altamente pesada, especialmente GitLab que en su estado pasiva consume muchos recursos, y tambien GitWeb que cuelga cada vez que se realiza subidas sobre HTTP.

% TODO: Revisar Conclusion General
La combinacion de Python 3 con librerias selecionadas para realizar los respectivos backends, es decir Django, PyModm, Bcrypt, PyJWT, Requests y ipython forma una combinacion potente para el desarollo rapido de servicios de produccion siempre y cuando se maneja de forma adecuada el diseño de los mismos y el manejo de distintos tipos de datos.
