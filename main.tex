%%%%%%%%%%%%%%%%%%%%%%%%%%%%%%%%%%%%%%%%%
% Masters/Doctoral Thesis 
% LaTeX Template
% Version 2.4 (22/11/16)
%
% This template has been downloaded from:
% http://www.LaTeXTemplates.com
%
% Version 2.x major modifications by:
% Vel (vel@latextemplates.com)
%
% This template is based on a template by:
% Steve Gunn (http://users.ecs.soton.ac.uk/srg/softwaretools/document/templates/)
% Sunil Patel (http://www.sunilpatel.co.uk/thesis-template/)
%
% Template license:
% CC BY-NC-SA 3.0 (http://creativecommons.org/licenses/by-nc-sa/3.0/)
%
%%%%%%%%%%%%%%%%%%%%%%%%%%%%%%%%%%%%%%%%%

%----------------------------------------------------------------------------------------
%	PACKAGES AND OTHER DOCUMENT CONFIGURATIONS
%----------------------------------------------------------------------------------------

\documentclass[
11pt, % The default document font size, options: 10pt, 11pt, 12pt
%oneside, % Two side (alternating margins) for binding by default, uncomment to switch to one side
%english, % ngerman for German
spanish, % ngerman for German
singlespacing, % Single line spacing, alternatives: onehalfspacing or doublespacing
%draft, % Uncomment to enable draft mode (no pictures, no links, overfull hboxes indicated)
%nolistspacing, % If the document is onehalfspacing or doublespacing, uncomment this to set spacing in lists to single
%liststotoc, % Uncomment to add the list of figures/tables/etc to the table of contents
%toctotoc, % Uncomment to add the main table of contents to the table of contents
%parskip, % Uncomment to add space between paragraphs
%nohyperref, % Uncomment to not load the hyperref package
headsepline, % Uncomment to get a line under the header
%chapterinoneline, % Uncomment to place the chapter title next to the number on one line
%consistentlayout, % Uncomment to change the layout of the declaration, abstract and acknowledgements pages to match the default layout
]{MastersDoctoralThesis} % The class file specifying the document structure

\usepackage[utf8]{inputenc} % Required for inputting international characters
\usepackage[T1]{fontenc} % Output font encoding for international characters

\usepackage{palatino} % Use the Palatino font by default

\usepackage{float}
\floatstyle{plaintop}
\restylefloat{table}
%\restylefloat{figure}
%\restylefloat{longtable}

\usepackage{lscape}

\usepackage{scrextend}

\usepackage{listings}

\usepackage{multicol}

\usepackage{makeidx}
\makeindex

\usepackage[backend=bibtex,style=authoryear,natbib=true]{biblatex} % Use the bibtex backend with the authoryear citation style (which resembles APA)

%\addbibresource{example.bib} % The filename of the bibliography
\addbibresource{main.bib} % The filename of the bibliography

\usepackage[autostyle=true]{csquotes} % Required to generate language-dependent quotes in the bibliography

%----------------------------------------------------------------------------------------
%	MARGIN SETTINGS
%----------------------------------------------------------------------------------------

\geometry{
	paper=a4paper, % Change to letterpaper for US letter
	inner=2.5cm, % Inner margin
	outer=3.8cm, % Outer margin
	bindingoffset=.5cm, % Binding offset
	top=1.5cm, % Top margin
	bottom=1.5cm, % Bottom margin
	%showframe, % Uncomment to show how the type block is set on the page
}

%----------------------------------------------------------------------------------------
%	THESIS INFORMATION
%----------------------------------------------------------------------------------------

\thesistitle{GitEDU y EduNube:\\ Un Doble Sistema de Integración de LTI, Git y Virtualización} % Your thesis title, this is used in the title and abstract, print it elsewhere with \ttitle
\supervisor{Dr. Jorge \textsc{López}} % Your supervisor's name, this is used in the title page, print it elsewhere with \supname
\examiner{} % Your examiner's name, this is not currently used anywhere in the template, print it elsewhere with \examname
\degree{Ingenería en Sistemas Informáticos y Computación} % Your degree name, this is used in the title page and abstract, print it elsewhere with \degreename
\author{Nicholas \textsc{Spalding Earley-Dolenc}} % Your name, this is used in the title page and abstract, print it elsewhere with \authorname
\addresses{} % Your address, this is not currently used anywhere in the template, print it elsewhere with \addressname

\subject{Ciencias de la Computación} % Your subject area, this is not currently used anywhere in the template, print it elsewhere with \subjectname
\keywords{} % Keywords for your thesis, this is not currently used anywhere in the template, print it elsewhere with \keywordnames
\university{\href{http://utpl.edu.ec}{Universidad Técnica Particular de Loja}} % Your university's name and URL, this is used in the title page and abstract, print it elsewhere with \univname
\department{\href{http://utpl.edu.ec}{Escuela de Ciencias de la Computación}} % Your department's name and URL, this is used in the title page and abstract, print it elsewhere with \deptname
\group{\href{http://utpl.edu.ec}{Laboratorio de Tecnologías Web}} % Your research group's name and URL, this is used in the title page, print it elsewhere with \groupname
\faculty{\href{http://utpl.edu.ec}{Departamento de Ciencias de la Computación y Electrónica}} % Your faculty's name and URL, this is used in the title page and abstract, print it elsewhere with \facname

\AtBeginDocument{
\hypersetup{pdftitle=\ttitle} % Set the PDF's title to your title
\hypersetup{pdfauthor=\authorname} % Set the PDF's author to your name
\hypersetup{pdfkeywords=\keywordnames} % Set the PDF's keywords to your keywords
%%%% BEGIN PERSONAL ADDITIONS
\hypersetup{urlcolor=black}
%%%% ENDED ADDED
}

%\babelhyphenation[spanish]{Con-te-ner-i-za-ción}

\begin{document}

\frontmatter % Use roman page numbering style (i, ii, iii, iv...) for the pre-content pages

\pagestyle{plain} % Default to the plain heading style until the thesis style is called for the body content

%----------------------------------------------------------------------------------------
%	TITLE PAGE
%----------------------------------------------------------------------------------------

\begin{titlepage}
\begin{center}

\vspace*{.06\textheight}
{\scshape\LARGE \univname\par}\vspace{1.5cm} % University name
\textsc{\Large Tesis de Ingenería}\\[0.5cm] % Thesis type

\HRule \\[0.4cm] % Horizontal line
{\huge \bfseries \ttitle\par}\vspace{0.4cm} % Thesis title
\HRule \\[1.5cm] % Horizontal line
 
\begin{minipage}[t]{0.4\textwidth}
\begin{flushleft} \large
\emph{Autor:}\\
\href{https://github.com/nishedcob}{\authorname} % Author name - remove the \href bracket to remove the link
\end{flushleft}
\end{minipage}
\begin{minipage}[t]{0.4\textwidth}
\begin{flushright} \large
\emph{Supervisor:} \\
\href{https://github.com/nishedcob}{\supname} % Supervisor name - remove the \href bracket to remove the link  
\end{flushright}
\end{minipage}\\[3cm]
 
\vfill

\large \textit{Tesis entregado para cumplir con los requisitos\\ del titulo de \degreename}\\[0.3cm] % University requirement text
\textit{en el}\\[0.4cm]
\groupname\\\deptname\\[2cm] % Research group name and department name
 
\vfill

{\large \today}\\[4cm] % Date
%\includegraphics{Logo} % University/department logo - uncomment to place it
 
\vfill
\end{center}
\end{titlepage}

%----------------------------------------------------------------------------------------
%	DECLARATION PAGE
%----------------------------------------------------------------------------------------

\begin{declaration}
%\addchaptertocentry{\authorshipname} % Add the declaration to the table of contents
% TODO: Revisar modelo de la universidad
\addchaptertocentry{Declaracion de Autoría} % Add the declaration to the table of contents
\noindent Yo, \authorname, declaro que esta tesis titulada, \enquote{\ttitle} y que el trabajo presentado aquí es p. Yo afirmo que:

\begin{itemize} 
\item Este trabajo fue realizado principalmente mientras que fui candidato de un titulo de investigacion en la Universidad.
\item Partes de esta tesis que se han sido utilizados por algun otro titulo o cualificacion, se lo ha señalado de una forma clara e transparente. 
\item Partes de esta tesis donde he consultado el trabajo publicado de terceros, se lo ha citado de la forma correcta.
\item Donde he citado el trabajo de tercersos, la fuente siempre esta dada. Con la excepcion de aquellos citas, esta thesis es totalmente mi propio trabajo.
\item He documentado mis principales fuentes de ayuda.
\item En lugares donde esta tesis esta basada en trabajo realizado por terceros, he documentado claramente cuales partes realizaron ellos y cuales partes yo realicé.\\
\end{itemize}
 
\noindent Firmado:\\
\rule[0.5em]{25em}{0.5pt} % This prints a line for the signature
 
\noindent Fecha:\\
\rule[0.5em]{25em}{0.5pt} % This prints a line to write the date
\end{declaration}

\cleardoublepage

%----------------------------------------------------------------------------------------
%	QUOTATION PAGE
%----------------------------------------------------------------------------------------

\vspace*{0.2\textheight}

%\noindent\enquote{\itshape Thanks to my solid academic training, today I can write hundreds of words on virtually any topic without possessing a shred of information, which is how I got a good job in journalism.}\bigbreak

%\hfill Dave Barry

\noindent\enquote{\itshape Elegimos\ldots{} hacer las\ldots{} cosas, no porque sea fácil, sino porque es difícil\ldots{} Porque esta meta, servirá para organizar y probar lo mejor de nuestras energías y habilidades.}

\hfill John F. Kennedy

%----------------------------------------------------------------------------------------
%	ABSTRACT PAGE
%----------------------------------------------------------------------------------------

% TODO: Rewrite at the end
\begin{abstract}
\addchaptertocentry{\abstractname} % Add the abstract to the table of contents
Para mejorar el entorno educativo para los estudiantes de carreras que involucran programación, es necesario el desarrollo de un sistema que integra los LMS de ahora a través del estándar LTI con Git, un editor de código en línea y un entorno para la ejeccución del mismo conlleva varias fases. En una fase inicial, es importante analizar la problemática, el entorno existente y características de sistemas similares para poder formular una solución facible y novedoso desde sus requisitos funcionales y diseño. Con un marco de diseño, se puede proceder a definir una arquiectura que asegurará que se cumple con los requisitos no funcionales. Con el entorno bien estudiado y un diseño robusto, se puede inicializar con varias fases ágiles e incrementales de desarrollo, pruebas y desplegue hasta llegar a un prototipo final que representa las funcionalidades basicas de la rafaga inicial de desarrollo.
%The Thesis Abstract is written here (and usually kept to just this page). The page is kept centered vertically so can expand into the blank space above the title too\ldots
\end{abstract}

%----------------------------------------------------------------------------------------
%	ACKNOWLEDGEMENTS
%----------------------------------------------------------------------------------------

\begin{acknowledgements}
%\addchaptertocentry{\acknowledgementname} % Add the acknowledgements to the table of contents
\addchaptertocentry{Agradecimientos} % Add the acknowledgements to the table of contents
Quiero agradecer a la Universidad Técnica Particular de Loja por darme un espacio para crecer y aprender lo justo y necesario para ser un buen profesional y persona que está al servicio de la sociedad.

Quiero agradecer a mi Director de Tesis, el Doctor Jorge López, quien, a pesar de mis limitaciones y amor para mi tesis ha logrado guiarme y controlar mi alcanze para que el reto de este trabajo de titulación sea alcanzable en el tiempo dado. Para mi siempre ha sido una gran voz de razón, me ha encaminado tanto a nivel profesional como personal. Además le quiero agradecer por brindarme la confianza y libertad para resolver los problemas planteados en mi trabajo de titulación.

Quiero agradecer al Ingeniero Rene Elizalde por darme aportarme las bases necesarias para entrar al hermoso mundo de Python y Django, que sin estos conocimientos que me dio, no hubiera tenido factibilidad técnica un proyecto de este alcance en el tiempo dado.

Quiero agradecer al Ingeniero Santiago Quiñones por motivarme a seguir estudiando, programando y por calificar de una forma muy justa el código de cada estudiante, no tanto por funcionalidad, si no, por calidad de lógica.

Quiero agradecer a mi profesor de historia mundial de colegio, Robert Peoples, quien me enseñó que lo más importante saber es el porqué de las cosas.

Quiero agradecer a mis maestros de matemáticas en especial a Ms. Pearman quien me brindó las bases necesarias para poder analizar problemas desde un punto analítico y computacional, una de los habilidades más importantes para cualquier ingeniero en sistemas.

Quiero agradecer a la Ingeniera Maria del Carmen Cabrera por su paciencia y disposición en la resolución oportuna de mis dudas y revisión de la tesis a lo largo de su desarrollo.

Quiero agradecer a la Ingeniera Pamela Guamán quien siempre ha sido una buena amiga, compañera y fuente de apoyo para acabar mi tesis a tiempo.

Quiero agradecer a la Ingeniera Leidy Yaguachi por su compañerismo y amistad, tanto en los buenos como en los malos momentos de la carrera.

Quiero agradecer a Marcelo Bravo, Galo Celly, Anita Cardenas y Dickson Armijos por su la amistad y apoyo a lo largo de la carrera. 

Quiero agradecer a mi abuelita que siempre, a pesar de tener dificultades económicas, siempre hizo el ahorro y sacrificio necesario para que podríamos ser compañeros de viaje y así conocer culturas y experiencias nuevas para tratar de formarme como una persona de mente abierta.

Quiero agradecer a mi abuelito por enseñarme a utilizar la computadora, a disfrutar usar la computadora, y de una manera muy inconsciente mostrarme la potencia de Linux para resolver problemas.

Quiero agradecer a mi papá por enseñarme que el éxito profesional está en la felicidad con lo que uno hace y eso solo se puede proveer de una fuerte ética profesional y conciencia social que busca el bien de la sociedad y todos sus miembros.

Quiero agradecer a mi mamá por darme la vida, nunca dejar mi lado y mostrarme paciencia y enseñarme el significado del amor incondicional.

Finalmente quiero agradecer a mi esposa Pao e hija Valerie por ser mi motivo de todos mis días para nunca dar nada menos que lo mejor de mi.
\end{acknowledgements}

%----------------------------------------------------------------------------------------
%	LIST OF CONTENTS/FIGURES/TABLES PAGES
%----------------------------------------------------------------------------------------

% TODO: look over accent marks
\tableofcontents % Prints the main table of contents

\listoffigures % Prints the list of figures

\listoftables % Prints the list of tables

%----------------------------------------------------------------------------------------
%	ABBREVIATIONS
%----------------------------------------------------------------------------------------

% TODO: Accents
\begin{abbreviations}{ll} % Include a list of abbreviations (a table of two columns)

\textbf{CSS} & \textbf{C}ascading \textbf{S}yle \textbf{S}heets (Hojas de Estilo en Cascada)\\
\textbf{FOSS} & \textbf{F}ree and \textbf{O}pen \textbf{S}ource \textbf{S}oftware (Software Abierto/Gratuito/Libre)\footnote{Significado abierto a debate} \\
\textbf{FLOSS} & \textbf{F}ree/\textbf{L}ibre and \textbf{O}pen \textbf{S}ource \textbf{S}oftware (Software Abierto y Libre) \\
\textbf{GNU} & \textbf{G}NU is \textbf{N}ot \textbf{U}NIX (GNU no es UNIX)\\
\textbf{GPL} & \textbf{G}NU \textbf{P}ublic \textbf{L}icence (Licencia Publica de GNU)\\
\textbf{GPLv3} & \textbf{GPL} \textbf{v}ersion \textbf{3} (Version 3 de la GPL)\\
\textbf{HTML} & \textbf{H}yper\textbf{T}ext \textbf{M}arkup \textbf{L}anguage (Lenguaje de Marcas de HiperTexto)\\
\textbf{HTTP} & \textbf{H}yper\textbf{T}ext \textbf{T}ransfer \textbf{P}rotocol (Protocolo de Transferencia de HiperTexto)\\
\textbf{HTTPS} & \textbf{H}yper\textbf{T}ext \textbf{T}ransfer \textbf{P}rotocol \textbf{S}ec
ure (Protocolo Seguro de Transferencia de HiperTexto)\\
\textbf{IaaS} & \textbf{I}nfrastructure \textbf{a}s \textbf{a} \textbf{S}ervice (Infrastructura como un Servicio)\\
\textbf{IAENG} & \textbf{I}nternational \textbf{A}ssociation of \textbf{Eng}ineers \\
\textbf{IDE} & \textbf{I}ntegrated \textbf{D}evelopment \textbf{E}nvironment (Entorno de Desarrollo Integrado)\\
\textbf{IMAP} & \textbf{I}nternet \textbf{M}essage \textbf{A}ccess \textbf{P}rotocol (Protocolo de Accesso de Mensajes de Internet)\\
\textbf{IMAPS} & \textbf{IMAP} over \textbf{S}SL (IMAP atravez de SSL)\\
\textbf{IPSec} & \textbf{I}nternet \textbf{P}rotocol \textbf{Sec}urity (Seguridad de Protocolos de Internet)\\
\textbf{JWT} & \textbf{J}SON \textbf{W}eb \textbf{T}okens, un estandar para validar clientes en entornos desconfiables, definido en RFC 7519 \\
\textbf{KLOC} & \textbf{K}ilo-\textbf{LOC} (mil lineas de codigo)\\
\textbf{L2F} & \textbf{L}ayer \textbf{2} \textbf{F}orwarding (Renvio de Capa 2)\\
\textbf{L2TP} & \textbf{L}ayer \textbf{2} \textbf{T}unneling \textbf{P}rotocol (Protocolo de Tunel de Capa 2)\\
\textbf{LA} & \textbf{L}earning \textbf{A}sset (Activo de Aprendizaje)\\
\textbf{LAN} & \textbf{L}ocal \textbf{A}rea \textbf{N}etwork (Red de Area Local)\\
\textbf{\LaTeX} & \textbf{La}mport \textbf{\TeX} (sistema de tipografia \TeX por Lamport)\\
\textbf{LDAP} & \textbf{L}ightweight \textbf{D}irectory \textbf{A}ccess \textbf{P}rotocol (Protocolo Ligero de Acceso a Directorios)\\
\textbf{LMS} & \textbf{L}earning \textbf{M}anagement \textbf{S}ystem (Sistema de Gestion de Aprendizaje)\\
\textbf{LO} & \textbf{L}earning \textbf{O}bject (Objeto de Aprendizaje)\\
\textbf{LOC} & \textbf{L}ine \textbf{o}f \textbf{C}ode [measurement] (medida, linea de codigo) \\
\textbf{LOR} & \textbf{L}earning \textbf{O}bjects \textbf{R}epository (Repositorio de Objetos de Aprendizaje)\\
\textbf{LTI} & \textbf{L}earning \textbf{T}ools \textbf{I}nteroperability (Interoperabilidad entre Harramientas de Aprendizaje)\\
\textbf{MOOC} & \textbf{M}assive \textbf{O}nline \textbf{O}pen \textbf{C}ourse (Curso Online Masivo Abierto)\\
\textbf{OAS} & \textbf{O}pen \textbf{A}doption \textbf{S}oftware (Software de Adoptacion Abierta)\\
\textbf{PPTP} & \textbf{P}oint to \textbf{P}oint \textbf{T}unneling \textbf{P}rotocol (Protocolo de Tunel de Punto a Punto)\\
\textbf{RHEL} & \textbf{R}ed\textbf{H}at \textbf{E}nterprise \textbf{L}inux\\
\textbf{SSH} & \textbf{S}ecure \textbf{SH}ell (Inteprete Seguro de Commandos)\\
\textbf{SSL} & \textbf{S}ecure \textbf{S}ocket \textbf{L}ayer (Capa de Puertos Seguros)\\
\textbf{\TeX} & Sistema de \textbf{T}ipografia por Donald Knuth\\
\textbf{TLS} & \textbf{T}ransport \textbf{L}evel \textbf{S}ecurity (Seguridad en la Capa de Transporte)\\
\textbf{SMTP} & \textbf{S}imple \textbf{M}ail \textbf{T}ransfer \textbf{P}rotocol (Protocolo para Transferencia Simple de Correo)\\
\textbf{SMTPS} & \textbf{SMTP} over \textbf{S}SL (SMTP atravez de SSL)\\
%\textbf{UNIX} & Una familia de Sistemas Operativos que siguen un modelo historico de Cliente-Servidor\\
\textbf{UTPL} & \textbf{U}niversidad \textbf{T}ecnica \textbf{P}articular de \textbf{L}oja\\
\textbf{VCS} & \textbf{V}ersion \textbf{C}ontrol \textbf{S}ystem (Sistema de Control de Versiones)\\
\textbf{VPN} & \textbf{V}irtual \textbf{P}rivate \textbf{N}etwork (Red Privada Virtual)\\
\textbf{WYSIWYG} & \textbf{W}hat \textbf{Y}ou \textbf{S}ee \textbf{I}s \textbf{W}hat \textbf{Y}ou \textbf{G}et (Lo Que Ves Es Lo Que Obtienes)\\
\textbf{WWW} & \textbf{W}orld \textbf{W}ide \textbf{W}eb (Red Informatica Mundial)\\

%\textbf{LAH} & \textbf{L}ist \textbf{A}bbreviations \textbf{H}ere\\
%\textbf{WSF} & \textbf{W}hat (it) \textbf{S}tands \textbf{F}or\\

\end{abbreviations}

%----------------------------------------------------------------------------------------
%	PHYSICAL CONSTANTS/OTHER DEFINITIONS
%----------------------------------------------------------------------------------------

%\begin{constants}{lr@{${}={}$}l} % The list of physical constants is a three column table

% The \SI{}{} command is provided by the siunitx package, see its documentation for instructions on how to use it

%Speed of Light & $c_{0}$ & \SI{2.99792458e8}{\meter\per\second} (exact)\\
%Constant Name & $Symbol$ & $Constant Value$ with units\\

%\end{constants}

%----------------------------------------------------------------------------------------
%	SYMBOLS
%----------------------------------------------------------------------------------------

%\begin{symbols}{lll} % Include a list of Symbols (a three column table)

%$a$ & distance & \si{\meter} \\
%$P$ & power & \si{\watt} (\si{\joule\per\second}) \\
%Symbol & Name & Unit \\

%\addlinespace % Gap to separate the Roman symbols from the Greek

%$\omega$ & angular frequency & \si{\radian} \\

%\end{symbols}

%----------------------------------------------------------------------------------------
%	DEDICATION
%----------------------------------------------------------------------------------------

%\dedicatory{For/Dedicated to/To my\ldots} 
\dedicatory{Dedicado a las tres mujeres más importantes en mi vida, mi hija Valerie, mi esposa Paola y mi madre Diana\ldots} 

%----------------------------------------------------------------------------------------
%	THESIS CONTENT - CHAPTERS
%----------------------------------------------------------------------------------------

\mainmatter % Begin numeric (1,2,3...) page numbering

\pagestyle{thesis} % Return the page headers back to the "thesis" style

% Include the chapters of the thesis as separate files from the Chapters folder
% Uncomment the lines as you write the chapters

% Chapter 1

\chapter{Introducción}
\label{capitulo1}

%\chapter{Chapter Title Here} % Main chapter title
%
%\label{Chapter1} % For referencing the chapter elsewhere, use \ref{Chapter1} 

%----------------------------------------------------------------------------------------

% Define some commands to keep the formatting separated from the content 
\newcommand{\keyword}[1]{\textbf{#1}}
\newcommand{\tabhead}[1]{\textbf{#1}}
\newcommand{\code}[1]{\texttt{#1}}
\newcommand{\file}[1]{\texttt{\bfseries#1}}
\newcommand{\option}[1]{\texttt{\itshape#1}}

%----------------------------------------------------------------------------------------

% TODO: "ni *para* estudiantes" pg. 1
Hoy en día se encuentra una necesidad creciente para programadores, informáticos y personas que pueden leer y entender código debido a una tendencia a tratar de siempre automatizar a un mayor grado las tareas humanas con la tecnología. Actualmente la Universidad Técnica Particular de Loja cuenta con métodos clásicos y manuales para enseñar y calificar código que estudiantes programan lo cual no es beneficioso para los docentes, quienes pierden mucho tiempo realizando tareas que se podría realizar de forma automática, ni estudiantes quienes tengan menos atención de sus docentes debido a que tienen mucho código que calificar.
 
Actualmente, en el mercado existen varios sistemas que permiten escribir, ejecutar y probar código en línea. Uno de estos es Repl.it que puede integrase con sistemas que disponen del protocolo de integración entre sistemas de aprendizaje LTI \index{LTI} \citep{Repl.it-Home} y puede calificar el código de estudiantes en base a pruebas unitarias pero el mismo tiene un alto coste que no se escala bien y no está orientando tanto al aprendizaje o colaboración si no a evaluación \citep{Lopez-Jorge}. Por otro lado, Io.livecode.ch es un sistema libre que utiliza contenedores de Docker para permitir la ejecución de código en línea, pero el problema de este sistema es que solo está orientado a enseñanza de programación mas no a realizar pruebas en línea y además no implementa nada de seguridad para proteger el servidor contra usuarios maliciosos \citep{io.livecode.ch}. Cloud9 IDE es otra herramienta popular para programar en línea, éste integra máquinas virtuales de Ubuntu con el mismo para permitir hacer pruebas en tiempo real \citep{Cloud9-Home}, pero el mismo tampoco es factible por su falta de interoperabilidad con LTI \index{LTI} y falta de enfoque ni en enseñanza ni en evaluación y calificación. 

% TODO: --con fines comerciales--
Como cualquier otro tipo de institución con fines comerciales, las universidades e instituciones educativas siempre deben estar buscando la manera en que pueden mantener su competitividad para dar la mejor educación posible y al menor costo. La tecnología de hoy en día ha alcanzado una madurez en donde puede apoyar en temas de automatizar algunos procesos dentro de la enseñanza, evaluación y calificación de estudiantes y de la misma forma liberar el tiempo de professores para que estos puedan ayudar a aquellos estudiantes que realmente necesitan un poco más de apoyo.
 
Frente a este problema, se propone un sistema para editar código en línea que a su vez integra LMS \index{LMS} externos (para autenticación y notas), un servidor de control de versiones externo (para la persistencia de código), y un servicio web de ejecución de código en línea de una forma segura, eficaz y eficiente (para dar un ambiente de ejecución y pruebas tanto para los usuarios del sistema como para calificar de forma automática). A continuación se presenta la problemática para definir el contexto del tema frente al cual se espera resolver dentro de este trabajo de titulación. 

% TODO: "Computación--,-- Electrónica, e Informática --especialmente los que estudian a distancia--." pg. 2
\section{Problemática}
Dentro de la universidad, existe la necesidad de contar con una mejor forma de enseñar y evaluar conocimientos de programación en los estudiantes de las titulaciones de Sistemas Informáticos y Computación y Electrónica, especialmente quienes estudian a distancia. Para solucionar este problema, existen sistemas alternativos que se analizan en el capítulo 2, pero estos resultan demasiado costosos para su implementación completa con los recursos actualmente disponibles.
 
Se considera que cualquier sistema implementado como solución debe ser capaz de captar información acerca de su propio uso con un fin de ayudar con la administración del mismo, debe guardar de manera segura y confiable el código escrito dentro del mismo. Así mismo, debe proporcionar de las herramientas que requieren los estudiantes y sus profesores para desarrollar, probar y calificar de manera eficaz, segura y eficiente el código escrito en línea. Además como plataforma en línea se considera que se podría aprovechar para temas de colaboración como: programación en pares y otras técnicas que apliquen los equipos de desarrollo para cumplir con sus responsabilidades de una forma paralela.

\section{Metodología}
La fase de investigación consiste en realizar una breve recopilación de lo que hay dentro del entorno de despliegue para realizar un estudio del mismo. Además se realiza un estudio de aquellas tecnologías que se utiliza dentro de la solución planteada. Para la investigación de todos los puntos anteriores, se basa en fuentes confiables como: fuentes académicos o comerciales con un fin de explicar alguna tecnología para vender la misma.

% TODO: ERS -> Especificación de Requeriemientos de Software pg. 2
Después de la fase de investigación, donde se formula buenas bases y entendimiento del contexto del ambiente en el que se encuentra el presente trabajo, se empieza una fase de análisis el cual se basa en un documento de visión que desglosa las necesidades existentes para el sistema, seguidamente un ERS o especificación de requerimientos demuestra de forma más exacta las capacidades que debe tener el sistema.
 
Una vez que se tiene entendidas las capacidades que debe tener el sistema, se procede a una fase de diseño donde se realiza diagramas para definir la arquitectura, despliegue e interacción que el sistema tendrá con los usuarios.
 
Con la definición de los requerimientos y el diseño del sistema como tal, lo que sigue es una fase de desarrollo la misma que se plantea llevar bajo una metodología de desarrollo iterativo incremental que se caracteriza por la revisión contínua con los stakeholders principales del proyecto. De esta forma se puede mantener un poco de agilidad en el desarrollo y consecuentemente desarrollar un producto que cumpla con las necesidades de los usuarios mientras se mantiene dentro de los límites de tiempo y recursos preestablecidos.
 
Además se propone tener solo iteraciones (basados en entregables) que duren máximo un mes, esto se lo propone para los dos entregables más grandes; el subsistema de ejecución de código por su complejidad y el capítulo seis que se trata del despliegue de la aplicación final. Las otras iteraciones son de dos semanas para tratar de mantener el proyecto en un estado fluido de actividad y presentación de entregables de forma contínua.
 
Las pruebas se llevaran igual en fases, donde primero en una prueba alfa se integra con sistemas reales para probar temas de integración. En base a los resultados de estas pruebas se realiza recomendaciones y mejoras previo a la próxima fase de pruebas que son constituidas por pruebas beta donde se prueba la aplicación en un entorno real con usuarios finales, tanto estudiantes como profesores para que puedan dar retroalimentación previo al despliegue de una versión final.
 
Finalmente se trabaja juntamente con los que serán encargados de mantener la aplicación para el despliegue del mismo. Esto se lleva en fases en donde se levantan primero las dependencias necesarias, como bases de datos y servicios externos entre otros previo al levantamiento y configuración del sistema. La última fase del despliegue consiste en levantar los proxies y servicios auxiliares que ayuden a mejorar temas de seguridad y rendimiento a la aplicación desplegado.

\section{Objetivos}
El objetivo principal es mejorar la enseñanza y evaluación de estudiantes de programación de la Universidad Técnica Particular de Loja. Para cumplir este objetivo se propone un sistema de edición y ejecución de código en línea que ayude a los docentes y estudiantes a interactuar 
%a calificar código que escriben sus estudiantes en base a pruebas unitarias
e integrar funcionalidades de sistemas LMS \index{LMS} institucionales adicionalmente de servidores de control de versiones institucionales. Para alcanzar este fin, también se plantea cumplir con algunos objetivos específicos en vía al principal. \index{Editar Código|textit}

% TODO: explicaciones de cada objetivo especifico debe ir antes o despues del listado de los mismos, no alli integrado pg. 3
\begin{itemize}
	\item Facilitar la autenticación de los estudiantes para que puedan entrar directamente desde sistemas LMS \index{LMS} institucionales a través del protocolo LTI\index{LTI}. Esto es necesario con la finalidad de que los estudiantes puedan tener una transición suave entre sistemas y no tener mayor interrupción en su aprendizaje. \index{Autenticación LTI}
	\item \index{Ejecutar Código|(} Proveer la capacidad de poder ejecutar código en línea. Con esta funcionalidad, el sistema se acerca más a un entorno completo de desarrollo donde estudiantes requieren un mínimo de programas instalados en sus equipos personales. \index{Ejecutar Código|)}
	%\item \index{Calificar Código|(} Autocalificar código en base a pruebas unitarias. El fin de esta característica es que el sistema sea un apoyo a los docentes en ayudarles automatizar y por lo tanto optimizar su flujo de calificar para que tengan más tiempo para estar enseñando y ayudando sus alumnos. Las mismas pruebas unitarias, por la forma en que automatizan la forma en que se pueden usar para calificar código en base a las funcionalidades que tiene y cómo reacciona frente ciertas entradas, también hace el proceso completamente objetivo debido a que elimina el factor humano del proceso. \index{Calificar Código|)}
	\item \index{Controlar Versiones de Código|(} Utilizar servidores de sistemas de control de versiones para persistir el código creado por estudiantes de forma externa. Eso permitirá, que de una forma transparente y sin mayor interacción del usuario, ver un historial de código escrito por sus usuarios en la plataforma y también servir de respaldo contínuo de los trabajos que realizan los estudiantes. \index{Persistir Código|textit} \index{Controlar Versiones de Código|)}
	%\item \index{Recolección de Datos|(} \index{Toma de Decisiones|(} Recolectar datos para la toma de decisiones estratégicas. Esta característica permite a la administración de la aplicación sea la universidad o equipo de mantenimiento sacar datos y estadísticas de uso de la aplicación para en base a ello realizar análisis de cómo se está siendo usado, que son los comportamientos de los estudiantes y otras decisiones estratégicas que se pueden tomar en base a los datos de uso del sistema. \index{Recolección de Datos|)} \index{Toma de Decisiones|)}
\end{itemize}

\section{Resultados Esperados}
En el presente trabajo de titulación pretende obtener los siguientes resultados:
% TODO: eliminar todos los resultados esperados? pg. 3
\begin{description}
	\item[Sistema Prototipo para Editar Código en Línea] que permita a los estudiantes y sus docentes tener una herramienta para mejorar el proceso de enseñanza, aprendizaje y evaluación de la programación. \index{Editar Código}  
 \item[Sistema Prototipo para Ejecutar Código en Línea que se integró con lo anterior] y facilita las pruebas que requieren los estudiantes o profesores con respeto a algún código al cual tengan acceso. \index{Ejecutar Código}
 %\item[Repositorio de datos recolectados para la toma de decisiones estratégicos] que permita a la titulación ver cómo se está utilizando el sistema y en base a ello planificar decisiones futuras. \index{Recolección de Datos} \index{Toma de Decisiones}
 %\item[Manual de Programador] que sea una ayuda para futuros desarrolladores que deseen extender la funcionalidades del sistema inicial.
 %\item[Manual de Usuario] que apoya a estudiantes y docentes que quieren aprender a usar el nuevo sistema.
 %\item[Manual de Mantenimiento/Administrador] que ayuda al equipo de mantenimiento poder mantener la aplicación al corto, mediano y largo plazo.
\end{description}

\section{Organización del Documento}
% TODO: --se organiza el documento y-- pg. 4
Dentro del capítulo uno, se introduce el tema y problemática, se organiza el documento y se plantea la metodología y objetivos que guían el proyecto desde su inicio hasta su fin.

% TODO: --ve el estado del arte que contenga-- *realiza* pg. 4
% --dentro del capítulo dos se--
% --un subsistema del estado del arte en la forma de--
En el capítulo dos, se ve el estado del arte que contenga un análisis de trabajos relacionados y sistemas similares a la que se plantea desarrollar. También dentro del capítulo dos se encuentra un subsistema del estado del arte en la forma de marco teórico, el cual desglosa la situación actual y permite ver la teoría de las tecnologías de apoyo para el sistema final.

En el capítulo tres, se da un análisis profundo del problema para entenderlo de forma completa previa a las fases siguientes. Dentro del mismo capítulo, en base al análisis, se plantea un diseño para el producto final, el cual se refleja más adelante en el desarrollo durante el curso del capítulo cuatro.

% TODO: --LTI que permite a los estudiantes autenticarse contra sistemas LMS ya existentes, un módulo de -- pg. 4
% --un módulo de --
% --institucionales--
% --un subsistema de--
% --un módulo de--
% --para calificaciones automáticas--
A lo largo del capítulo cuatro, se lleva al lector por el proceso de desarrollo. Por lo tanto el mismo se divide en los varios módulos que se necesitan desarrollar para el proyecto; un módulo de autenticación LTI \index{LTI} que permite a los estudiantes autenticarse contra sistemas LMS \index{LMS} ya existentes, un módulo de editar código en línea, un módulo de persistencia de código en servidores de control de versiones institucionales, un subsistema de ejecución de código en línea, un módulo de pruebas unitarias para calificaciones automáticas y un módulo para sincronizar calificaciones con sistemas LMS \index{LMS} institucionales.
 
El quinto capítulo se basa en la fase anterior de desarrollo para planificar y realizar pruebas de funcionamiento e integración. De esta manera se encuentran las fallas existentes para resolver los mismos antes de seguir con el despliegue en el sexto capítulo.
 
A continuación, en el capítulo sexto se planifica y se despliega la aplicación final para que pueda empezar su vida útil dentro de la universidad.
 
Dentro del séptimo capítulo se resume los resultados y se concluye el trabajo realizado en base al cumplimiento de los objetivos planteados dentro del capítulo uno. Además se da recomendaciones e ideas para trabajos futuros en base al proyecto realizado y para la evolución continua del prototipo final.

Al final de todos los capítulos anteriores se encuentran los anexos con documentos de apoyo como: el documento de vision (Anexo \ref{AnexoA}) y especificacion de requisitos (Anexo \ref{AnexoB}), implementacion tecnicas llevadas a lo largo del desarollo (Anexos \ref{AnexoC}, \ref{AnexoD}, \ref{AnexoE}), pruebas realizadas para validar funcionalidades (Anexos \ref{AnexoF}, \ref{AnexoG}, \ref{AnexoH}), y ubicacion y licencias de codigo llevado en el curso de la tesis (Anexo \ref{AnexoI}).

% Chapter 2

\chapter{Estado del Arte}
\label{capitulo2}

\section{Trabajos Relacionados}
Dentro de la temática de este trabajo de titulación es importante entender trabajos relacionados los mismos que han sido realizados para en base a ellos entender investigaciones que ya se han realizado y de esta forma aprender de ellos.

\subsection{GitEduERP}
En un trabajo reciente del autor con sus colegas, frente el problema de necesitar ofrecer una ambiente de programación a estudiantes en línea, se realizó un editor de código en línea con sistema de permisos y la capacidad de compartir entre usuarios en un backend de Django y utilizando una librería ACE liberado por Cloud9 IDE que guardaba código editado en una instancia de GitLab CE. Además ofrece chat en línea con una liberia TogetherJS [UTPL-GitEduERP].

\subsection{Sistema de Encuestas Online}
Para mejorar temas de business analytics, se trató de diseñar procesos de negocio para realizar colección de datos en tiempo real a través de encuestas y analizar las mismas con un fin de ayudar la toma de decisiones estratégicas de negocio en tiempo real. Plantearon soluciones con un fin de optimizar tiempos y recursos a través del uso de soluciones tecnológicas [UTPL-Thesis-Encuestas-Online].

\subsection{Metodología de Enseñanza con la Web 2.0}
Se buscó utilizar la manera en que la moda de lectura-escritura en la Web 2.0 se podría crear cursos interactivos con estudiantes online como base de una nueva metodología de enseñanza. Destaca esos temas desde el punto de vista de ingeniería en sistemas para proporcionar soluciones netamente técnicas y estratégicas para dar el mejor aporte posible a quienes deseen implementar un sistema de este tipo [UTPL-Thesis-Edu-Web-2.0].

\subsection{Xen Web-based Terminal for Learning Virtualization and Cloud Computing Management}
Los autores, Abdullah Almurayh y Sudhanshu Semwal, propusieron e implementaron una arquitectura de cliente-servidor para la distribución de recursos educativos y en el proceso enseñar a estudiantes de Linux, programación orientada a la nube y gestión de nubes/servidores. Su aplicación web ofrece terminales SSH donde cada estudiante y docente disponía de una máquina virtual de tal forma que tenían su propio ambiente con permisos de superusuario y al mismo tiempo eran aislados de la infraestructura real y de los demás usuarios, motivo por el cual se podía dar una solución flexible y a su vez segura. Como hipervisor ocuparon un sistema de Xen que controlan remotamente con su servidor de aplicación (servidor web) [almurayh2014xen].

\subsection{Comparación de Trabajos Relacionados}
\begin{table}[h!]
    \begin{tabular}{|p{0.17\textwidth}|p{0.105\textwidth}|p{0.105\textwidth}|p{0.15\textwidth}|p{0.15\textwidth}|p{0.17\textwidth}|}
        \hline
            & Editar Código en Línea & Persistir Código en Línea & \mbox{Recolección} y \mbox{Análisis} de \mbox{Datos} para \mbox{decisiones} \mbox{estratégicas} en tiempo real & Metodología de \mbox{Enseñanza} Online & Ambientes Virtualizados \\
        \hline
        GitEduERP & x & x & & & \\
        \hline
        Sistema de Encuestas Online & & & x & & \\
        \hline
        Metodología de \mbox{Enseñanza} con la Web 2.0 & & & & x & \\
        \hline
        Xen Web-based Terminal for Learning \mbox{Virtualization} & x & x &  & x & x \\
        \hline
    \end{tabular}
	\caption{Comparación de Trabajos Relacionados.}
    \label{trabajos-relacionados-comparacion}
\end{table}

\section{Sistemas Similares}
De la misma manera donde se analiza investigaciones similares dentro de la parte de trabajos relacionados, es importante conocer también el contexto actual del mercado para ver las alternativas que ofrece la competencia y saber si ya hay alguna solución que sea de mayor beneficio a la universidad que el mismo sistema que se plantea (y por tal razón sería más factible implementar dicha solución en lugar de desarrollar algo nuevo).

\subsection{Repl.it}
Repl.it es una plataforma en línea para escribir y probar código en tiempo real como un IDE en línea. Su producto tiene un componente educativo que permite integración con otros sistemas de aprendizaje, organización por aulas, texto que guía las tareas, calificación automática de tareas a través de pruebas unitarias entre otras características que mejoran la interacción entre docentes y sus alumnos en su aprendizaje de código. Actualmente usa “máquinas completas de linux” para soportar “más de 30 lenguajes” de una manera “fiable y segura” [Repl.it-Home].

\subsection{io.livecode.ch}
io.livecode.ch es una plataforma prototipo que convierte repositorios públicos de GitHub en tutoriales interactivos y documentados de programación. Permite la ejecución de código en contenedores de Docker en tiempo real. El mismo utiliza máquinas virtuales alojados en DigitalOcean y permite la extensión por terceras personas con mas tutoriales [io.livecode.ch].

\subsection{Cloud9 IDE}
Cloud9 IDE ofrece un espacio de trabajo personal rápido y escalable (pero administrado por la misma empresa) basado en contenedores de Ubuntu encima de Docker para cada usuario, con soporte para 40 lenguajes de programación. Permite ver aplicaciones web en tiempo real con una variedad de navegadores y sistemas operativos. También permiten conectarse a servidores privados del usuario por SSH para utilizar estos en lugar de los servidores que ellos proveen. También tiene la capacidad de compartir código entre varios usuarios bajo permisos de lectura o lectura y escritura, permitiendo que los mismos editan en tiempo real (visible a todos) y que pueden comunicarse a través de chat. El mismo código, una vista previa de ello o su versión completa también se puede compartir públicamente con usuarios que no pertenecen a la plataforma. Todo esto es con un fin de reemplazar todo el ambiente de desarrollo local con un entorno completamente en la nube; ofrece un terminal, editores avanzados de código, división de pantalla, un debugger, temas, personalización de atajos, comandos communes, modo vim y modo sublime para el editor y un editor de imágenes [Cloud9-Home].

\subsection{GitLab}
GitLab en adición a ser un servidor de Git, y por lo tanto llevar un control de versiones de los datos que aloja, también ofrece otras características asociados con entornos profesionales de desarrollo como seguimiento de incidentes, revisión de código, un IDE para editar código en línea, la capacidad de tener wikis asociado con proyectos, un sistema de integración continua para probar, compilar y desplegar código en una variedad de ambientes. Su versión de pago también soporta el consumo de un servidor de LDAP para autenticar usuarios, hooks de Git para tomar acciones personalizadas en respuesta a eventos de Git y capacidades para auditoria por parte de administradores [GitLab]. Además, GitLab puede importar proyectos de otros plataformas y servidores de Git a través de una URI, gestión de snippets, ramas protegidas, un API que permite controlar al servidor de GitLab y un registro de contenedores de Docker asociados con cada proyecto [GitLab-Features].

\subsection{OverLeaf}
Overleaf es una plataforma en línea para editar y publicar de forma colaborativo documentos de LaTeX. Permite editar LaTeX directamente o usar un editor WYSIWG para quienes no conozcan bien el sistema TeX. La plataforma compila el documento de LaTeX en tiempo real como se lo va editando para disponer una vista previa con los últimos cambios y de la misma forma avisa de errores que genera al compilador. Como es una plataforma en línea, se puede editar el mismo documento varias personas al mismo tiempo desde distintos clases de dispositivos y cómo el sistema que respalda los documentos es un motor completo de LaTeX/TeX, permite una gran variedad de tipos de documentos y contenido en ellos [Overleaf]. También se integra Overleaf con Git ya que el mismo dispone de un servidor interno de Git para interactuar con sus proyectos a través de este sistema de control de versiones [Overleaf-Git].

\subsection{Google Drive}
Google Drive ofrece 15 Gigabytes de Almacenamiento gratis para guardar cualquier tipo de archivo, los mismos que pueden ser compartidos con cualquier persona para facilitar colaboración entre personas. Tiene para editar en línea documentos, hojas de cálculo, presentaciones, formularios (para hacer encuestas), dibujos y más con una API que permite mayor expansión por terceros. También controla las versiones de forma automática para que se puede volver a revisar versiones anteriores y quienes han introducido cambios y como fueron los mismos cambios. Además en caso de requerir acceder ciertos archivos sin internet, se puede señalar a la plataforma de Google Drive que se deben sincronizar en segundo plano cada vez que hay internet para también tener la disponible y actualizada cada vez que se encuentre sin conexión [Google-Drive-Usage].

% TODO
\subsection{Comparación de Sistemas Similares}
\begin{table}[h!]
    \begin{tabular}{|p{0.16\textwidth}|p{0.115\textwidth}|p{0.105\textwidth}|p{0.15\textwidth}|p{0.15\textwidth}|p{0.17\textwidth}|}
        \hline
            & Escribir Código Online & Probar Código Online & Integración LTI & Sistema de Autocalificación & Compilación en Tiempo Real \\
        \hline
        Repl.it & Si & Si & Si & Si & No \\
        \hline
        io.livecode.ch & Si & Si & No & No & No, pero si permite ejeccución de código en linea \\
        \hline
        Cloud9 IDE & Si & Si & No & No & Si para ciertos lenguajes \\
        \hline
        GitLab & Si & No & No & No & No \\
        \hline
        Overleaf & Solo \LaTeX & Solo \LaTeX & No & No & Si, \LaTeX \\
        \hline
        Google Drive & No, \mbox{solo} se lo \mbox{considera} \mbox{como} texto & No & No & No & No \\
        \hline
    \end{tabular}
	\caption{Comparación de Características Generales entre \mbox{Sistemas} Similares.}
    \label{comparacion-sistemas-similares-1}
\end{table}

\begin{table}[h!]
    \begin{tabular}{|p{0.16\textwidth}|p{0.105\textwidth}|p{0.16\textwidth}|p{0.15\textwidth}|p{0.15\textwidth}|p{0.13\textwidth}|}
        \hline
            & Control \mbox{Interno} de \mbox{Versiones} & Integración de algún \mbox{sistema} de \mbox{control} de versiones & Sistema de \mbox{Permisos} & Sistema de \mbox{Compartir} & Compartir / Editar en Tiempo Real \\
        \hline
        Repl.it & No & No & Si, basado en docente / alumno & No & No \\
        \hline
        io.livecode.ch & No & No & No & Todo es Público & No \\
        \hline
        Cloud9 IDE & Si & No & Si, \mbox{basado} en \mbox{usuarios} & Si, \mbox{basado} en \mbox{usuarios} & Si \\
        \hline
        GitLab & Si & Si, Git & Si, \mbox{basado} en \mbox{usuarios} y grupos & Si, \mbox{basado} en \mbox{usuarios} y grupos & No \\
        \hline
        Overleaf & Si & Si, Git & Si, \mbox{basado} en \mbox{usuarios} & Si, \mbox{basado} en \mbox{usuarios} & Si \\
        \hline
        Google Drive & Si & No & Si, \mbox{basado} en \mbox{usuarios} & Si, \mbox{basado} en \mbox{usuarios} & Si \\
        \hline
    \end{tabular}
	\caption{Comparación de Características Sociales entre Sistemas Similares.}
    \label{comparacion-sistemas-similares-2}
\end{table}

\begin{table}[h!]
    \begin{tabular}{|p{0.08\textwidth}|p{0.17\textwidth}|p{0.16\textwidth}|p{0.2\textwidth}|p{0.16\textwidth}|p{0.21\textwidth}|}
        \hline
            & Control Completo sobre \mbox{ambiente} de ejecución de código & Acceso \mbox{Remoto} al ambiente de Ejecución de Código & Sistema de Documentación & API sobre HTTP(S) & Acceso fuera de línea \\
        \hline
        Repl.it & Si para \mbox{instalación} de \mbox{dependencias} & No & Si, \mbox{documentación} del ejercicio & Si, \mbox{pero} \mbox{está} \mbox{cerrado} a nuevos clientes & No \\
        \hline
        io\ldots\footnote{io.livecode.ch} & Si, script de Bash & No en \mbox{tiempo} real & Si, HTML & No \mbox{mantiene} estados & No \\
        \hline
        Cloud9 IDE & Si, Terminal & Si, SSH & No, fuera de \mbox{documentos} en el servidor & Si & No \\
        \hline
        GitLab & No & No & Si, Wikis con GitHub Markdown & Si & Si, si es que se ha bajado todo anteriormente con Git \\
        \hline
        Ov\ldots\footnote{Overleaf} & No & No & No, todo el sistema es de \mbox{documentación} & Si & Si, si es que se ha bajado todo anteriormente con Git \\
        \hline
        Google Drive & No & No & No, todo el sistema es de \mbox{documentación} & Si & Si \\
        \hline
    \end{tabular}
	\caption{Comparación de Características Avanzadas entre \mbox{Sistemas} Similares.}
    \label{comparacion-sistemas-similares-3}
\end{table}

\section{Marco Teorico}
Para llevar a cabo de forma exitosa el trabajo de titulación y entender el contexto del ambiente en que sea implementado es importante establecer una línea base de conocimiento que se ve a continuación y dividido de la siguiente forma:
\begin{description}
	\item[Aspectos de Propiedad Intelectual] que se topa con las licencias y culturas de código abierto y libre que se encuentra actualmente en el entorno.
    \item[Aspectos Ambientales del Entorno de Desarrollo] se trata de dar mejor contexto al entorno en el cual se encuentra la aplicación planteada. Esto incluye pero no está limitado a conceptos teóricos, aplicaciones y protocolos.
\end{description}

\subsection{Aspectos de Propiedad Intelectual}
Hoy en día, en un mundo cada vez más conectado y con cada vez más informacion compartida entre personas distintas, es importante conocer bien temas de derechos de autor y propiedad intelectual, no sólo para desarrollar y despues dar licencia a un trabajo de titulación, si no también para entender lo que se puede y no se puede hacer dentro de un entorno de desarrollo que cada vez involucra más algún codigo o trabajo que fue desarrollado alguna licencia abierta o libre.

\subsubsection{Software Libre}
Software Libre es un movimiento sociopolítico que busca liberar códigos fuentes. Se divide en dos campos que son: los de Open Source (Fuentes Libres) que quieren liberar código fuente sólo en base a los méritos de desarrollo colaborativo y los de Free/Libre Software (Software Libre) que buscan liberar código fuente en base a ciertos derechos de compartir, colaborar y tener control de lo que hacen sus dispositivos que se les pertenecen (su propiedad) para todos los usuarios que ocupan el software [GNU-FLOSS-vs-FOSS] [GNU-Open-vs-Free]. En el contexto de esta tesis, Software Libre se refiere al segundo campo, no al primero. Para el mismo, la fundación GNU con su fundador Richard M. Stallman define 4 libertades mínimas que deben ser cumplidas para constituir Software Libre y los cuales se los enumera desde el 0 hasta el 3 [GNU-Freedom] [GNU-Free-Software]:

\begin{enumerate}
\item La libertad de ejecutar el programa para cualquier propósito que desee
\item La libertad de estudiar cómo funciona el programa y poderlo modificar como desee. Un requisito para esta libertad es que el usuario tenga acceso al código fuente original.
\item La libertad de distribuir copias del programa original con terceros.
\item La libertad de distribuir copias de sus versiones modificadas a terceros. Un requisito para esta libertad es que el usuario tenga acceso al código fuente original.
\end{enumerate}

[GNU-Freedom] [GNU-Free-Software]

Donde el software no cumpla con uno de los anteriores, ya deja de ser considerado libre. Sólo por ser software libre no significa que no puede ser comercializado, únicamente se requiere nuevos modelos de negocio los cuales si se los puede encontrar en uso diario [GNU-Free-Software]. Como sociedad, Stallman argumenta, es importante proteger esas libertades porque avanzan la humanidad, como la libertad de expresión, ya que las libertades que se proponen proteger, buscan sostener una sociedad que trabaja por el bien de todos, no de solo unos pocos que saben más de la funcionalidad de ciertos aspectos tecnológicos [GNU-Open-vs-Free]. En la época actual de gobiernos y empresas que espían sin vergüenza, se ha revivido el movimiento político y se ha convertido en algo más necesario debido a que todos tenemos y  queremos nuestro derecho a la privacidad [GNU-Freedom].

\paragraph{OAS}
OAS o Adopción de Software Libre es la tendencia que hay en el mundo empresarial de adoptar soluciones de tecnologías abiertas como los que ofrecen el mundo de software libre ya que las mismas pueden llegar a ser superiores en cuanto su eficacia y costo, que lo que ofrece su competencia comercial. Se estima que más del 78\% de instituciones usan software  de Fuentes Abiertas y menos del 3\% indican que no utilizan nada de software de fuentes abiertas. Eso demuestra un enorme mercado creciente y emergente a nivel mundial [ACCEL-OAS].

\subsection{Aspectos Ambientales del Entorno de Desarrollo}

\subsubsection{Tipos de Recurso de Aprendizaje}

\subsubsection{MOOC}
\subsubsection{CMS}
\subsubsection{LMS}
\subsubsection{LTI}

\subsubsection{Sistema de Control de Versionamiento}
\subsubsection{Git}

\subsubsection{Virtualización}
\subsubsection{Hipervisores}

\subsubsection{Protocolo de Túnel}

\subsubsection{Criptografía}

\subsubsection{SSH}

\subsubsection{SSL}

\subsubsection{URI}

\subsubsection{HTTP}

% Chapter 3

\chapter{Analisis y Diseño}
\label{capitulo3}

\section{Analisis}

\subsection{Vision}

\subsection{Especificacion de Requerimientos}

\section{Diseño}

\subsection{Arquitectura}

\subsection{Sistemas, Subsistemas y Modulos}

\subsection{Diagrama de Desplegue}

% Chapter 4

\chapter{Desarrollo}
\label{capitulo4}

\section{Preperaciones del Ambiente de Desarrollo}

\section{GitEDU}

\subsection{Autenticacion por LTI}

\subsection{Editor de Codigo en Linea}

\subsection{Persistencia de Codigo}

\subsection{Sincronizacion de Notas por LTI}

\subsection{API Externa}

\section{EduNube}

\subsection{Ejeccucion de Codigo en Linea}

\subsection{Calificacion Automatizada con Pruebas Unitarias}

\subsection{API Externa}

%% Chapter 5

\chapter{Pruebas}
\label{capitulo5}

\section{Preperaciones del Ambiente de Pruebas}

\section{Plan de Pruebas}

\section{Pruebas Funcionales}

\subsection{GitEDU}

\subsection{EduNube}

\section{Pruebas de Integracion}

\section{Resultados}


%% Chapter 6

\chapter{Despliegue}
\label{capitulo6}

\section{Plan de Despliegue}
En el curso del desarrollo, depuración y ejecución de pruebas llegó a ser evidente en enorme consumo de recursos necesarios para levantar todos los servicios desarrollados y auxiliares a la vez, razón por el cual, se ha visto la necesidad de un rediseño de forma mas liviana de la manera en que se despliegue todos los componentes integrados a la aplicación para con ello terminar las fases mencionadas anteriormente y dar paso a ejecución en ambientes de pocos recursos, tanto en el ámbito de desarrollo como de producción. En la implementación original de la arquitectura física para el desarrollo, se había planteado una colección de maquinas virtuales quienes representarían servidores distintas dentro de una red, pero para temas de optimizar recursos, se ha realizado los siguientes cambios los cuales se documentan a lo largo de este capitulo:
\begin{itemize}
	\index{Kubernetes}
	\item Kubernetes (backend de virtualización) se ubica en una maquina virtual de Xen (con paravirtualización activa para aumentar el rendimiento del sistema virtualizado) dado que el mismo, por temas de seguridad, debe seguir de una forma aislada. De esta forma se simula que esta en un servidor aparte, sea físico o virtual (se recomienda mejor virtualizar todo el cluster final de Kubernetes en algún hipervisor de tipo 1 para garantizar mayor seguridad y aislamiento) pero sin la mayor parte de las perdidas de rendimiento que se dio con MiniKube (alojado en VirtualBox).
    \index{Docker}
    \item Los servicios auxiliares en lugar de ser maquinas virtuales de Xen ahora son contenedores de Docker.
    \index{SystemD}
    \item Los servicios desarrollados se los han convertido en servicios (Systemd) del sistema operativo, no tanto por temas de rendimiento (aunque se podría mejorar el rendimiento de esta forma, asignándoles a un usuario con mayor prioridad de ejecución como el usuario root\footnote{Pero realizarlo esta afuera del alcance de este tesis, una solución así de software tampoco puede cambiar los recursos físicos de la maquina.}\footnote{Nunca se debe asignar un servicio que se consume en la red externa a algún usuario con privilegios de superusuario, como root, ya que el mismo abre todo el sistema operativo a un nuevo vector de ataque por el mismo servicio. En este caso debe ser un nuevo usuario, preferiblemente uno por cada servicio, con acceso restringido que tiene mayor prioridad de ejecución para sus procesos.}) si no por temas de facilitar la administración del mismo.
    \index{NGinX}
    \item La integración de los servicios con NGinX para que el mismo puede protegerlos y operar en la capacidad de proxy inversa, proxy de terminación SSL/TLS, servidor de archivos estáticos y validador de peticiones sin mayor perdida de rendimiento.
\end{itemize}
% TODO: diagrama de red/fisica

\section{Preparaciones del Ambiente de Despliegue}
Para preparar el ambiente de despliegue se lo ve pertinente planificar direcciones IP, dominios y puertos para los varios servicios que requieren ser levantados:
\begin{description}
	\item[Maquina Física] con Debian 9.x para Servicios y Docker: \textit{10.10.10.1}
    \begin{description}
    	\item[NGinX] \textit{http://0.0.0.0:80} \index{NGinX}
        \begin{itemize}
            \item \textit{http://10.10.10.1:80} \& \textit{http://git.localhost:80} -> GitWeb (HTTP)
            \item \textit{http://gitedu.localhost:80} -> GitEDU (HTTP)
            \item \textit{http://edunube.localhost:80} -> EduNube (HTTP)
            \item \textit{http://gitsrvendpoint.localhost:80} -> GitServerHTTPEndpoint (HTTP)
            \item \textit{http://gitlab.localhost:80} -> GitLab (HTTP)
            \item \textit{http://moodle.localhost:80} -> Moodle (HTTP)
            \item \textit{http://registry.localhost:80} -> Registry (HTTP)
        \end{itemize}
        \item[GitEDU] \textit{http://0.0.0.0:8000}
        \item[EduNube] \textit{http://0.0.0.0:8010}\footnote{Previamente el puerto asignado fue :8001, pero esto estaba en conflicto con el Dashboard/Proxy de Kubernetes, por lo tanto se ha cambiado este servicio al puerto :8010}
        \item[GitServerHTTPEndpoint] \textit{http://0.0.0.0:8020}\footnote{Para temas de consistencia con el cambio de puerto de EduNube, se ha cambiado el numero de puerto de :8002 a :8020}
        \item[Kubernetes Proxy] \textit{http://0.0.0.0:8001}
        \item[Docker] \textit{10.10.10.1}
        \begin{description}
        	\item[MySQL] \textit{0.0.0.0:3306} -> \textit{moodledb:3306} (TCP/MySQL)
        	\item[Moodle] \textit{0.0.0.0:8201} -> \textit{moodle:80} (HTTP)
        	\item[GitLab CE] \textit{10.10.10.1}
            \begin{itemize}
            	\item \textit{0.0.0.0:8143} -> \textit{gitlab:443} (HTTPS)
            	\item \textit{0.0.0.0:8101} -> \textit{gitlab:80} (HTTP)
            	\item \textit{0.0.0.0:8122} -> \textit{gitlab:22} (SSH)
            \end{itemize}
            \item[Registry] \textit{0.0.0.0:5000} -> \textit{registry:5000} (HTTP)
        \end{description}
    \end{description}
    \item[Maquina Virtual (Xen)] con Debian 9.x para el Cluster (solo 1 nodo maestro) de Kubernetes: \textit{10.10.10.12}
\end{description}

\subsubsection{Moodle en Docker}
La instalación de Moodle en Docker consiste de los siguientes pasos:
\begin{enumerate}
	\item Bajar el imagen de Docker oficial de MySQL:
    \begin{lstlisting}    
docker pull mysql
    \end{lstlisting}
    \item Bajar un imagen de Docker de Moodle:
    \begin{lstlisting}    
docker pull jauer/moodle
    \end{lstlisting}
    \item Ejecutar el imagen de MySQL en el fondo (-d) con un nombre posible de identificar (moodledb), paso de puertos (10.10.10.1:3306 -> moodledb:3306), un volumen de persistencia para el motor de base de datos, nombres de bases de datos, usuarios y contraseñas de MySQL y una petición de que siempre se reinicia el contenedor a lo que muere con algún error:
    \begin{lstlisting}    
docker run -d --name moodledb -p 3306:3306 -v \
/srv/moodle/mysql:/var/lib/mysql -e \
MYSQL_DATABASE=moodle -e MYSQL_ROOT_PASSWORD=moodle \
-e MYSQL_USER=moodle -e MYSQL_PASSWORD=moodle \
--restart always mysql
    \end{lstlisting}
    \item Ejecutar el imagen de Moodle con opciones similares al anterior, con un enlace a la base de datos con un dominio local de DB, metadatos del url en que debe responder y ocupar el puerto 8201 local como puerto 80 del contenedor (un passthrough):
    \begin{lstlisting}    
docker run -d -P --name moodle --link moodledb:DB\
-e MOODLE_URL=http://10.10.10.1:8201 -p 8201:80 \
-v /srv/moodle/data:/var/moodledata \
--restart always jhardison/moodle
    \end{lstlisting}
    \item Y finalmente visitamos http://10.10.10.1:8201/ para terminar la instalación inicial de Moodle y realizar su configuración inicial.
\end{enumerate}

\subsubsection{GitLab en Docker}
Para instalar GitLab en Docker, se sigue pasos similares a los vistos previamente con Moodle:
% TODO: Cite https://hub.docker.com/r/gitlab/gitlab-ce/
% TODO: Cite https://docs.gitlab.com/omnibus/docker/
\begin{enumerate}
	\item Bajar el imagen de Docker:
    \begin{lstlisting}    
docker pull gitlab/gitlab-ce
    \end{lstlisting}
    \item Ejecutar el imagen de Docker:
    \begin{lstlisting}    
docker run --detach --hostname 10.10.10.1 \
--publish 8143:443 --publish 8101:80 \
--publish 8122:22 --name gitlab --restart always \
--volume /srv/gitlab/config:/etc/gitlab \
--volume /srv/gitlab/logs:/var/log/gitlab \
--volume /srv/gitlab/data:/var/opt/gitlab \
gitlab/gitlab-ce:latest
    \end{lstlisting}
    \item Visitar http://10.10.10.1:8101/ para cambiar la contraseña de root
\end{enumerate}

\subsubsection{Registry en Docker}
A diferencia de el ambiente de desarrollo, que ocupa el Docker Registry de GitLab\footnote{Se puede encontrar en \url{https://gitlab.com/nishedcob/GitEDU/container\_registry}}, para el ambiente de despliegue se plantea replicar el servicio de forma local para reducir el consumo de red. Para ello, se utiliza un contenedor de Docker:
\begin{enumerate}
	\item Bajar el imagen de Docker:
    \begin{lstlisting}
docker pull registry
    \end{lstlisting}
    \item Ejecutar el imagen de Docker:
    \begin{lstlisting}
docker run -d -p 5000:5000 --restart always \
--name registry registry
    \end{lstlisting}
\end{enumerate}
El proyecto de Docker, en parte por su mala fama de ser insegura, toma muy enserio la seguridad. Por lo tanto, aunque se puede consumir normalmente un Docker Registry por HTTP con el dominio localhost, al ocupar cualquier otro dominio o direccion IP, el cliente de Docker por defecto exige que las llamadas al API en el servidor sean sobre HTTPS. Con la finalidad de no complicar la configuracion más y tambien por el hecho de que la arquitectura fisica del despliegue realmente no sale del localhost, aunque para Docker le parece que si (obviamente no tiene conocimiento de son maquinas virtuales a los cuales el acceso es por adaptadores de red virtuales, etc), se ha optado para realizar la adecuada configuracion al archivo \texttt{/etc/docker/daemon.json} en cada host que consume el Registry sobre HTTP para que los mismos no exigen HTTPS donde solo existe HTTP. La configuracion es lo siguiente (en este caso no existia el archivo JSON, entonces se lo creaba con los contenidos a continuacion, en otros casos se agregaria solo los valores abajo a los valores existentes):
\begin{lstlisting}
{
          "insecure-registries" : [
                  "10.10.10.1:5000"
          ]
}
\end{lstlisting}
Despues de realizar aquello configuracion es necesario reiniciar Docker \index{Docker} para que tome en cuenta los cambios con el commando \texttt{systemctl restart docker}. Para popular el registry con los contenedores necesarias para ejecuccion de codigo con el sistema EduNube, puede referirse a los commandos encontrados en \\
\texttt{docker/commands-local-doc.sh} del repositorio principal. Este script toma los contenedores compilados previamente durante el desarrollo, los prepara para subirse y se los sube al Registry para que otros consumidores del mismo Registry tengan acceso a ellos.

\index{Hipervisor} \index{Virtualización} \index{Contenedor} \index{Xen}
\subsubsection{Máquina Virtual de Xen}
Con el fin de proteger la maquina física contra usuarios finales, sean maliciosos o solo sin conocimientos adecuados, hay la necesidad de que el ambiente que ejecuta código sea aislado con virtualización. Pero el rendimiento de esta maquina virtual necesita ser maximizada para poder permitir su uso con un mínimo de recursos. Es, por lo tanto, que se ha propuesta utilizar una maquina virtual de Xen de baja nivel con mejores de rendimiento con la tecnología de paravirtualización con la finalidad de que esta misma logra ofrecer alta aislamiento, y por lo tanto seguridad, frente una alta rendimiento.

\index{Kubernetes}
\paragraph{Construcción de Servidor Virtualizado para Cluster de Kubernetes}
Las características mínimas que pide Kubernetes para su nodo maestro son 2 núcleos y 2 GiB de RAM, es por aquello razón que se crea una maquina virtual paravirtualización con estas mismas características. En el curso del año que se ha trabajado este trabajo de titulación la comunidad de Debian se ha logrado arreglar  el bug que antes causó problemas la ultima vez que se realizó una maquina virtual nueva de Debian 9 y es con este motivo que se puede crear una maquina virtual nueva sin ninguna necesidad para pasos adicionales:
\begin{lstlisting}
xen-create-image --hostname=debian-k8s-master \
--ip=10.10.10.12 --netmask=255.255.255.0 \
--gateway=10.10.10.1 --memory=2048mb --vcpus=2 \
--lvm=Xephyr-VG --pygrub --dist=stretch --force \
--size=10240mb --swap=1024mb
\end{lstlisting}
% TODO: include Screenshot_2017-12-29_15-21-49.png

\index{Docker}
\paragraph{Instalación de Docker}
Para utilizar Kubernetes dentro de esta maquina virtual es necesario primero contarnos con una instalación de Docker. Kubernetes no garantiza que va a funcionar con las ultimas versiones de Docker ya que el API de Docker cambia constantemente, especialmente con actualizaciones de seguridad, pero de esta misma forma para la seguridad de la misma, queremos contar con la misma forma estable. Se instala de la siguiente manera en Debian (para otros sistemas operativos, a lo mucho solo se tendría que cambiar el gestor de paquetes apt por el respectivo de su sistema):
\begin{enumerate}
	\item Actualizar el sistema operativo
    \begin{lstlisting}
apt update
apt upgrade
    \end{lstlisting}
    \item Instalar curl si es que no esta instalado previamente
    \begin{lstlisting}
apt install curl
    \end{lstlisting}
    \item Bajar el instalador actual de Docker
    \begin{lstlisting}
curl -fsSL get.docker.com -o get-docker.sh
    \end{lstlisting}
    \item Ejecutar el instalador de Docker
    \begin{lstlisting}
sh get-docker.sh
    \end{lstlisting}
\end{enumerate}
% TODO: include Screenshot_2017-12-29_15-31-49.png

Para utilizar el Registry creado anteriormente, se agrega el siguiente configuracion a las maquinas del cluster (seguido por el reinicio del servicio de Docker con \texttt{systemctl restart docker}):
\begin{lstlisting}
{
          "insecure-registries" : [
                  "10.10.10.1:5000"
          ]
}
\end{lstlisting}

\index{Kubernetes}
\paragraph{Instalación de Cluster de Kubernetes}
% TODO: Cite https://kubernetes.io/docs/setup/independent/install-kubeadm/
Para instalar el cluster de Kubernetes, el proceso es relativamente sencillo con una herramienta que se llama \texttt{kubeadm} que se encarga de levantar, gestionar y bajar nodos del cluster. Primero para instalar el mismo:
\begin{enumerate}
	\item Debemos tener suporte en el gestor de paquetes APT para el protocolo HTTPS:
    \begin{lstlisting}
apt update && apt install -y apt-transport-https
    \end{lstlisting}	
	\item Agregamos el repositorio de Kubernetes para Debian y Ubuntu:
    \begin{lstlisting}
curl -s \
	https://packages.cloud.google.com/apt/doc/apt-key.gpg \
	| apt-key add -
cat <<EOF >/etc/apt/sources.list.d/kubernetes.list
deb http://apt.kubernetes.io/ kubernetes-xenial main
EOF
    \end{lstlisting}
    \item Actualizamos los índices de APT y procedemos a instalar:
    \begin{lstlisting}
apt-get update
apt-get install -y kubelet kubeadm kubectl
    \end{lstlisting}
\end{enumerate}

% TODO: Cite https://kubernetes.io/docs/setup/independent/create-cluster-kubeadm/
El levantamiento de un nodo maestro básica se puede hacer con el comando:
    \begin{lstlisting}
kubeadm init
    \end{lstlisting}
Que utiliza todos los valores por defecto, pero Kubernetes requiere para su funcionamiento algún driver de red de los cuales se ha elegido Flannel ya que es el más sencillo y no necesitamos mayor funcionalidad como Switches programables, ni enrutamiento o ACLs en las redes de nuestro cluster debido a que no se busca levantar sistemas de producción aquí, solo contenedores independientes y obviamente otros drivers de red con mayor funcionalidad tienen un costo de rendimiento mayor. Si es que se levanto el cluster con el comando anterior, se lo puede destruir con el siguiente comando: 
    \begin{lstlisting}
kubeadm reset
    \end{lstlisting}
Ya que Flannel requiere que se define el rango de IPs con que se trabajara el cluster como 10.244.0.0/16\footnote{Si, parece que tiene que ser exactamente este rango y no suporta mas que $65,536$ contenedores al mismo tiempo, pero eso debe ser suficiente para el propósito actual.}, por lo tanto en realidad, para trabajar con Flannel es necesario agregar un argumento al comando de levantamiento del cluster como se lo indica a continuación:
    \begin{lstlisting}
kubeadm init --pod-network-cidr=10.244.0.0/16
    \end{lstlisting}
Esto se genera una salida como la que se demuestra a continuación:
    \begin{lstlisting}
To start using your cluster, you need to run the following(...)
as a regular user:

  mkdir -p $HOME/.kube
  sudo cp -i /etc/kubernetes/admin.conf $HOME/.kube/config
  sudo chown $(id -u):$(id -g) $HOME/.kube/config

You should now deploy a pod network to the cluster.
Run "kubectl apply -f [podnetwork].yaml" with one of the(...)
options listed at:
  https://kubernetes.io/docs/concepts/cluster-(...)
  administration/addons/

You can now join any number of machines by running the (...)
following on each node
as root:

  kubeadm join --token 655cb5.2275aa7df206fe69 \
  10.10.10.12:6443 --discovery-token-ca-cert-hash \
  sha256:4919df120063c4535fd03e909ce11dfe9e6448f8a7\
  67be914e86b16660d267c8
    \end{lstlisting}

Por defecto Kubernetes no permite que el nodo maestro aloje contenedores como una política de seguridad, pero en este caso se quiere levantar un cluster de un solo nodo y por lo tanto se requiere cambiar esta política con el siguiente comando:
    \begin{lstlisting}
KUBECONFIG=/etc/kubernetes/admin.conf kubectl taint nodes \
--all node-role.kubernetes.io/master-
    \end{lstlisting}
El Variable de Entorno de KUBECONFIG da el archivo que se ve a continuación que autentica el cliente con el cluster. Una vez que se configura bien el cliente, este mismo deja de ser necesario. Entonces a continuación para configurar el cliente:
    \begin{lstlisting}
mkdir -p $HOME/.kube
cp -i /etc/kubernetes/admin.conf $HOME/.kube/config
chown $(id -u):$(id -g) $HOME/.kube/config
    \end{lstlisting}

Se puede probar la conexión pidiendo la versión del cliente y del servidor:
\begin{lstlisting}
kubectl version
\end{lstlisting}

% TODO: Cite https://kubernetes.io/docs/concepts/cluster-administration/networking/
% TODO: Cite https://kubernetes.io/docs/setup/independent/create-cluster-kubeadm/
Para instalar el driver de red Flannel (que es necesaria cualquier driver de red para Kubernetes previo a funcionamiento -- no se instala con ningún por defecto para que sea al elección del administrador quien instala ya que un cluster de Kubernetes solo puede ocupar un driver de red a la vez):
\begin{lstlisting}
sysctl net.bridge.bridge-nf-call-iptables=1
cat sysctl.conf
cat >> /etc/sysctl.conf << EOF

# For Kubernetes Flannel
net.bridge.bridge-nf-call-iptables = 1

EOF
kubectl apply -f \
https://raw.githubusercontent.com/coreos/flannel/v0.9.1/\
Documentation/kube-flannel.yml
\end{lstlisting}

A continuación se puede validar que Flannel o cualquier driver de red para Kubernetes esta funcionando con el comando:
\begin{lstlisting}
kubectl get pods --all-namespaces
\end{lstlisting}
Y que se revisa en la salida del mismo si se logra levantarse \texttt{kube-dns} con una linea similar a:
\begin{lstlisting}
kube-system   kube-dns-...-...  3/3   Running  ... ...
\end{lstlisting}

Para utilizar volúmenes de Git dentro del cluster, es necesario instalar Git en cada nodo:
\begin{lstlisting}
apt install git
\end{lstlisting}

Para consumir el cluster desde el servicio de EduNube, es necesario copiar el archivo /etc/kubernetes/admin.conf al otro equipo. Para realizar la copia, lo que mas conviene, especialmente en redes no tan confiables (ya que el admin.conf contiene los credenciales de superusuario de Kubernetes), es realizar la copia sobre SSH. Primero para validar la existencia de sshd:
\begin{lstlisting}
apt install net-tools
netstat -tupln
\end{lstlisting}
Se puede buscar un servicio llamado sshd que normalmente escucha en el puerto 22 (pero puede ser configurado en otro puerto para ofrecer un poco mas de seguridad\footnote{Normalmente un cambio de puerto no ofrece mayor seguridad y es mal consejo, pero en la experiencia del autor, la mayoría de ataques de fuerza bruta de SSH no utilizan ningún escaneo de puertos ya que son script kiddies o bots bien básicos}). Si es que no existe, hay que instalar el servicio:
\begin{lstlisting}
apt install openssh-server
\end{lstlisting}
Pero en la instalación que realiza xen-tools se instala sshd por defecto. Dentro del servidor de Kubernetes no se tiene acceso al admin.conf cualquier usuario, entonces se requiere conectar por SSH como root, algo que normalmente es muy peligroso y se deshabilita por defecto. Para permitirlo, hay que editar \texttt{/etc/ssh/sshd\_config} y reiniciar el servicio:
\begin{lstlisting}
vim.tiny /etc/ssh/sshd_config

# Editar la linea que diga PermitRootLogin para que diga:
PermitRootLogin yes

systemctl restart sshd
\end{lstlisting}

Desde el equipo que tiene EduNube como servicio se realiza la copia por SSH:
\begin{lstlisting}
scp root@10.10.10.12:/etc/kubernetes/admin.conf .
\end{lstlisting}
Se puede probar que funcionan los credenciales copiados con una petición a Kubernetes de una lista de las maquinas que componen el cluster:
\begin{lstlisting}
kubectl --kubeconfig ./admin.conf get nodes
\end{lstlisting}
Si es que funcionan los credenciales y el sistema que aloja el nodo maestro de Kubernetes esta expuesto a alguna red donde no hay suficiente confianza para dejar que se conecta root por SSH, ya se puede deshabilitar la configuración que se realizo anteriormente y reiniciar el servicio.

Para instalar los credenciales de Kubernetes para que EduNube tiene acceso (pero sin perderse los credenciales al MiniKube instalado anteriormente) se realiza los siguientes comandos:
\begin{lstlisting}
mv admin.conf ~/.kube/config.xen
cp ~/.kube/config ~/.kube/config.minikube
cp ~/.kube/config.xen ~/.kube/config
\end{lstlisting}

Para validar que los credenciales se instalaron correctamente:
\begin{lstlisting}
kubectl version
\end{lstlisting}

\index{Kubernetes}
\subparagraph{Validación de Cluster de Kubernetes}
Para validar el cluster de Kubernetes se puede entrar a la carpeta kubernetes dentro del repositorio de GitEDU/EduNube y realizar las siguientes validaciones del nuevo cluster:
\begin{lstlisting}
cd kubernetes/
kubectl create -f debian-pod.yaml
kubectl get pods/utility
kubectl describe pods/utility
kubectl create -f debian-pod-2.yaml
for manifest in `ls *.json`; do
    kubectl create -f $manifest;
done
kubectl get jobs
# no todos tendran exito al ejecutarse, algunos
# trabajos requieren configuraciones especiales
# que solo se encontraron en el entorno de
# experimentos, y algunos de los experimentos
# no fueron exitosos
watch -n 15 "kubectl get jobs"
# en este caso el primero en terminar:
kubectl describe jobs/pi
# El pod que fue creado (sera diferente):
# pi-kf8zv
kubectl describe pods/pi-kf8zv
# Ver la salida del trabajo
kubectl logs jobs/pi
\end{lstlisting}

\paragraph{Interfaz de Administracion}
Para instalar la interfaz web de administracion para Kuberntes, se sigue los siguientes pasos:
\begin{enumerate}
	\item Para evitar problemas, se connecta mediante SSH al nodo maestro del cluster y se instala Dashboard:
		\begin{lstlisting}
ssh root@10.10.10.12
kubectl apply -f https://raw.githubusercontent.com/kubernetes/dashboard/master/src/deploy/recommended/kubernetes-dashboard.yaml
		\end{lstlisting}
	\item Desde el mismo servidor o otro que tiene conexion al API del nodo maestro (desde donde se quiere tener acceso en su navegador), se ejecuta:
    	\begin{lstlisting}
kubectl proxy
cat > dashboard-admin.yml << EOF       
apiVersion: rbac.authorization.k8s.io/v1beta1
kind: ClusterRoleBinding
metadata:
  name: kubernetes-dashboard
  labels:
    k8s-app: kubernetes-dashboard
roleRef:
  apiGroup: rbac.authorization.k8s.io
  kind: ClusterRole
  name: cluster-admin
subjects:
- kind: ServiceAccount
  name: kubernetes-dashboard
  namespace: kube-system
EOF
kubectl create -f dashboard-admin.yml
    	\end{lstlisting}
\end{enumerate}
Para ver el Dashboard se visita en su navegador: \href{http://localhost:8001/api/v1/namespaces/kube-system/services/https:kubernetes-dashboard:/proxy/} y en lugar de autenticarse, se pone "skip".
\index{Contenedor} \index{Virtualización} \index{Hipervisor}

\section{Despliegue}
Para el despliegue de los servicios y su infrastructura de apoyo se propone utilizar NGinX como proxy inversa, con uWSGI como servidor de aplicacion para los servicios de Django/Python y todo gestionado por el sistema operativo de despliegue con Systemd que ahora es un estandar por defecto en casi todos los distribuciones de Linux. La segmentacion de servicios en el mismo puerto (80/HTTP) de NGinX se realiza con subdominios como se indica al inicio de este capitulo.

La configuracion de \texttt{http://10.10.10.1} y \texttt{http://git.localhost} se encuentra en el capitulo \ref{capitulo4} de Desarrollo.

La configuracion de los tres servicios de Django, GitEDU, EduNube y GitServerHTTPEndpoint se detallan a continuacion por seperado por el hecho de que la configuracion de cada uno es mas extenso.

% TODO: Configuracion de Gitlab (Docker) con NGinX

% TODO: Configuracion de Moodle (Docker) con NGinX

% TODO: Configuracion de Docker Registry (Docker) con NGinX

% TODO: discuss NGinX config for deployment, introduce idea of SystemD Services
% servicios y su asignacion de puertos
% GitEDU - 8000/HTTP = gitedu.localhost
% EduNube - 8010/HTTP = edunube.localhost
% GitServerHTTPEndpoint - 8020/HTTP = gitsrvendpoint.localhost
% Kubernetes Proxy - 8001/HTTP = kubernetes.proxy.localhost
% aux services (redir NGinX):
%		http://git.localhost:80 -> GitWeb
%		http://gitedu.localhost:80 -> GitEDU
%		http://edunube.localhost:80 -> EduNube
%		http://gitsrvendpoint.localhost:80 -> GitServerHTTPEndpoint
%		http://gitlab.localhost:80 -> GitLab
%		http://moodle.localhost:80 -> Moodle
%		http://registry.localhost:80 -> Registry
%\begin{description}
%	\item[Maquina Física] con Debian 9.x para Servicios y Docker: \textit{10.10.10.1}
%    \begin{description}
%    	\item[NGinX] \textit{http://0.0.0.0:80} \index{NGinX}
%        \begin{itemize}
%            \item \textit{http://10.10.10.1:80} \& \textit{http://git.localhost:80} -> GitWeb (HTTP)
%            \item \textit{http://gitedu.localhost:80} -> GitEDU (HTTP)
%            \item \textit{http://edunube.localhost:80} -> EduNube (HTTP)
%            \item \textit{http://gitsrvendpoint.localhost:80} -> GitServerHTTPEndpoint (HTTP)
%            \item \textit{http://gitlab.localhost:80} -> GitLab (HTTP)
%            \item \textit{http://moodle.localhost:80} -> Moodle (HTTP)
%            \item \textit{http://registry.localhost:80} -> Registry (HTTP)
%        \end{itemize}
%        \item[GitEDU] \textit{http://0.0.0.0:8000}
%        \item[EduNube] \textit{http://0.0.0.0:8010}\footnote{Previamente el puerto asignado fue :8001, pero esto estaba en conflicto con el Dashboard/Proxy de Kubernetes, por lo tanto se ha cambiado este servicio al puerto :8010}
%        \item[GitServerHTTPEndpoint] \textit{http://0.0.0.0:8020}\footnote{Para temas de consistencia con el cambio de puerto de EduNube, se ha cambiado el numero de puerto de :8002 a :8020}
%        \item[Kubernetes Proxy] \textit{http://0.0.0.0:8001}
%        \item[Docker] \textit{10.10.10.1}
%        \begin{description}
%        	\item[MySQL] \textit{0.0.0.0:3306} -> \textit{moodledb:3306} (TCP/MySQL)
%        	\item[Moodle] \textit{0.0.0.0:8201} -> \textit{moodle:80} (HTTP)
%        	\item[GitLab CE] \textit{10.10.10.1}
%            \begin{itemize}
%            	\item \textit{0.0.0.0:8143} -> \textit{gitlab:443} (HTTPS)
%            	\item \textit{0.0.0.0:8101} -> \textit{gitlab:80} (HTTP)
%            	\item \textit{0.0.0.0:8122} -> \textit{gitlab:22} (SSH)
%            \end{itemize}
%            \item[Registry] \textit{0.0.0.0:5000} -> \textit{registry:5000} (HTTP)
%        \end{description}
%    \end{description}
%    \item[Maquina Virtual (Xen)] con Debian 9.x para el Cluster (solo 1 nodo maestro) de Kubernetes: \textit{10.10.10.12}
%\end{description}

Para instalar uWSGI, se lo instala como root a nivel de sistema operativo para Python 3:
\begin{lstlisting}
pip3 install uwsgi
\end{lstlisting}

El usuario y grupo que se utiliza para los servidores de aplicaciones (menos la de GitServerHTTPEndpoint que utiliza el usuario Git) es uWSGI, por lo tanto hay la necesidad de crear dicho usuario:
\begin{lstlisting}
adduser uwsgi
\end{lstlisting}

% TODO: Instead of nyx, maybe uWSGI would be better as user/group, ie chown root:uwsgi
El servicio de SystemD para gestionar cada servidor de aplicacion (uWSGI) mediante un archivo de configuracion \texttt{.ini} se define de la siguiente manera: 
\begin{lstlisting}
[Unit]
Description=uWSGI Emperor service

[Service]
#ExecStartPre=/bin/bash -c 'mkdir -p /run/uwsgi; chown uwsgi:uwsgi /run/uwsgi'
ExecStart=/usr/local/bin/uwsgi --emperor /etc/uwsgi/sites
Restart=always
KillSignal=SIGQUIT
Type=notify
NotifyAccess=all
User=uwsgi
Group=uwsgi

[Install]
WantedBy=multi-user.target
\end{lstlisting}

El servicio de uWSGI se empieza con el siguiente manera:
\begin{lstlisting}
systemctl start uwsgi
\end{lstlisting}

Se puede revisar el estado del mismo con:
\begin{lstlisting}
systemctl status uwsgi
\end{lstlisting}

Cuando se ha confirmado su correcto funcionamiento, se puede activar el servicio para iniciar con el sistema operativo:
\begin{lstlisting}
systemctl enable uwsgi
\end{lstlisting}

\subsection{GitEDU}
% TODO: GitEDU uWSGI ini

\subsection{EduNube}
% TODO: EduNube uWSGI ini

\subsection{GitServerHTTPEndpoint}
Para que el usuario git puede levantar un servidor uwsgi para su servicio GitServerHTTPEndpoint, se le agrege al grupo uwsgi:
\begin{lstlisting}
adduser git uwsgi
\end{lstlisting}

% TODO: add git user to uwsgi group, gid = uwsgi
Para levantar el servicio de GitServerHTTPEndpoint se guarda el archivo \\ \texttt{/etc/uwsgi/sites/gitserverhttpendpoint.ini} con los siguientes contenidos:
\begin{lstlisting}
[uwsgi]
project = GitServerHTTPEndpoint
project_location = %(project)
username = git
base = /home/%(username)
environment = %(project)/env

chdir = %(base)/%(project_location)
home = %(base)/%(environment)
module = %(project).wsgi:application
logto = /usr/share/uwsgi.%(project).log

master = true
processes = 2

uid = %(username)
gid = uwsgi
http-socket = 127.0.0.1:8021
\end{lstlisting}

Se crea el archivo log con los permisos adecuados:
\begin{lstlisting}
touch /usr/share/uwsgi.gitserverhttpendpoint.log
chown git:uwsgi /usr/share/uwsgi.gitserverhttpendpoint.log
\end{lstlisting}

Se reinicia el servicio de uWSGI para que se levanta con el nuevo archivo de configuracion que representa el respectivo servicio:
\begin{lstlisting}
systemctl restart uwsgi
\end{lstlisting}
Cualquier error que se da, se lo puede investigar mediante el archivo de log \\ \texttt{/usr/share/uwsgi.gitserverhttpendpoint.log}.

% TODO: root:uwsgi as group?
Se crea una ubicacion para archivos estaticos y se le da los permisos adecuados:
\begin{lstlisting}
mkdir -p /static/service/uwsgi/githttpserverendpoint/static
find /static -type d -exec chmod +x -c {} \;
chown -cR git:root /static/service/uwsgi/githttpserverendpoint
find /static/service/uwsgi/githttpserverendpoint -type f \
    -exec chmod -c 660 {} \;
\end{lstlisting}

Se cambia el settings.py del servicio:
\lstset{language=Python}
\begin{lstlisting}
STATIC_ROOT = "/static/service/uwsgi/githttpserverendpoint/static"
\end{lstlisting}
\lstset{language=Bash}

Se copia los archivos al directorio que utiliza NGinX para servir los archivos estaticos, este paso se tiene que repetir siempre y cuando los archivos estaticos cambien:
\begin{lstlisting}
python manage.py collectstatic
\end{lstlisting}

Para la configuracion de NGinX se escribe las siguientes lineas en el archivo \\ \texttt{/etc/nginx/sites-available/gitserverhttpendpoint}:
\begin{lstlisting}
server {
        listen 8020;
        listen [::]:8020;
        server_name _;
        location /static/ {
                root /static/service/uwsgi/githttpserverendpoint;
        }
        location / {
                proxy_pass http://127.0.0.1:8021;
        }
}
\end{lstlisting}

Se agrega la nueva configuracion de NGinX, valida y renicia el servicio de NGinX:
\begin{lstlisting}
ln -s -r /etc/nginx/sites-available/gitserverhttpendpoint \ 
    /etc/nginx/sites-enabled/33-gitserverhttpendpoint
nginx -t
systemctl restart nginx
\end{lstlisting}

Se puede validar que el servicio esta funcionando en visitar: \url{http://10.10.10.1:8020/auth/login}

\section{Pruebas del Despliegue y Resultados}
% TODO: Testing and test results of deployment

% Chapter 7

\chapter{Conclusiones}
\label{capitulo7}

\section{Resultados}
% TODO: Resultados %% REVIEW
El principal resultado obtenido es el realizar un prototipo de la plataforma propuesta donde se integra tres servicios cuyas funcionalidades esenciales son:
\begin{description}
	\item[GitEDU,] ofrece una autenticación clásica y autenticación de LTI, manejo de namespace y repositorios. Adicionalmente cuenta con un editor de código en línea para gestionar archivos dentro de un repositorio, esto, llevado bajo un sistema de control de versiones interno.
    \item[EduNube,] mediante su API, ofrece toda la funcionalidad necesaria para que cada uno de los usuarios de GitEDU pueda ejecutar el código que desarrolló dentro de la plataforma. El servicio está planteado para extenderse con nuevas funcionalidades a futuro, por ejemplo el dar soporte a la ejecución de más lenguajes de programación (mediante contenedores de Docker), y la calificacion del código escrito por los usuarios.
    \item[GitServerHTTPEndpoint,] mediante su API, permite que los cambios realizados en el editor de GitEDU sean persistentes en repositorios de Git externos para su consumo por parte de aplicaciones y usuarios externos.
\end{description}

% TODO: get input on conclusions (are they okay for the thesis) %% REVIEW
% Minimo 1 Conclusion frente cada Objetivo Especifico
% Minimo 1 Conclusion frente cada Linea Problematica
% Lo que no va en resultados, puede ir a conclusiones
% Conclusiones tienen que ser declaraciones positivas o negativas, no van preguntas
% Cuantas Conclusiones poner? 3-5 es acceptable -> 7 conclusiones suele ser demaciado
% Tipicamente un recomendacion por cada cada conclusion

\section{Conclusiones}
A lo largo del desarrollo de este trabajo de titulación, se ha llegado a las siguientes conclusiones:

\begin{itemize}
  \item Para el desarrollo de sistemas educativos, LTI puede ser una buena solución a tomar en cuenta, para facilitar el flujo de autenticación entre diferentes sistemas de aprendizaje y enseñanza.
  \item Para datos que no tienen estructuras muy fijas o que tienen campos de longitudes altamente variables, como es el caso del codigo, bases de datos no-relacionales como MongoDB, ofrecen una buena alternativa frente a las bases de datos relacionales y simplifica el trabajo a ser realizado.
  \item Kubernetes y Docker son herramientas muy potentes para generar entornos confiables, entregar sistemas de producción valiosas y garantizar escalabilidad y portabilidad de código realizado debido a que su curva de aprendizaje es minima para el nivel de potencia que los mismos ofrecen.
  \item Cualquier servidor de Git es altamente pesada, especialmente GitLab que en su estado pasivo consume muchos recursos, y tambien GitWeb que tiene inconvenientes cada vez que recibe nuevos objetos de Git sobre HTTP.
  % TODO: Revisar Conclusion General %% REVIEW
  % Pame, la conclusion general no debe ser el ultimo?
  \item La combinacion de Python 3 con librerias selecionadas para realizar los respectivos backends, es decir Django, PyModm, Bcrypt, PyJWT, Requests y ipython forman una combinación potente para el desarollo rápido de servicios de producción siempre y cuando se maneje de forma adecuada el diseño de los mismos y el manejo de distintos tipos de datos.
  \item Muchos usuarios solo se fijan en que tan llamativo es un interfaz de usuario y no en la funcionalidad que esta por detras. Por lo tanto es importante encontrar las herramientas adecuadas para que el editor de código en la plataforma sea tanto llamativa como funcional, porque la primera impresión es lo que más cuenta.
\end{itemize}


% Previously Chapter 8: Recomendaciones, united with Chapter 7: Conclusiones by recomendation of Ing. Maria del Carmen
%\chapter{Recommendaciones}
\section{Recomendaciones}
%\label{capitulo8}
% TODO: revisar si recomendations son adecuadas %% REVIEW
%Tipicamente un recomendacion por cada cada conclusion

En base a lo que se ha aprendido en el camino de este trabajo de titulación, se puede realizar las siguientes recomendaciones:

\begin{itemize}
  \item Cuando se desarrolla sistemas con un enfoque educativo, se recomienda reutilizar una implementacion de LTI para evitar problemas de integración con el software nuevo que se desarrolla.
  \item En el curso de realizacion de interfaces web, se recomiendo reutilizar el trabajo de terceros siempre y cuando hay la posibilidad, con el fin de poder dar mayor enfoque a la funcionalidad del backend.
  \item Para trabajar con cualquier base de datos, sea relacional o no relacional, lo más recomendable es trabajar con un ORM que permite abstraer la base de datos y facilitar el desarrollo y persistencia de datos en la misma.
  \item Siempre se recomienda utilizar nuevas tecnologias, los cuales pueden ser utiles para resolver problemas actuales ya que ofrecen nuevas perspectivas y soluciones de los cuales no se los ha podido considerar antes.
  \item Donde hay posibilidad, se recomienda no trabajar con GitLab, a menos de que dispone de los recursos necesarios para ello y tambien que requiere las características avanzadas del mismo. En entornos donde sea posible, lo más recomendable es trabajar con Git sobre SSH, que además de ser más seguro, tiene mayor soporte por parte de Git y mayor inteligencia para la sincronización cuando se trabaja con este protocolo que puede dar como resultado mayor eficiencia en uso de red.
\end{itemize}

% TODO: escribir recomendacion general?

\section{Trabajos Futuros}
Se considera los siguientes aspectos que se podrian y/o se deben trabajar a futuro, los cuales no esta en ningun orden especifico:
\begin{itemize}
	\item Arreglar Errores en la funcionalidad existente, por ejemplo:
    \begin{itemize}
    	\item EduNube no actualiza de forma adecuada repositorios de ejecuccion, lo cual resulta muchas veces en la ejecuccion de una version antigua del mismo.
        \item EduNube no genera IDs de ejecuccion de forma adecuada.
    \end{itemize}
    \item Autenticacion en GitEDU y/o otros servicios mediante LDAP.
    \item Sincronizacion en tiempo real de codigo editado en GitEDU y sus respectivo backends de codigo, tal ves mediante websockets.
    \item Socializacion de vistas para editar codigo (GitEDU) con la finalidad de promover interaccion y collaboracion entre usuarios sobre los mismos.
    \item Un sistema de permisos para el editor de codigo (GitEDU).
    \item Distintas formas de calificacion, incluyendo:
    \begin{itemize}
    	\item Calificacion Manual por parte de professores.
        \item Calificacion Automatizado por parte del sistema (tal vez en forma de un servicio nuevo?) en base a pruebas unitarias definidos por professores.
        \item Calificacion Hibrida que combina los anteriores.
    \end{itemize}
    \item Sincronizacion de Notas (generados por la(s) modalidad(es) de calificacion anteriores) por LTI con los sistemas adecuadas para el manejo de los mismos, por ejemplo los LMS.
    \item Mas backends de Git para el servicio GitServerHTTPEndpoint, como por ejemplo GitLab o Djacket.
    \item Mas backends de persistencia de codigo para el servicio GitEDU como Redis o GitLab.
    \item Mas backends de virtualizacion para EduNube como OpenStack, Docker, etc.
    \item Versionamiento de APIs en todos los servicios.
    \item Auditoria de Seguridad del Sistema Desarrollado, especialmente en el caso de los APIs que no manejan estados y solo se protegen con API tokens JWT y a lo mucho TLS.
    \item Mejor gestion de la configuracion, actualmente hay componentes de algunos servicios que dejen de funcionar o que funcionan de forma inadecuada cuando no disponen de sus servicios dependientes. Debe ser configurable cuales servicios existen o no, dinamico la manera en que se encuentran y cada servicio tolerante a fallos en los demas servicios.
    \item Convertir los servicios en imagenes de Docker para montarlos mismos en un cluster de Kubernetes y realizar un estudio de alta disponabilidad/escalabilidad.
    \item Implementar el sistema de plantillas de GitEDU.
    \item Migrar servicios a Django 2.x (la proxima version de soporte a largo plazo de Django esta planificado como el 2.3) ya que en realizar esta migracion de version, actualmente se rompe la forma en que se llevan los URIs con un namespace para cada app.
    \item Automatizar y aumentar las pruebas unitarias de la aplicacion de acuerdo con el plan de pruebas original.
    \item Uso bidireccional del Git (por el momento es unidireccional, GitEDU solo guarda en los repositorios de GitWeb, nunca recupera codigo de usuarios guardado alli, que obviamente a futuro podria dificulta la interaccion entre el usuario y el sistema, en obligarle a siempre editar proyectos dentro del plataforma).
    \item Mayor soporte para caracteristicas de Git como ramas, tags, etc.
    \item Llevar metadatos de lenguaje de programacion / ejecutor seleccionado para repositorios en GitEDU, para su uso al momento de llamar al API de EduNube, se puede realizar la llamada adecuada (actualmente esta quemada el uso del ejecutor de Python 3).
    \item Recolleccion de datos para apoyar la toma de decisiones estrategicas.
\end{itemize}


%----------------------------------------------------------------------------------------
%	THESIS CONTENT - APPENDICES
%----------------------------------------------------------------------------------------

\appendix % Cue to tell LaTeX that the following "chapters" are Appendices

% Include the appendices of the thesis as separate files from the Appendices folder
% Uncomment the lines as you write the Appendices


% Appendix A - Vision Document

\chapter{Documento de Vision}
\label{visionDoc}

% TODO

\section{Proposito}
Ayudar a mejorar los métodos de enseñanza que ofrece la Universidad Técnica Particular de Loja en cuanto a la programación y uso de base de datos para las carreras de Sistemas y Electrónica.
\paragraph{Alcance}
Se propone un sistema de editar código en línea que a su vez integra LMS externos (para autenticación y notas), un servidor de control de versiones externo (para la persistencia de código), y un servicio web de ejecución de código en línea de una forma segura, eficaz y eficiente (para dar un ambiente de ejecución y pruebas tanto para los usuarios del sistema como para calificar de una forma automática).
\section{Definiciones, Acrónimos y Abreviaciones}
\begin{description}
	\item[LMS] Learning Management System (Sistema de Gestión de Aprendizaje)
    \item[LTI] Learning Tools Interoperability (estandar de Interoperabilidad entre Herramientas de Aprendizaje)
    \item[GitEDU] sistema de Git EDUcation
\end{description}
\subsection{Posición y Oportunidad de Negocios}
Con el avance continuo de la tecnología y su introducción en mas aspectos de la vida diaria de cada uno, hay una necesidad creciente de ingenieros en sistemas e electrónica que pueden programar e entender el software que hace todo funcionar en adición a los bases de datos que están por detrás de estos mismos sistemas.
 
Es por aquella razón que ahora está de moda ofrecer plataformas en línea para la enseñanza dinámica de la programación. GitEDU espere ofrecer las mismas funcionalidades a un costo institucional menor a travez de integración con sistemas existentes e innovación para proveer una mejor experiencia de usuario, tanto estudiantes como profesores para llevarse a cabo un mejor proceso de aprendizaje.

%\pagebreak
%\paragraph{Definición de Problema}
\begin{table}[h!]
  \begin{tabular}{|p{0.2\textwidth}|p{0.7\textwidth}|}
    \hline
    El problema de & enseñar y evaluar programación \\
    \hline
    afecta a & los estudiantes y docentes de las carreras de Sistemas y Electrónica \\
    \hline
    el impacto de lo cual es & el uso ineficiente de recursos universitarios en la enseñanza de la programación \\
    \hline
    Una solución exitosa seria & una aplicación web que ofrece un editor de código en línea, la capacidad de ejecutar este código para proveer mejor interacción con los estudiantes, la capacidad de ejecutar pruebas unitarias para automatizar el proceso de calificaciones, la persistencia de código en un repositorio de control de versiones remoto para su fácil revisión después por parte de profesores y estudiantes, y la integración transparente con sistemas de gestión de aprendizaje externos para la autenticación de usuarios y la respectivo registro de notas. \\
    \hline
  \end{tabular}
  \caption{Definición del Problema.}
  \label{def-prob}
\end{table}

%\pagebreak
%\paragraph{Posición de Producto}
\begin{table}[h!]
  \begin{tabular}{|p{0.2\textwidth}|p{0.7\textwidth}|}
    \hline
    Para & docentes y estudiantes de la Universidad Técnica Particular de Loja \\
    \hline
    Quienes & tienen dificultades en la enseñanza y aprendizaje con la programación \\
    \hline
    GitEDU & es una plataforma web \\
    \hline
    Que & provee un espacio para la interacción entre profesores y alumnos para la enseñanza y aprendizaje de la programación y uso de las bases de datos \\
    \hline
    A diferencia de & otras plataformas altamente costosas que no se integran completamente con sistemas existentes de la universidad ni permiten alta interacción entre docentes y alumnos \\
    \hline
    Nuestro producto & da mayor capacidad para interacción entre estudiantes y sus docentes y se integra bien con las sistemas existentes para dar una mejor experiencia a todos los involucrados a un costo menor. \\
    \hline
  \end{tabular}
  \caption{Posición de Producto.}
  \label{pos-prod}
\end{table}

\pagebreak

\section{Usuarios e Interesados}
\subsection{Demográfica del Mercado}
En el año 2011, la Universidad Técnica Particular de Loja contaba con aproximadamente 4000 estudiantes presenciales y 24000 estudiantes a distancia con una tendencia creciente [UTPL-Datos-Estadisticos]. En la experiencia personal del autor, las carreras de Sistemas y Electrónica, por lo menos en la modalidad presencial, juntos representan aproximadamente un 10\% de todos los estudiantes en la universidad lo cual daría un mercado de estudiantes afectados por un nuevo sistema de aproximadamente un mínimo 2800 estudiantes. Según el directorio de docentes de la universidad, son 60 profesores en el departamento de Ciencias de la Computación y Electrónica [UTPL-Directorio-Docentes]. Con eso se puede estimar un mínimo de 2860 usuarios lo los cuales el sistema propuesto podría llegar a afectar.

Es precisamente la parte de la población, de usuario potenciales mencionado anteriormente, que está en el proceso de enseñar, evaluar y aprender habilidades de programación y consultar bases de datos que forman la base de usuarios de la aplicación.

%\pagebreak
%\paragraph{Resumen de Interesados}
%\begin{table}[h!]
  %\begin{tabular}{|p{0.15\textwidth}|p{0.35\textwidth}|p{0.4\textwidth}|}
% TODO: Caption before longtable, not after
\begin{longtable}{|p{0.2\textwidth}|p{0.35\textwidth}|p{0.35\textwidth}|}
  \hline
  \textbf{Nombre} & \textbf{Descripción} & \textbf{Responsabilidades} \\
  \hline
  \endhead
  Analista & Trabaja con el Asesor Principal y Auxiliar para entender bien las necesidades institucionales para poder llevar un buen ingeniería de requerimientos y diseño del sistema propuesto. & Definir bien el problema para analizarlo, generar requerimientos en base a las necesidades para diseñar y documentar componentes del sistema final para el beneficio del Arquitecto de Software, Programador, Gestor de Proyecto y futuro mano de obra en el proyecto. \\
  \hline
  Arquitecto de Software & Trabaja con el Asesor Principal y Auxiliar para definir una arquitectura que garantiza que se cumple con los atributos de calidad que se requiere el sistema y será compatible con infraestructura y sistemas institucionales ya existentes. & Diseñar los modelos de interacción entre todos los componentes internos y externos del sistema para con ello lograr un flujo eficaz y eficiente, que también cumple con los parámetros de los requerimientos no funcionales, en el sistema final. \\
  \hline
  Gestor de Proyecto & Trabaja con el Asesor Principal y Auxiliar para evaluar, estimar y establecer el alcance, los recursos y el cronograma del proyecto. & Distribuir de manera eficiente y eficaz los recursos para ayudar el analista, arquitecto de software, programador y administrador de sistemas y bases de datos cumplir dentro de los recursos, alcance y cronograma preestablecido. \\
  \hline
  Programador & Trabaja con el Asesor Principal, Asesor Auxiliar, y Analista para implementar soluciones técnicas que cumplen con el diseño dado por el analista. & Desarrollar el sistema en todos sus componentes. \\
  \hline
  Administrador de Sistemas y Bases de Datos & Trabaja con el Asesor Principal y Auxiliar para desplegar la aplicación en la institución. & El despliegue correcto del sistema con todos sus componentes en la institución respectivo. \\
  \hline
  Asesor Principal & Trabaja con el Asesor Auxiliar para poder aconsejar de la mejor manera el Analista, Arquitecto de Software, Gestor de Proyecto, Programador y Administrador de Sistemas y Bases de Datos. & La definición de necesidades y aprobación del producto final. \\
  \hline
  Asesor Auxiliar & Trabaja con el Asesor Principal para poder aconsejar de la mejor manera el Analista, Arquitecto de Software, Gestor de Proyecto, Programador y Administrador de Sistemas y Bases de Datos. & La definición de necesidades y aprobación del producto final. \\
  \hline
  Asesor de Documentación & Trabaja con el Analista y Gestor de Proyecto para asegurar la calidad de la documentación que se genera a lo largo del proyecto y que se cumple con el cronograma establecido entre el gestor del proyecto y los asesores principales y auxiliares. & La aprobación de los avances en la documentación y la documentación completa al final del proyecto. \\
  \hline
  %\end{tabular}
  \caption{Resumen de Interesados.}
  \label{res-inter}
\end{longtable}
%\end{table}

%\pagebreak
%\paragraph{Posición de Producto}
\begin{table}[h!]
  \begin{tabular}{|p{0.225\textwidth}|p{0.225\textwidth}|p{0.225\textwidth}|p{0.225\textwidth}|}
    \hline
    \textbf{Nombre} & \textbf{Descripción} & \textbf{Responsabilidades} & \textbf{Interesado} \\
    \hline
    Estudiantes & Usuario Final Primaria del Sistema & Use la aplicación para cumplir con las tareas, pruebas y exámenes que le pone el docente & Los mismos \\
    \hline
    Professores & Usuario Final Primaria del Sistema & Use la aplicación para dar tareas, pruebas y exámenes a los estudiantes y calificar los mismos & Los mismos \\
    \hline
    Administradores & Quienes administran el sistema en su ambiente de despliegue & Mantener el sistema & El mismo \\
    \hline
  \end{tabular}
  \caption{Resumen de Usuarios.}
  \label{res-user}
\end{table}

\subsection{Ambiente de Usuario}
El sistema será disponible para el uso por usuarios que estén en el campus universitario y en sus casas.

\subsection{Perfiles de Interesados}
Para entender a fondo cada clase de interesado se da a continuacion un analisis de los mismos.

\pagebreak

%\textbf{Analista}
%\pagebreak
\begin{table}[h!]
  \begin{tabular}{|p{0.45\textwidth}|p{0.45\textwidth}|}
    \hline
    \textbf{Descripción} & El analista del equipo de desarrollo \\
    \hline
    \textbf{Tipo} & Miembro del Equipo de Desarrollo \\
    \hline
    \textbf{Responsabilidades} & Definir bien el problema para analizarlo, generar requerimientos en base a las necesidades para diseñar y documentar componentes del sistema final para el beneficio del Arquitecto de Software, Programador, Gestor de Proyecto y futuro mano de obra en el proyecto. \\
    \hline
    \textbf{Criteria de Exito} & Que se lleva a cabo exitosamente el proyecto bajo todos sus requerimientos funcionales y no funcionales \\
    \hline
    \textbf{Involucramiento} & En cada fase del proyecto \\
    \hline
    \textbf{Entregables} & Documentación del Sistema y su funcionamiento \\
    \hline
    \textbf{Comentarios / Preocupaciones} & Que se cumple con todas las necesidades institucionales \\
    \hline
  \end{tabular}
  \caption{Perfil de Interesado: Analista.}
  \label{per-inter-analista}
\end{table}

\vfill

%\textbf{Arquitecto de Software}
%\pagebreak
\begin{table}[h!]
  \begin{tabular}{|p{0.45\textwidth}|p{0.45\textwidth}|}
    \hline
    \textbf{Descripción} & El arquitecto de software en el equipo de desarrollo \\
    \hline
    \textbf{Tipo} & Miembro del Equipo de Desarrollo \\
    \hline
    \textbf{Responsabilidades} & Diseñar los modelos de interacción entre todos los componentes internos y externos del sistema para con ello lograr un flujo eficaz y eficiente, que también cumple con los parámetros de los requerimientos no funcionales, en el sistema final. \\
    \hline
    \textbf{Criteria de Exito} & Que se lleva a cabo exitosamente el proyecto bajo todos sus requerimientos no funcionales \\
    \hline
    \textbf{Involucramiento} & En cada fase del diseño y despliegue \\
    \hline
    \textbf{Entregables} & Modelos Arquitectónicos del Sistema y su interacción con otros sistemas \\
    \hline
    \textbf{Comentarios / Preocupaciones} & Que se cumple con todas las atributos de calidad que la institución manda \\
    \hline
  \end{tabular}
  \caption{Perfil de Interesado: Arquitecto de Software.}
  \label{per-inter-arquitecto}
\end{table}

\pagebreak

%\textbf{Gestor de Proyecto}
%\pagebreak
\begin{table}[h!]
  \begin{tabular}{|p{0.45\textwidth}|p{0.45\textwidth}|}
    \hline
    \textbf{Descripción} & El gestor de proyecto en el equipo de desarrollo \\
    \hline
    \textbf{Tipo} & Miembro del Equipo de Desarrollo \\
    \hline
    \textbf{Responsabilidades} & Distribuir de manera eficiente y eficaz los recursos para ayudar el analista, arquitecto de software, programador y administrador de sistemas y bases de datos cumplir dentro de los recursos, alcance y cronograma preestablecido. \\
    \hline
    \textbf{Criteria de Exito} & Que se lleva a cabo exitosamente el proyecto bajo todos sus requerimientos y dentro de los recursos y cronograma preestablecido. \\
    \hline
    \textbf{Involucramiento} & En cada fase del proyecto \\
    \hline
    \textbf{Entregables} & Documentación de la gestión del proyecto y el adecuado seguimiento y control interno a lo largo del mismo \\
    \hline
    \textbf{Comentarios / Preocupaciones} & Que se cumple con todo el proyecto dentro de los recursos y cronograma preestablecido \\
    \hline
  \end{tabular}
  \caption{Perfil de Interesado: Gestor de Proyecto.}
  \label{per-inter-project-manager}
\end{table}

\vfill

%\textbf{Programador}
%\pagebreak
\begin{table}[h!]
  \begin{tabular}{|p{0.45\textwidth}|p{0.45\textwidth}|}
    \hline
    \textbf{Descripción} & El programador del equipo de desarrollo \\
    \hline
    \textbf{Tipo} & Miembro del Equipo de Desarrollo \\
    \hline
    \textbf{Responsabilidades} & Desarrollar el sistema en todos sus componentes. \\
    \hline
    \textbf{Criteria de Exito} & Que se lleva a cabo exitosamente el proyecto bajo todos sus requerimientos funcionales y no funcionales \\
    \hline
    \textbf{Involucramiento} & En cada fase del desarrollo del proyecto \\
    \hline
    \textbf{Entregables} & Código del Sistema \\
    \hline
    \textbf{Comentarios / Preocupaciones} & Que se logra programar segun la especificacion que el analista le da \\
    \hline
  \end{tabular}
  \caption{Perfil de Interesado: Programador.}
  \label{per-inter-programer}
\end{table}

\pagebreak

%\textbf{Administrador de Sistemas y Bases de Datos}
%\pagebreak
\begin{table}[h!]
  \begin{tabular}{|p{0.45\textwidth}|p{0.45\textwidth}|}
    \hline
    \textbf{Descripción} & El administrador de sistemas y bases de datos del equipo de desarrollo \\
    \hline
    \textbf{Tipo} & Miembro del Equipo de Desarrollo \\
    \hline
    \textbf{Responsabilidades} & El despliegue correcto del sistema con todos sus componentes en la institución respectivo. \\
    \hline
    \textbf{Criteria de Exito} & Que se lleva a cabo exitosamente el proyecto bajo todos sus requerimientos funcionales y no funcionales \\
    \hline
    \textbf{Involucramiento} & En cada fase del desarrollo y despliegue del proyecto \\
    \hline
    \textbf{Entregables} & Documentación de los Servicios Desplegados \\
    \hline
    \textbf{Comentarios / Preocupaciones} & Que se logra desplegar la aplicación según la especificación del arquitecto de software \\
    \hline
  \end{tabular}
  \caption{Perfil de Interesado: Administrador de Sistemas y Bases de Datos.}
  \label{per-inter-admn}
\end{table}

\vfill

%\textbf{Asesor Principal}
%\pagebreak
\begin{table}[h!]
  \begin{tabular}{|p{0.45\textwidth}|p{0.45\textwidth}|}
    \hline
    \textbf{Descripción} & El asesor principal para equipo de desarrollo \\
    \hline
    \textbf{Tipo} & Asesor del Equipo de Desarrollo \\
    \hline
    \textbf{Responsabilidades} & La definición de necesidades y aprobación del producto final. \\
    \hline
    \textbf{Criteria de Exito} & La entrega del producto final \\
    \hline
    \textbf{Involucramiento} & En cada fase del proyecto \\
    \hline
    \textbf{Entregables} & Aprobación del Producto Final \\
    \hline
    \textbf{Comentarios / Preocupaciones} & Que se logra realizar la aplicación. \\
    \hline
  \end{tabular}
  \caption{Perfil de Interesado: Asesor Principal.}
  \label{per-inter-a-prin}
\end{table}

\vfill

%\textbf{Asesor Auxiliar}
%\pagebreak
\begin{table}[h!]
  \begin{tabular}{|p{0.45\textwidth}|p{0.45\textwidth}|}
    \hline
    \textbf{Descripción} & El asesor auxiliar para equipo de desarrollo \\
    \hline
    \textbf{Tipo} & Asesor del Equipo de Desarrollo \\
    \hline
    \textbf{Responsabilidades} & La definición de necesidades y aprobación del producto final. \\
    \hline
    \textbf{Criteria de Exito} & La entrega del producto final \\
    \hline
    \textbf{Involucramiento} & En cada fase del proyecto \\
    \hline
    \textbf{Entregables} & Aprobación del Producto Final \\
    \hline
    \textbf{Comentarios / Preocupaciones} & Que se logra realizar la aplicación. \\
    \hline
  \end{tabular}
  \caption{Perfil de Interesado: Asesor Auxiliar.}
  \label{per-inter-a-aux}
\end{table}

\pagebreak

%\textbf{Asesor de Documentación}
%\pagebreak
\begin{table}[h!]
  \begin{tabular}{|p{0.45\textwidth}|p{0.45\textwidth}|}
    \hline
    \textbf{Descripción} & El asesor de documentación para equipo de desarrollo \\
    \hline
    \textbf{Tipo} & Asesor del Equipo de Desarrollo \\
    \hline
    \textbf{Responsabilidades} & La aprobación de los avances en la documentación y la documentación completa al final del proyecto. \\
    \hline
    \textbf{Criteria de Exito} & La entrega de la documentación final \\
    \hline
    \textbf{Involucramiento} & En cada fase del proyecto \\
    \hline
    \textbf{Entregables} & Aprobación de la Documentación Final y los respectivos avances \\
    \hline
    \textbf{Comentarios / Preocupaciones} & Que se logra realizar la documentación. \\
    \hline
  \end{tabular}
  \caption{Perfil de Interesado: Asesor de Documentación.}
  \label{per-inter-a-doc}
\end{table}

%%%%%%%%%%%%%%%%%%%%%%%%%%%%%
\vfill

\subsection{Perfiles de Usuarios}
Para entender a fondo cada clase de usuario se da a continuacion un analisis de los mismos.

%\textbf{Estudiante}
%\pagebreak
\begin{table}[h!]
  \begin{tabular}{|p{0.45\textwidth}|p{0.45\textwidth}|}
    \hline
    \textbf{Descripción} & Estudiantes de la Modalidad Presencial y a Distancia de las carreras de Sistemas y Electrónica \\
    \hline
    \textbf{Tipo} & Usuario Final Primaria \\
    \hline
    \textbf{Responsabilidades} & Probar el sistema \\
    \hline
    \textbf{Criteria de Exito} & Que el sistema sea de utilidad para su experiencia educativa \\
    \hline
    \textbf{Involucramiento} & En la fase de pruebas \\
    \hline
    \textbf{Entregables} & Ninguno \\
    \hline
    \textbf{Comentarios / Preocupaciones} & Que la aplicación sea fácil de usar \\
    \hline
  \end{tabular}
  \caption{Perfil de Usuario: Estudiante.}
  \label{per-user-estu}
\end{table}

%\textbf{Professor}
%\pagebreak
\begin{table}[h!]
  \begin{tabular}{|p{0.45\textwidth}|p{0.45\textwidth}|}
    \hline
    \textbf{Descripción} & Profesores de Programación y de Bases de Datos de la Modalidad Presencial y a Distancia de las carreras de Sistemas y Electrónica \\
    \hline
    \textbf{Tipo} & Usuario Final Primaria \\
    \hline
    \textbf{Responsabilidades} & Probar el sistema \\
    \hline
    \textbf{Criteria de Exito} & Que el sistema sea de utilidad para su docencia \\
    \hline
    \textbf{Involucramiento} & En la fase de pruebas \\
    \hline
    \textbf{Entregables} & Ninguno \\
    \hline
    \textbf{Comentarios / Preocupaciones} & Que la aplicación les facilita el proceso de enseñanza \\
    \hline
  \end{tabular}
  \caption{Perfil de Usuario: Professor.}
  \label{per-user-prof}
\end{table}

\pagebreak

%\textbf{Administrador}
%\pagebreak
\begin{table}[h!]
  \begin{tabular}{|p{0.45\textwidth}|p{0.45\textwidth}|}
    \hline
    \textbf{Descripción} & Administradores Institucionales de Sistemas y de Bases de Datos \\
    \hline
    \textbf{Tipo} & Usuario Final Secundaria \\
    \hline
    \textbf{Responsabilidades} & Mantener el Sistema \\
    \hline
    \textbf{Criteria de Exito} & Que el sistema sea fácil de mantener \\
    \hline
    \textbf{Involucramiento} & En las fases de pruebas y despliegue \\
    \hline
    \textbf{Entregables} & Ninguno \\
    \hline
    \textbf{Comentarios / Preocupaciones} & Que la aplicación sea lo suficientemente documentado \\
    \hline
  \end{tabular}
  \caption{Perfil de Usuario: Administrador.}
  \label{per-user-admn}
\end{table}

\subsection{Necesidades de Interesados y Usuarios Principales}
En base a los interesados y usuarios principales definidos, se define las necesidades del sistema a continuacion.

%\begin{table}[h!]
%	\begin{tabular}{|p{0.13\textwidth}|p{0.13\textwidth}|p{0.20\textwidth}|p{0.14\textwidth}|p{0.20\textwidth}|}
% TODO: Caption before longtable, not after
\begin{longtable}{|p{0.13\textwidth}|p{0.13\textwidth}|p{0.20\textwidth}|p{0.14\textwidth}|p{0.20\textwidth}|}
		\hline
        \textbf{Necesidad} & \textbf{Prioridad} & \textbf{Preocupaciones} & \textbf{Solución Seleccionado} & \textbf{Soluciones Propuestas} \\
        \hline
        \endhead
        Seguridad en el Ambiente de Ejecución de Código & Alta & Los partes del sistema que se encargan de ejecución de código, necesitan alta protección contra usuario maliciosos y daños accidentales. & Ver las soluciones propuestas & Virtualización:	
        	\begin{itemize}
        		\item \small{Hipervisor de Tipo 1}
                \item Hipervisor de Tipo 2
                \item Contener-ización
        	\end{itemize}
            \\
        \hline
        Extensi-bilidad & Alta & El sistema debe soportar funcionalidad agregadas en el futuro. & Aplicación orientado a la Modularidad, ver la solución propuesta para mayor detalle. & Modularidad entre componentes del sistema para facilitar el proceso de agregar componentes nuevo o reemplazar componentes existentes sin tocar los demás \\
        \hline
        Facilidad en Autenticación & Mediana & El sistema debe ser fácil para autenticar todos sus usuarios. & Autentica-ción contra LMS existente mediante LTI & Autenticación contra LMS existente mediante LTI \\
        \hline
        Facilidad en Notas & Mediana & El sistema debe ser capaz de registrar notas en sistemas externas sin interacción del profesor. & Registro de Notas mediante LTI & Registro de Notas mediante LTI \\
        \hline
        Facilidad en Persistencia de Código & Mediana & El sistema, sin necesidad de interacción del usuario debe persistir el código escrito en un servidor de algún sistema de control de versiones externa & Transferencia de código mediante sistemas de control de versiones ya existentes & Transferencia de código mediante sistemas de control de versiones ya existentes \\
        \hline        
%	\end{tabular}
	\caption{Necesidades de Interesados y Usuarios Principales}
    \label{nec-inter-user}
%\end{table}
\end{longtable}

\subsection{Alternativas y Competidores}
\begin{enumerate}
	\item Repl.it
    \item Cloud9 IDE
\end{enumerate}

\pagebreak

\section{Vista General de Producto}
\subsection{Perspectiva del Producto}
\begin{figure}[h!]
  \begin{center}
  	% TODO: Bigger?
    \includegraphics[width=0.75\textwidth]{Figures/pers-prod.png}
  \end{center}
  \caption{Perspectiva del Producto.}
  \label{pers-prod}
\end{figure}

%\pagebreak

\subsection{Resumen de Capacidades}
\begin{table}[h!]
  \begin{tabular}{|p{0.45\textwidth}|p{0.45\textwidth}|}
    \hline
    \textbf{Beneficio al Cliente} & \textbf{Característica de Apoyo} \\
    \hline
    Facilidad de Autenticación & Autenticación por LTI contra LMS institucional \\
    \hline
    Gestión de la Configuración para el Código Escrito & Uso de Servidores Externos de Control de Versiones \\
    \hline
    Ejecución y Pruebas Unitarias en Línea & Sistema de Apoyo con Ambiente de Ejecución de Código \\
    \hline
    Generación y Sincronización de Notas & Calificación a través de pruebas unitarias, Sincronización de Notas a través de LTI con LMS institucional \\
    \hline
  \end{tabular}
  \caption{Resumen de Capacidades.}
  \label{res-cap}
\end{table}

\subsection{Presuposiciones y Dependencias}
\begin{enumerate}
	\item La institución seguirá usando Moodle y Open EDX como sus plataformas y proveedores de LMS.
	\item La institución seguirá usando como proveedor de un servidor de control de versiones de la institución, un servidor institucional con GitLab Community Edition.
	\item Alternativas como Repl.it y Cloud9 IDE no dan la funcionalidad necesaria o son demasiados costosos para su uso general por la institución.
\end{enumerate}

%\pagebreak

\section{Caracteristicas de Producto}
\begin{enumerate}
	\item Caracteristicas de Autenticación
    	\begin{enumerate}
			\item Autenticarse por LTI
			\item Autenticarse con Usuario y Contraseña
			\item Cerrar Session   
    	\end{enumerate}
    
	\item Caracteristicas de Editar Código
    	\begin{enumerate}
			\item Nuevo Proyecto
			\item Nuevo Proyecto basado en otro Proyecto
			\item Ver Proyectos
			\item Ver Detalles de un Proyecto
			\item Editar Detalles de un Proyecto
			\item Gestionar Usuarios de un Proyecto
			\item Gestionar Permisos de un Proyecto
			\item Abrir Proyecto
			\item Editar Archivos en un Proyecto
			\item Crear Nuevo Archivo en un Proyecto
			\item Borrar Archivo de un Proyecto
			\item Interactuar con Usuarios dentro de un Proyecto
    	\end{enumerate}
    
	\item Caracteristicas de Persistencia de Codigo
    	\begin{enumerate}
			\item Guardar cambios en un Proyecto
			\item Subir cambios en un Proyecto
			\item Bajar cambios en un Proyecto
    	\end{enumerate}
	\item Caracteristicas de Ejecutar y Calificar Código
    	\begin{enumerate}
			\item Ejecutar Codigo en ambiente interactivo en el navegador
			\item Agregar Pruebas Unitarias a un Proyecto
			\item Ejecutar Pruebas Unitarias en un Proyecto
			\item Calificar un Proyecto
			\item Calificiar un Proyecto de forma Automática en base a Pruebas Unitarias
			\item Enviar Calificaciones a un Sistema Externo
    	\end{enumerate}
\end{enumerate}

\pagebreak

\section{Precedencia y Prioridades}
\begin{table}[h!]
  \begin{tabular}{|p{0.45\textwidth}|p{0.45\textwidth}|}
    \hline
    \textbf{Prioridad} & \textbf{Característica (Según su número)} \\
    \hline
    Alta & 1.a, 1.c, 2.a, 2.b, 2.c, 2.h, 2.i, 3.a, 3.b, 4.a \\
    \hline
    Media & 1.b, 2.f, 2.g, 2.j, 3.c, 4.b, 4.c, 4.d, 4.f \\
    \hline
    Baja & 2.d, 2.e, 2.k, 2.l, 4.e \\
    \hline
  \end{tabular}
  \caption{Precedencia y Prioridades.}
  \label{precedencia-y-prioridades}
\end{table}

\section{Restricciones}
\subsection{Seguridad}
Protección contra ataques en la red

Aislamiento de ambientes de ejecución de código de usuarios
\subsection{Extensibilidad}
Soporte al nivel de sistema y documentación para extensión con más lenguajes de programación y motores de base de datos al futuro
\subsection{Usabilidad}
Ser usable

Requerir un mínimo de interacción del usuario para que puede enfocarse en realizar sus responsabilidades en el sistema
\subsection{Escalabilidad}
Tener capacidad para escalar frente mayor carga en el futuro
\subsection{Rendimiento}
Minimizar las requerimientos mínimos de hardware requerido
\section{Otros Requisitos de Producto}
\subsection{Normas}
Ninguna.
\subsection{Requisitos de Sistema}
Ninguno.
\subsection{Requisitos de Rendimiento}
Ninguno.
\subsection{Requisitos Ambientales}
Ninguno.
\section{Requisitos de Documentación}
\subsection{Manual del Programador}
Documentación para explicar el funcionamiento de la aplicación para que futuros desarrolladores pueden extender sin mayor dificultad la aplicación.
\subsection{Manual de Mantenimiento}
Documentación para definir los procesos de mantenimiento de la aplicación para guiar la gobernanza y administración del mismo.
\subsection{Manual de Usuario}
Documentación para enseñar el uso de la aplicación a docentes y alumnos.



% Appendix B - Software Requirements Specification

\chapter{Especificacion de Requisitos de Software}

% Appendix C-E - Apendices de Desarrollo

%\chapter{Frequently Asked Questions} % Main appendix title

\chapter{Preparaciones de Desarrollo}
\label{AnexoC} % For referencing this appendix elsewhere, use \ref{AppendixA}

\section{Servicios Nativos}
% TODO: Intro to Section?

\subsection{Servidor de Git}
Instalar paquetes para un interfaz web sencilla de Git:
\begin{lstlisting}
apt install git gitweb fcgiwrap
\end{lstlisting}

Se crea un usuario del sistema operativo solo para Git:
\begin{lstlisting}
adduser git
\end{lstlisting}

Se autentica como el usuario creado:
\begin{lstlisting}
su - git
\end{lstlisting}

Para tener repositorios ejemplares, se clona algunos repositorios:
\begin{lstlisting}
git clone --bare\
	https://gitlab.com/nishedcob/GitEDU.git\
	repositories/GitEDU.git
git clone --bare\
	https://gitlab.com/ArqAppGrpBravoEarleyVargas/\
    	GitEduERP.git\
	repositories/GitEduERP.git
\end{lstlisting}

Entrar al directorio de repositorios y arreglar permisos para permitir git push desde repositorios remotos:
\begin{lstlisting}
cd repositories/
for repo in `ls`; do
	cd $repo;
    pwd;
    chmod -R g+ws .;
    chgrp -R git .;
    git --bare update-server-info;
    cp hooks/post-update.sample hooks/post-update;
    chmod a+x hooks/post-update;
    cd ..;
done
\end{lstlisting}

Después se configura NGinX para trabajar con gitweb:
\begin{lstlisting}
# NGinX Config:
server {
        listen 80;
        listen [::]:80;
        server_name git.localhost 192.168.99.1 10.10.10.1;
        root /usr/share/gitweb;
        access_log /var/log/nginx/gitweb.access.log;
        # static repo files for cloning over https
        location ~ ^.*\.git/objects/([0-9a-f]+/[0-9a-f]+\
        	|pack/pack-[0-9a-f]+.(pack|idx))$ {
                root /home/git/repositories/;
        }

        # requests that need to go to git-http-backend
        location ~ ^.*\.git/(HEAD|info/refs|objects/info/.*\
        	|git-(upload|receive)-pack)$ {
                root /home/git/repositories;

                fastcgi_pass unix:/var/run/fcgiwrap.socket;
                fastcgi_param SCRIPT_FILENAME   /usr/lib/\
                	git-core/git-http-backend;
                fastcgi_param PATH_INFO          $uri;
                fastcgi_param GIT_PROJECT_ROOT  /home/git/\
                	repositories;
                fastcgi_param GIT_HTTP_EXPORT_ALL "";
                fastcgi_param REMOTE_USER $remote_user;
                include fastcgi_params;
        }

        # send anything else to gitweb if it's not a real file
        try_files $uri @gitweb;
        location @gitweb {
                fastcgi_pass unix:/var/run/fcgiwrap.socket;
                fastcgi_param SCRIPT_FILENAME   /usr/share/\
                	gitweb/gitweb.cgi;
                fastcgi_param PATH_INFO          $uri;
                fastcgi_param GITWEB_CONFIG      /etc/gitweb\
                	.conf;
                include fastcgi_params;
        }
}
\end{lstlisting}

Además se configura gitweb con la siguiente configuración:
\begin{lstlisting}
# Edit /etc/gitweb.conf
# path to git projects (<project>.git)
#$projectroot = "/var/lib/git";
$projectroot = "/home/git/repositories";
\end{lstlisting}

Prueba de que funciona:
\begin{lstlisting}
git clone http://git.localhost/nishedcob/GitEDU.git \
	GitEDU-test
\end{lstlisting}

Dado que el mismo GitWeb no tenga problemas con bajada de datos (\texttt{git fetch} / \texttt{git pull} / \texttt{git clone}) pero si tiene problemas con subida de datos como se lo require en EduNube para armar repositorios de ejecuccion validadas, se ve una necesidad de habilitar subida al mismo por SSH:

Como root en el servidor de Git, se cambia la clave del usuario de Git:
\begin{lstlisting}
passwd git
\end{lstlisting}

Como el usuario de EduNube en el servidor para el mismo:
\begin{lstlisting}
cd .ssh/
# Deja la llave generada sin clave:
ssh-keygen -f id_git
# Copia la llave publica al servidor
ssh-copy-id -i ~/.ssh/id_git.pub git@10.10.10.1
# Prueba que funciona
ssh -i ~/.ssh/id_git -vvv git@10.10.10.1
# Guardar la configuracion:
cat >> ~/.ssh/config < EOF

Host git
     HostName 10.10.10.1
     User git
     Port 22
     IdentityFile $HOME/.ssh/id_git

EOF
# Probar con:
ssh -vvv git
\end{lstlisting}

\section{Servicios Virtualizados}
% TODO: Section Introduction?

\subsection{Maquina Virtual de Xen para Moodle}
Para el ambiente de Moodle (LMS \index{LMS} contra el cual se ha llevado el desarrollo), se crea una maquina virtual de Debian Stretch (9) con 1 GiB de RAM, 1 CPU virtual, 6 GiB de disco, 512 MiB de intercambio y una dirección IP fija de 10.10.10.10. Los resultados del mismo comando se puede ver en la figura \ref{vm-moodle}.
\begin{lstlisting}
	xen-create-image --hostname=debian-moodle\
    		--ip=10.10.10.10 --netmask=255.255.255.0\
        	--gateway=10.10.10.1 --memory=1024mb\
        	--vcpus=1 --lvm=Xephyr-VG --pygrub\
        	--dist=stretch --force --size=6144mb\
        	--swap=512mb
\end{lstlisting}

\begin{figure}
	\begin{center}
    	\includegraphics[width=0.75\textwidth]{Figures/crear-moodle.png}
    \end{center}
  	\caption{Crear maquina virtual para Moodle.}
    \label{vm-moodle}
\end{figure}

Se renombró el archivo de configuración de la maquina virtual generado en el paso anterior para temas de consistencia.

\begin{lstlisting}
	mv /etc/xen/debian-moodle.cfg\
    		/etc/xen/domU-debian-moodle.cfg
\end{lstlisting}

Un bug de Xen-Tools causa que no se instala correctamente un núcleo de Linux en la maquina virtual y por lo tanto es necesario entrar al mismo con un Chroot y instalar los paquetes faltantes (y hacer las adecuadas configuraciones para permitir su arranque independiente de ayuda externa)\footnote{6 meses despues de la redaccion de estos pasos, y Xen-Tools ya no tiene este bug, motivo por el cual se puede emitir este paso si es que la maquina virtual esta funcionando de forma correcta.}.

\begin{lstlisting}
	mount /dev/Xephyr-VG/debian-moodle-disk /mnt
	mount -o bind /proc /mnt/proc
	mount -o bind /sys /mnt/sys
	mount -o bind /dev /mnt/dev
	cp /etc/resolv.conf /mnt/etc/resolv.conf
	chroot /mnt /bin/bash
	apt install linux-image-amd64
	vim.tiny /boot/grub/menu.lst
	# Revisar que los archivos referenciados existen
	#		de verdad por ejemplo:
	# Replace initrd.img- con initrd.img
    # guarda y sale
	exit
	umount /mnt/proc            
	umount /mnt/sys 
	umount /mnt/dev 
	umount /mnt	
\end{lstlisting}

Para levantar la maquina virtual:

\begin{lstlisting}
	xl create /etc/xen/domU-debian-moodle.cfg -c
\end{lstlisting}

Se debe seleccionar la tercera opción (Default Kernel).

Se puede dar una revisión a la configuración de red para asegurarse de que esta correcto:

\begin{lstlisting}
	vim.tiny /etc/network/interfaces
\end{lstlisting}

Debe contener:

\begin{lstlisting}
auto eth0
iface eth0 inet static
 address 10.10.10.10
 gateway 10.10.10.1
 netmask 255.255.255.0
\end{lstlisting}

En el presente caso, Xen-Tools logró configurar esta parte de forma correcta.

A continuación se procede con la instalación de Moodle:

\begin{lstlisting}
# actualiza el sistema
apt update
apt upgrade

# instala dependencias
apt install apache2 php7.0 mysql-server php7.0-mysql
apt install libapache2-mod-php7.0 php7.0-gd php7.0-curl
apt install php-xml php-zip php-mbstring php-soap
apt install php7.0-xmlrpc php7.0-intl
vim.tiny /etc/php/7.0/apache2/php.ini

# agrega:
extension=mysql.so 
extension=gd.so

# edita:

memory_limit = 40M
# dejado con el valor por defecto de 128M

post_max_size = 80M
upload_max_filesize = 80M

# guarda y sale

# reinicia apache para coger los cambios
systemctl restart apache2

\end{lstlisting}

A continuación se configura la base de datos:

\begin{lstlisting}

# clave de root es root
mysqladmin -u root password "root"

# iniciar session como root
mysql -u root -p

# crear base de datos y hacer que ocupa UTF-8
mysql> CREATE DATABASE moodle;
mysql> ALTER DATABASE moodle charset=utf8;
mysql> exit;

## No se implemento ##
# Moodle queja de UTF8

# Se podria arreglar
# (antes de instalar Moodle)
# con:

mysql -u root -p

mysql> ALTER DATABASE moodle charset=utf8mb4;
mysql> exit;
######################

systemctl restart mysql

\end{lstlisting}

Se realiza la instalación de la ultima versión de Moodle (3.3 con sus respectivos patches de fallas desde que el mismo salio):

\begin{lstlisting}

# descargar
wget https://download.moodle.org/download.php/direct/stable33/\
	moodle-latest-33.tgz

# descomprimir
tar -zxvf moodle-latest-33.tgz

# meter en la ubicacion para apache
mv moodle /var/www

# ir a ubicacion para apache
cd /var/www

# crear ubicacion para datos
mkdir moodledata

# arreglar permisos
chown -R www-data:www-data moodle
chown -R www-data:www-data moodledata
chmod -R 755 moodle
chmod -R 755 moodledata

# modificar configuracion de apache
vim.tiny /etc/apache2/sites-available/000-default.conf

# editar
DocumentRoot "/var/www/moodle"

# guardar y salir

# reiniciar apache para aplicar los cambios
systemctl restart apache2

\end{lstlisting}

Arreglar la base de datos para que acepte conexiones desde Moodle:

\begin{lstlisting}

mysql -u root -p

GRANT ALL PRIVILEGES on *.* to
	'root'@'localhost' IDENTIFIED BY 'root';
GRANT ALL PRIVILEGES on *.* to
	'root'@'localhost' IDENTIFIED BY 'root';
FLUSH PRIVILEGES;
exit;

\end{lstlisting}

Para seguir con la instalación se abre un navegador con la dirección \url{http://10.10.10.10/} para seguir las instrucciones que se le lleve por toda la configuración inicial del Moodle.

Al final se agregue un trabajo de cron para ayudar con los tareas periódicas que Moodle requiere para su mantenimiento continuo:

\begin{lstlisting}

crontab -u www-data -e
# add line:
*/10 * * * * /usr/bin/php
		/var/www/moodle/admin/cli/cron.php
        		>/dev/null

\end{lstlisting}

Estos pasos fueron adaptados de la guía oficial del proyecto de Moodle para instalación en Debian \citep{MOODLE-Install-Debian}.

Ahora que todo esta funcionando se recomienda editar /boot/grub/menu.lst para comentar las entradas del pyGrub que son defectuosas (los primeros dos) para que se puede levantar la maquina virtual sin intervención humana.

\subsection{Máquina Virtual de Xen}
Para crear el servidor de Git con GitLab Community Edition, se va a empezar con las mismas piezas del servidor del LMS/Moodle, es decir una maquina virtual de Debian 9. El mismo se crea como una maquina virtual de Debian Stretch (9) con 1 GiB de RAM, 1 CPU virtual, 6 GiB de disco, 512 MiB de intercambio y una dirección IP fija de 10.10.10.11.
\begin{lstlisting}
	xen-create-image --hostname=debian-gitlab\
    		--ip=10.10.10.11 --netmask=255.255.255.0\
        	--gateway=10.10.10.1 --memory=1024mb\
        	--vcpus=1 --lvm=Xephyr-VG --pygrub\
        	--dist=stretch --force --size=6144mb\
        	--swap=512mb
\end{lstlisting}

Se renombró el archivo de configuración de la maquina virtual generado en el paso anterior para temas de consistencia.

\begin{lstlisting}
	mv /etc/xen/debian-gitlab.cfg\
    		/etc/xen/domU-debian-gitlab.cfg
\end{lstlisting}

Un bug de Xen-Tools causa que no se instala correctamente un núcleo de Linux en la maquina virtual y por lo tanto es necesario entrar al mismo con un Chroot y instalar los paquetes faltantes (y hacer las adecuadas configuraciones  para permitir su arranque independiente de ayuda externa)\footnote{6 meses despues de la redaccion de estos pasos, y Xen-Tools ya no tiene este bug, motivo por el cual se puede emitir este paso si es que la maquina virtual esta funcionando de forma correcta.}.

\begin{lstlisting}
	mount /dev/Xephyr-VG/debian-gitlab-disk /mnt
	mount -o bind /proc /mnt/proc
	mount -o bind /sys /mnt/sys
	mount -o bind /dev /mnt/dev
	cp /etc/resolv.conf /mnt/etc/resolv.conf
	chroot /mnt /bin/bash
	apt install linux-image-amd64
	vim.tiny /boot/grub/menu.lst
	# Revisar que los archivos referenciados existen
	#		de verdad por ejemplo:
	# Replace initrd.img- con initrd.img
    # guarda y sale
	exit
	umount /mnt/proc            
	umount /mnt/sys 
	umount /mnt/dev 
	umount /mnt	
\end{lstlisting}

Para levantar la maquina virtual:

\begin{lstlisting}
	xl create /etc/xen/domU-debian-gitlab.cfg -c
\end{lstlisting}

Se debe seleccionar la tercera opción (Default Kernel).

Se puede dar una revisión a la configuración de red para asegurarse de que esta correcto:

\begin{lstlisting}
	vim.tiny /etc/network/interfaces
\end{lstlisting}

Debe contener:

\begin{lstlisting}
auto eth0
iface eth0 inet static
 address 10.10.10.11
 gateway 10.10.10.1
 netmask 255.255.255.0
\end{lstlisting}

En el presente caso, Xen-Tools logró configurar esta parte de forma correcta.

A continuación se procede con la instalación de Gitlab:

\begin{lstlisting}
# actualiza el sistema
apt update
apt upgrade

# A partir del Debian 9, Gitlab CE esta ofrecido
#    en los repositorios oficiales:
apt install gitlab

# Arreglar problema con API
vim.tiny /usr/share/gitlab-shell/config.yml
# cambiar gitlab_url a "http://10.10.10.11/"

/usr/share/gitlab-shell/bin/check
\end{lstlisting}

Visitar \url{http://10.10.10.11/} y configurar la contraseña del usuario root. Con el usuario root configurado se procede a crear un usuario especial para la aplicación con el nombre de usuario 'GitEdu'. Este usuario lo damos permisos de administración y también se genera un token en esta dirección: \url{http://10.10.10.11/profile/personal_access_tokens}. El token se debe guardar para su uso después (GitLab nunca le vuelve a mostrar).

Ahora que todo esta funcionando se recomienda editar /boot/grub/menu.lst para comentar las entradas del pyGrub que son defectuosas (los primeros dos) para que se puede levantar la maquina virtual sin intervención humana.

\section{Kubernetes}
% TODO: Introduccion a Kubernetes?

\index{Kubernetes}
\subsection{Instalación de Kubernetes}
Para instalar la ultima version del cliente de Kubernetes, tanto en desarrollo como en produccion, solo se necesita bajar la ultima version de kubectl desde los repositorios de Google y poner el mismo dentro del \$PATH:
\begin{lstlisting}[breaklines=true]
# Bajar la ultima version de Kubectl para Linux:
curl -LO https://storage.googleapis.com/kubernetes-release/\
	release/$(curl -s https://storage.googleapis.com/\
    	kubernetes-release/release/stable.txt)\
	/bin/linux/amd64/kubectl
# Hacerlo ejecutable
chmod +x kubectl
# Revisar el $PATH actual
echo $PATH
# Ubicar dentro del $PATH
mv kubectl ~/bin/
# verificar instalacion con:
kubectl --help
# o con:
kubectl version
# Este segundo commando debe dar un error de servidor hasta\
#	que se instala y configura el servidor al cual se\
#	debe connectar
\end{lstlisting}

\index{Kubernetes}
\subsection{MiniKube}
Dentro del ambiente de desarrollo, fue ocupado un cluster de Kubernetes, de solo un nodo, virtualizado en VirtualBox. La instalacion del mismo se detalle a continuacion:
\begin{lstlisting}[breaklines=true]
# Se descarga desde los repositorios de Google el commando de MiniKube
curl -Lo minikube https://storage.googleapis.com/minikube/releases/v0.23.0/minikube-linux-amd64
# Se lo hace ejecutable
chmod +x minikube
# se lo ubica en el $PATH
mv minikube ~/bin/
# para validar la instalacion:
minikube --help
# o
minikube version
\end{lstlisting}

Para levantar el cluster de MiniKube y realizar la configuracion automatica del entorno para el uso del mismo, se utiliza el commando:
\begin{lstlisting}
minikube start
\end{lstlisting}

Para bajar el cluster de MiniKube:
\begin{lstlisting}
minikube stop
\end{lstlisting}
\index{Contenedor} \index{Virtualización} \index{Hipervisor}

\index{Docker|(}
\section{Docker}
% TODO: Introduccion?

\subsection{Tipo de Contenedor: Shell-Executor}
% TODO: Introduccion?

\subsubsection{Debian}
El Debian Shell-Executor se crea con los siguientes lineas en el Dockerfile: 
\begin{lstlisting}[breaklines=true]
FROM debian:stretch

ENV shell=/bin/sh
ENV user=user

# Greatly increases image size, optional:
#RUN apt-get update && apt-get install -y lsb-release

RUN mkdir -p /code && echo "USER=$user" && echo "SHELL=$shell" && useradd -ms $shell $user && chown -v $user:$user /code
VOLUME ["/code"]

ENTRYPOINT ["/bin/sh"]
CMD ["/code/exec.sh"]
\end{lstlisting}
La primera linea, ''FROM ...'' define el imagen base, en este caso la versión actual de Debian Stretch (Debian 9.x). La tercera y cuarta linea, ''ENV ...'', definen variables de entorno que se puede cambiar previa a la construcción del imagen en adición a sus valores por defecto. La sexta y séptima linea, documenta y define un paso opcional donde se instala un paquete adicional, lsb-release, que provee información estandarizado de la distribución de Linux en ejecución. Este paso opcional agrega mucho peso al contenedor y por lo tanto no se lo ha dejado comentado. La novena y décima linea define comandos preliminares para preparar el contenedor como:
\begin{itemize}
	\item Crear una carpeta "/code" donde se puede copiar el código a ser ejecutado desde afuera del contenedor
	\item Visualización los variables de entorno para temas informativos y de debug
	\item La creación del usuario que vamos a utilizar dentro del contenedor
	\item Dar el usuario los permisos adecuados para que tiene acceso a la carpeta creada
\end{itemize}
La onceava linea define la carpeta creada como un volumen de Docker, el cual se monta externamente al crear y ejecutar el contenedor. La decimotercera linea define el argumento cero del contenedor al momento de iniciar la misma, en este caso el shell que se encuentra del contenedor. La decimocuarta linea define los argumentos con que se debe ejecutar el comando definido previamente, en este caso el script de entrada, cargado desde un sistema de ficheros externo al contenedor.

\subsubsection{Alpine Linux}
El Alpine Shell-Executor se crea con los siguientes lineas en el Dockerfile: 
\begin{lstlisting}
FROM alpine:3.6

ENV shell=/bin/sh
ENV user=user

RUN mkdir -p /code && echo "USER=$user" && echo "SHELL=$shell"\
&& echo "SHELL is not used in this Dockerfile" &&\
    adduser -D $user && chown -v $user:$user /code
VOLUME ["/code"]

ENTRYPOINT ["/bin/sh"]
CMD ["/code/exec.sh"]
\end{lstlisting}
La primera linea, ''FROM ...'' define el imagen base, en este caso la versión actual de Alpine Linux (Alpine 3.6). La tercera y cuarta linea, ''ENV ...'', definen variables de entorno que se puede cambiar previa a la construcción del imagen en adición a sus valores por defecto. La sexta, séptima y octava lineas define comandos preliminares para preparar el contenedor como:
\begin{itemize}
	\item Crear una carpeta "/code" donde se puede copiar el código a ser ejecutado desde afuera del contenedor
	\item Visualización los variables de entorno para temas informativos y de debug
	\item La creación del usuario que vamos a utilizar dentro del contenedor
	\item Dar el usuario los permisos adecuados para que tiene acceso a la carpeta creada
\end{itemize}
La novena linea define la carpeta creada como un volumen de Docker, el cual se monta externamente al crear y ejecutar el contenedor. La onceava linea define el argumento cero del contenedor al momento de iniciar la misma, en este caso el shell que se encuentra del contenedor. La duodécima linea define los argumentos con que se debe ejecutar el comando definido previamente, en este caso el script de entrada, cargado desde un sistema de ficheros externo al contenedor. 

\subsubsection{Plantilla de Ejecuccion}


\subsection{Tipo de Contenedor: Python3-Executor}
% TODO: Introduction?

\subsubsection{Debian}
El Debian Python3-Executor se define de la siguiente manera a través de su Dockerfile:
\begin{lstlisting}[breaklines=true]
FROM registry.gitlab.com/nishedcob/gitedu/shell-executor:debian-stretch

RUN apt-get update && apt-get install -y python3 python3-dev python3-pip virtualenv

VOLUME ["/code"]
WORKDIR "/code"
\end{lstlisting}
La primera y segunda linea define herencia del contenedor Debian Shell-Executor que definimos previamente. La tercera y cuarta linea instala Python3, Pip3 y Virtualenv dentro del contenedor. La sexta linea vuelve a declarar el volumen de Docker porque al aparecer la versión actual de Docker a propósito no suporta herencia de esta linea en los Dockerfile. La séptima linea define la ubicación inicial utilizada cuando se inicia el contenedor.

\subsubsection{Alpine Linux}
El Alpine Python3-Executor se define a través del Dockerfile que esta a continuación:
\begin{lstlisting}[breaklines=true]
FROM registry.gitlab.com/nishedcob/gitedu/shell-executor:alpine-3.6

RUN apk update && apk add python3 python3-dev py-virtualenv

VOLUME ["/code"]
WORKDIR "/code"
\end{lstlisting}
La primera y segunda linea define herencia del contenedor Alpine Shell-Executor que definimos previamente. La tercera linea instala Python3, Pip3 y Virtualenv dentro del contenedor. La quinta linea vuelve a declarar el volumen de Docker porque al aparecer la versión actual de Docker a propósito no suporta herencia de esta linea en los Dockerfile. La sexta linea define la ubicación inicial utilizada cuando se inicia el contenedor.

\subsection{Tipo de Contenedor: PostgreSQL-Executor}
% TODO: introduccion?

\subsubsection{Debian}
El Debian PostgreSQL-Executor se define con el siguiente Dockerfile:
\begin{lstlisting}[breaklines=true]
FROM registry.gitlab.com/nishedcob/gitedu/shell-executor:debian-stretch

RUN apt-get update && apt-get install -y postgresql postgresql-client

RUN echo "Starting PostgreSQL Cluster..." ; /usr/bin/pg_ctlcluster 9.6 main start && echo "Started cluster!" || echo "Failed to start cluster!"; su - postgres -c "createuser user && createdb -O user userdb"; echo "Stopping PostgreSQL Cluster..." ; /usr/bin/pg_ctlcluster 9.6 main stop && echo "Stopped cluster!" || echo "Failed to stop cluster!";

VOLUME ["/code"]
WORKDIR "/code"
\end{lstlisting}
La primera y segunda linea define herencia del contenedor Debian Shell-Executor que definimos previamente. La tercera y cuarta linea instala PostgreSQL dentro del contenedor. La sexta hasta décima linea levanta el motor de base de datos PostgreSQL con la finalidad de crear un usuario y base de datos por defecto sobre el cual se puede trabajar. Finalizando este proceso se desactiva el contenedor para reducir el tamaño del mismo y no introducir comportamiento desconocido. La duodécima linea vuelve a declarar el volumen de Docker porque al aparecer la versión actual de Docker a propósito no suporta herencia de esta linea en los Dockerfile. La decimotercera linea define la ubicación inicial utilizada cuando se inicia el contenedor.

\subsubsection{Alpine Linux}
El Alpine PostgreSQL-Executor se define con el siguiente Dockerfile:
\begin{lstlisting}[breaklines=true]
FROM registry.gitlab.com/nishedcob/gitedu/shell-executor:alpine-3.6

RUN apk update && apk add postgresql && su - postgres -c "export PGDATA=/var/lib/postgresql/data && initdb"

RUN echo "Starting PostgreSQL Cluster..." ; mkdir -p /run/postgresql && chown -R postgres:postgres /run/postgresql && chmod 755 /run/postgresql && mkdir -p /var/run/postgresql && chown -R postgres:postgres /var/run/postgresql && chmod 2777 /var/run/postgresql && su - postgres -c "export PGDATA=/var/lib/postgresql/data && postgres &" && echo "Started cluster!" || echo "Failed to start cluster!"; sleep 5s && netstat -tupln && su - postgres -c "createuser user && createdb -O user userdb"; echo "Stopping PostgreSQL Cluster..." ; killall postgres && echo "Stopped cluster!" || echo "Failed to stop cluster!";

VOLUME ["/code"]
WORKDIR "/code"
\end{lstlisting}
La primera y segunda linea define herencia del contenedor Alpine Shell-Executor que definimos previamente. La tercera y cuarta linea instala PostgreSQL dentro del contenedor en adición a inicializar la base de datos. La sexta hasta decimoséptima linea levanta el motor de base de datos PostgreSQL con la finalidad de crear un usuario y base de datos por defecto sobre el cual se puede trabajar. Finalizando este proceso se desactiva el contenedor para reducir el tamaño del mismo y no introducir comportamiento desconocido. La decimonovena linea vuelve a declarar el volumen de Docker porque al aparecer la versión actual de Docker a propósito no suporta herencia de esta linea en los Dockerfile. La vigésima linea define la ubicación inicial utilizada cuando se inicia el contenedor.
\index{Docker|)}

\chapter{Desarollo de GitEDU}
\label{AnexoD}

\section{Configuracion de la Base de Datos Relacional}
Primero es necesario crear y configurar un usuario y base de datos para ser ocupado:
\begin{lstlisting}
postgresql-setup initdb
systemctl enable postgresql
systemctl start postgresql
systemctl status postgresql
vim /etc/postgresql/9.6/main/pg_hba.conf
# agregar una linea antes de las lineas similares y
# que diga (sin el numeral adelante):
#local   all      postgres             peer
systemctl restart postgresql
systemctl status postgresql
su - postgres
psql
\end{lstlisting}
\begin{lstlisting}
	postgres=# CREATE USER giteduser WITH PASSWORD 'g1T3d_$3r';
	postgres=# CREATE DATABASE gitedudb WITH OWNER giteduser;
	postgres=# \q
psql gitedudb -U postgres
	gitedudb=# CREATE SCHEMA giteduapp AUTHORIZATION giteduser;
	gitedudb=# ALTER USER giteduser SET search_path TO giteduapp;
	gitedudb=# \q
psql gitedudb -U giteduser
	gitedudb=> SELECT current_schema();
		Debe decir: giteduapp
	gitedudb=> \q
\end{lstlisting}

Segundo, se abre el settings.py (dentro de la carpeta GitEDU/GitEDU) y se remplaza (para desarrollo, en producción debe llevar valores distintos) el atributo DATABASES con lo siguiente:
\lstset{language=Python}
\begin{lstlisting}
DATABASES = {
    'default': {
        'ENGINE': 'django.db.backends.postgresql_psycopg2',
        'NAME': 'gitedudb',
        'USER': 'giteduser',
        'PASSWORD': 'g1T3d_$3r',
        'HOST': '127.0.0.1',
        'PORT': '5432',
    }
}
\end{lstlisting}
\lstset{language=Bash}

\section{Configuracion de la Base de Datos No Relacional}
Para crear la base de datos de MongoDB:
\begin{lstlisting}
mongo
\end{lstlisting}
\lstset{language=sql}
\begin{lstlisting}
use gitEduDB
db.createUser(
    {
        user: "gitEduUser",
        pwd: "G1TedU$3r",
        roles: [ "readWrite", "dbAdmin" ]
    }
)
\end{lstlisting}
\lstset{language=Bash}

El mismo se define en el settings de la siguiente manera:
\lstset{language=Python}
\begin{lstlisting}
NOSQL_DATABASES = {
    'nosql': {
        'NAME': 'gitEduDB',
        'USER': "gitEduUser",
        'PASSWORD': 'G1TedU$3r',
        'HOST': '127.0.0.1',
        'PORT': '27017',
    }
}
\end{lstlisting}
\lstset{language=Bash}

\section{Compatibilidad con EduNube en el Mismo Repositorio}
Para reducir el numero de repositorios involucrados en este trabajo de titulación, se ha optado por llevar el desarrollo de EduNube dentro del mismo repositorio, con una separación de dependencias con otro entorno virtual y aislamiento de código en desarrollo con varias ramas de Git. Por lo tanto, se realizo una nueva rama de Git con el comando:
\begin{lstlisting}
git checkout -b gitedu
\end{lstlisting}

Y una refacturación del entorno virtual a ser ''env-ge'' en lugar de ''env'', en lugar de ''requirements.txt'', utilizar ''requirements.ge.txt'' y un nuevo script de activar el entorno, el cual se activa ahora con ''source activate-ge.sh'':
\begin{lstlisting}
#! /usr/bin/head -n 2 
# run with `source activate-ge.sh`
PROJECT=ge
ENV_DIR=env-$PROJECT
if [ ! -d $ENV_DIR ]; then
	virtualenv --python=python3 $ENV_DIR
fi
source $ENV_DIR/bin/activate
pip3 install -r requirements.$PROJECT.txt
\end{lstlisting}

Para desarrollar ambos sistemas en paralelo, y poder trabajar en ramas independientes con ambos sistemas levantados al mismo tiempo para probar y desarrollar características de integración, se clono de forma local el repositorio:
\begin{lstlisting}
git clone GitEDU GitEDU-copy
\end{lstlisting}

Para el desarrollo se ocupa el puerto 8000 de localhost para GitEDU y el puerto 8001 de localhost para EduNube:
\begin{lstlisting}
# Levantar GitEDU en el ambiente de desarrollo:
cd GitEDU-copy
source activate-ge.sh
git checkout gitedu
cd GitEDU
python manage.py runserver 8000

# Levantar EduNube en el ambiente de desarrollo:
cd GitEDU
source activate-en.sh
git checkout edunube
cd EduNube
python manage.py runserver 8001
\end{lstlisting}\footnote{Para produccion, se cambio el puerto 8001 de EduNube para 8010}


\index{Autenticación LTI}
\section{Autenticacion por LTI}
Primero es necesario que este activo y levantado el ambiente del LMS \index{LMS} como se documenta en la sección \ref{instalacion-moodle} de instalación de Moodle. Para el desarrollo de este trabajo de titulación se esta considerando una instalación en un servidor aparte (virtualizado en el mismo equipo) en la dirección IP 10.10.10.10.

% Instalar dependencias

Instalación de dependencias desde GitHub (no se encuentran en los repositorios oficiales de Pip/Pypi; además para superar problemas de dependencias en las librerías, se ha optado para ocupar forks personales del autor con las mejores necesarias para su funcionamiento):
\begin{lstlisting}
pip install git+https://github.com/nishedcob/django-app-lti
    @master#egg=django-app-lti
pip install git+https://github.com/nishedcob/
    django-auth-lti@master#egg=django-auth-lti
\end{lstlisting}

Como son dependencias no se encuentran en los repositorios oficiales, su manejo dentro del requirements.txt también tiene que ser especial ya que un `pip freeze` no los guardaran correctamente dentro del mismo. En lugar de eso hay que agregar dos lineas al requirements.txt para el manejo de estas dependencias:
\begin{lstlisting}
-e git+https://github.com/nishedcob/django-app-lti.git
    @38b32989e22b189345e421b183684f9b5453e99a
    #egg=django-app-lti
-e git+https://github.com/nishedcob/django-auth-lti.git
    @71c9da8d0aa07ebc3139bf3f113b5c521d61b1f1
    #egg=django-auth-lti
\end{lstlisting}

% Modificar settings

\lstset{language=Python}

Se pone a editar el settings.py (dentro de GitEDU/GitEDU) con las siguientes configuraciones:
\begin{itemize}
	\item a INSTALLED\_APPS agregamos las siguientes lineas:
   		\begin{lstlisting}
    'django_auth_lti',
    'django_app_lti',
    	\end{lstlisting}
    \item a MIDDLEWARE agregamos la siguiente linea:
   		\begin{lstlisting}
    'django_auth_lti.middleware.LTIAuthMiddleware',
    	\end{lstlisting}
    \item a AUTHENTICATION\_BACKENDS agregamos la siguiente linea:
   		\begin{lstlisting}
    'django_auth_lti.backends.LTIAuthBackend',
    	\end{lstlisting}
        Si es que no existe AUTHENTICATION\_BACKENDS lo creamos con los siguientes valores:
        \begin{lstlisting}
AUTHENTICATION_BACKENDS = (
    'django.contrib.auth.backends.ModelBackend',
    'django_auth_lti.backends.LTIAuthBackend',
)
        \end{lstlisting}
    \item también agregamos los siguientes atributos:
    	\begin{itemize}
    		\item LTI\_SETUP
            	\begin{lstlisting}
LTI_SETUP = {
    "TOOL_TITLE": "GitEDU",
    "TOOL_DESCRIPTION": "Sistema para Programar en Linea",
    "LAUNCH_URL": "lti:launch",
    "LAUNCH_REDIRECT_URL": "ideApp:decode",
    "INITIALIZE_MODELS": False,
    "EXTENSION_PARAMETERS": {
        "10.10.10.10": {
            "privacy_level": "public",
            "course_navigation": {
                "enabled": "true",
                "default": "disabled",
                "text": "GitEDU LMS Playground",
            }
        }
    }
}
            	\end{lstlisting}
            \item LTI\_OAUTH\_CREDENTIALS con texto aleatorio (fue ocupado OpenSSL, específicamente el comando \texttt{openssl rand -hex 10} para generar los valores de abajo\footnote{Un ambiente de producción debe ocupar valores distintos}).
            	\begin{lstlisting}
LTI_OAUTH_CREDENTIALS = {
    "GitEduLMS_Playground": "b2e0158c3cb4ddb0202d", 
         # (Para pruebas)
    "GitEduLMS_Playground_Assignments":
         "57b3a14734566c49bcaf",
         # (Para deberes/examenes/pruebas/talleres/etc)
    "GitEduLMS_Playground_Classes":
         "f7a0b6accc2631779e84",
         # (Para materias)
}
            	\end{lstlisting}
    	\end{itemize}
\end{itemize}

También se edita el urls.py (dentro de la misma dirección que el settings.py) con las siguientes lineas:
\begin{lstlisting}
from django.conf.urls import include
import django_app_lti.urls

# dentro de:
urlpatterns = [
    # agregar:
    url(r'^lti/', include(django_app_lti.urls, namespace="lti")),
]

\end{lstlisting}

\lstset{language=Bash}

Para arreglar un problema de una versión muy desactualizada de ims-lti-py, se agrego la siguiente linea al requirements.txt (para ocupar un fork del propio librería por el propio autor para resolver los problemas dados):
\begin{lstlisting}
-e git+https://github.com/nishedcob/ims_lti_py.git
    @a6576d7892ea4f69b76572788b118aaa4cdcf749
    #egg=ims_lti_py-develop
\end{lstlisting}

Para arreglar un problema de limpieza de datos en oauth2, se agrego la siguiente linea al requirements.txt (para ocupar un fork del propio librería por el propio autor para resolver los problemas dados):
\begin{lstlisting}
-e git+https://github.com/nishedcob/python-oauth2.git
    @176fc35aa35d626afcb6a23459482a4c96782c88
    #egg=oauth2
\end{lstlisting}

Después se migra la base de datos:
\begin{lstlisting}
python manage.py makemigrations
python manage.py migrate
\end{lstlisting}

\section{Migracion para llenar tabla AuthenticationType}
Con una migración se puede llenar de forma automática el catalogo de AuthenticacionType:
\lstset{language=Python}
\begin{lstlisting}
def fill_auth_types(apps, schema_editor):
    auth_type = models.AuthenticationType
    classic = auth_type(name="Clasica")
    classic.save()
    lti = auth_type(name="LTI")
    lti.save()


class Migration(migrations.Migration):

    dependencies = [
        ('authApp',
           '0002_authenticationtype_userauthentication'),
    ]

    operations = [
        migrations.RunPython(fill_auth_types),
    ]
\end{lstlisting}
\lstset{language=Bash}

\section{Accesibilidad a Datos de Autetnicacion de LTI}
Primero al settings se agrega dos atributos que indiquen cual llave se considera el sistema para compartir materias y otro para compartir deberes/exámenes/pruebas/talleres/etc\ldots{}:
\lstset{language=Python}
\begin{lstlisting}
LTI_ASSIGNMENTS_KEY = 'GitEduLMS_Playground_Assignments'
LTI_CLASSES_KEY = 'GitEduLMS_Playground_Classes'
\end{lstlisting}
\lstset{language=Bash}

Para controlar las partes de la configuración que se presenta a los usuarios finales, agregamos un nuevo campo de configuración al settings:
\lstset{language=Python}
\begin{lstlisting}
LTI_CONFIG_EXPOSE = {
    "LTI_KEYS": True,
    "LTI_ASSIGNMENT_KEY": True,
    "LTI_CLASS_KEY": True,
    "LTI_OTHER_KEYS": False,
    "LTI_SETUP": False,
}
\end{lstlisting}
\lstset{language=Bash}

Para que los profesores (como usuario final de GitEdu) también tengan acceso a estos credenciales para poderlos ocupar, se cree una vista en la ubicación '/auth/lti/credentials' que devuelve un JSON con los datos de LTI:
\lstset{language=Python}
\begin{lstlisting}[breaklines=true]
class LTICredentialsView(View):

    def get(self, request):
        if not request.user.is_authenticated:
            raise PermissionError("No tiene acceso a esta vista hasta que se autentica...")

        lti_expose = settings.LTI_CONFIG_EXPOSE

        lti_cred_json = {}

        if lti_expose['LTI_KEYS']:
            if lti_expose['LTI_ASSIGNMENT_KEY']:
                lti_cred_json['LTI_ASSIGNMENT_KEY'] = {
                    settings.LTI_ASSIGNMENTS_KEY: 
                        settings.LTI_OAUTH_CREDENTIALS
                            [settings.LTI_ASSIGNMENTS_KEY]
                }
            if lti_expose['LTI_CLASS_KEY']:
                lti_cred_json['LTI_CLASS_KEY'] = {
                    settings.LTI_CLASSES_KEY:
                        settings.LTI_OAUTH_CREDENTIALS
                            [settings.LTI_CLASSES_KEY]
                }
            if lti_expose['LTI_OTHER_KEYS']:
                lti_cred_json['LTI_OTHER_KEYS'] =
                    settings.LTI_OAUTH_CREDENTIALS

        if lti_expose['LTI_SETUP']:
            lti_cred_json['LTI_SETUP'] =
                settings.LTI_SETUP

        return JsonResponse(lti_cred_json)
\end{lstlisting}
\lstset{language=Bash}

\section{Connectividad al API de GitLab}
La conexión a la API de GitLab se realiza de la siguiente manera:
\lstset{language=Python}
\begin{lstlisting}
gitlab_default_srv = GITLAB_DEFAULT_SERVER


def connect_to_gitlab_token(protocol=None,
        host=None, port=None, token=None):
    if host is None:
        return None
    # gitlab_conn = gitlab.Gitlab(protocol
            + host + port, token)
    gitlab_conn = gitlab.Gitlab(protocol
            + host, token)
    #gitlab_conn.auth()
    return gitlab_conn


def connect_to_settings_gitlab_token(
        indx=GITLAB_DEFAULT_SERVER):
    return connect_to_gitlab_token(
            protocol=GITLAB_SERVERS
                [indx]['API_PROTOCOL'],
            host=GITLAB_SERVERS[indx]
                ['HOST'],
            port=GITLAB_SERVERS[indx]
                ['API_PROTOCOL'],
            token=GITLAB_SERVERS[indx]
                ['TOKEN'])


def connect_to_gitlab_user_password(protocol=None,
        host=None, port=None, user=None,
        password=None):
    if host is None:
        return None
    gitlab_conn = gitlab.Gitlab(protocol + host
            + port, email=user, password=password)
    gitlab_conn.auth()
    return gitlab_conn


def connect_to_settings_gitlab_user_password(
        indx=GITLAB_DEFAULT_SERVER):
    return connect_to_gitlab_user_password(
            protocol=GITLAB_SERVERS[indx]
                    ['API_PROTOCOL'],
            host=GITLAB_SERVERS[indx]['HOST'],
            port=GITLAB_SERVERS[indx]
                    ['API_PROTOCOL'],
            user=GITLAB_SERVERS[indx]['USER'],
            password=GITLAB_SERVERS[indx]
                    ['PASSWORD'])


def connect_to_settings_gitlab(
        indx=GITLAB_DEFAULT_SERVER):
    if GITLAB_SERVERS[indx]['WITH_TOKEN']:
        return connect_to_settings_gitlab_token(
                indx)
    elif GITLAB_SERVERS[indx]['WITH_CRED']:
        return connect_to_settings_gitlab_user_password(
                indx)
    else:
        return None

gitlab_srv = None
try:
    gitlab_srv = connect_to_settings_gitlab(
            gitlab_default_srv)
except Exception as e:
    print("No se pudo connectar a GitLab")
    print(e)
\end{lstlisting}
\lstset{language=Bash}

\chapter{Desarollo de EduNube}
\label{AnexoE}

\section{Configuracion de Base de Datos Relacional}
Primero es necesario crear y configurar un usuario y base de datos para ser ocupado:
\begin{lstlisting}
su - postgres
psql
\end{lstlisting}
\begin{lstlisting}[breaklines]
	postgres=# CREATE USER edunubeser WITH PASSWORD '3d?N_6E';
	postgres=# CREATE DATABASE edunubedb WITH OWNER edunubeser;
	postgres=# \q
psql edunubedb -U postgres
	edunubedb=# CREATE SCHEMA edunubeapp AUTHORIZATION edunubeser;
	edunubedb=# ALTER USER edunubeser SET search_path TO edunubeapp;
	edunubedb=# \q
psql edunubedb -U edunubeser
	edunubedb=> SELECT current_schema();
		Debe decir: edunubeapp
	edunubedb=> \q
\end{lstlisting}

Segundo, se abre el settings.py (dentro de la carpeta EduNube/EduNube) y se remplaza (para desarrollo, en producción debe llevar valores distintos) el atributo DATABASES con lo siguiente:
\lstset{language=Python}
\begin{lstlisting}[breaklines]
DATABASES = {
    'default': {
        'ENGINE': 'django.db.backends.postgresql_psycopg2',
        'NAME': 'edunubedb',
        'USER': 'edunubeser',
        'PASSWORD': '3d?N_6E',
        'HOST': '127.0.0.1',
        'PORT': '5432',
    }
}
\end{lstlisting}
\lstset{language=Bash}

\section{Configuracion de Base de Datos No Relacional}
Para crear la base de datos de MongoDB:
\begin{lstlisting}
mongo
\end{lstlisting}
\lstset{language=sql}
\begin{lstlisting}
use eduNubeDB
db.createUser(
    {
        user: "eduNubeUser",
        pwd: "3d?N_6E",
        roles: [ "readWrite", "dbAdmin" ]
    }
)
\end{lstlisting}
\lstset{language=Bash}

El mismo se define en el settings de la siguiente manera:
\lstset{language=Python}
\begin{lstlisting}
NOSQL_DATABASES = {
    'nosql': {
        'NAME': 'eduNubeDB',
        'USER': "eduNubeUser",
        'PASSWORD': '3d?N_6E',
        'HOST': '127.0.0.1',
        'PORT': '27017',
    }
}
\end{lstlisting}
\lstset{language=Bash}

\section{Implementacion de API Tokens con JWT}
A continuación se presenta la lógica para descifrar y cifrar los tokenes con el fin de generar/actualizarlo y también poderlo validar contra una tabla interna de la base de datos a lo que un cliente se lo envía:
\begin{lstlisting}
def decode_api_token(api_token=None):
    if api_token is None:
        raise ValueError("API_Token can't be None")
    return jwt.decode(api_token.token, api_token.secret_key,\
        algorithms=[api_token.token_algo])

def update_api_token(api_token=None, regen_secret_key=False):
    if api_token is None:
        raise ValueError("API_Token can't be None")
    if regen_secret_key or api_token.secret_key is None or\
            len(api_token.secret_key) == 0:
        api_token.secret_key = bcrypt.gensalt()
            # Generate Random Unique Secret_Key
    api_token.edit_date_in_token =\
        api_token.edit_date.__str__()
    payload = {
        'app_name': api_token.app_name,
        'created_date': api_token.created_date.__str__(),
        'edit_date': api_token.edit_date_in_token,
        'expires': api_token.expires
    }
    if api_token.expires:
        if api_token.expire_date is not None:
            payload['expire_date'] = api_token.expire_date
    api_token.token = jwt.encode(payload,\
        api_token.secret_key, algorithm='HS256')\
        # Generate Token
    api_token.save()
\end{lstlisting}

Como parte de la administración de este aspecto del sistema de autenticación, se encuentra implementado las funcionalidades de crear, actualizar, leer y borrar estos tokens de autenticación, siempre y cuando el usuario logueado tenga el permiso ''auth\_admin.manage\_tokens''. Si no se crea este permiso, como es el caso del ambiente de desarrollo, la política de Django es solo permitir cuentas de superusuario para acceder a estas vistas.

% TODO: Transfer from Chapter 4 to here

%\section{How do I change the colors of links?}
%
%The color of links can be changed to your liking using:
%
%{\small\verb!\hypersetup{urlcolor=red}!}, or
%
%{\small\verb!\hypersetup{citecolor=green}!}, or
%
%{\small\verb!\hypersetup{allcolor=blue}!}.
%
%\noindent If you want to completely hide the links, you can use:
%
%{\small\verb!\hypersetup{allcolors=.}!}, or even better: 
%
%{\small\verb!\hypersetup{hidelinks}!}.
%
%\noindent If you want to have obvious links in the PDF but not the printed text, use:
%
%{\small\verb!\hypersetup{colorlinks=false}!}.
%
% Appendix F - H - Pruebas

\chapter{Pruebas de GitEDU}

\label{AnexoF} 

\chapter{Pruebas de EduNube}

\label{AnexoG}

\chapter{Pruebas de GitServerHTTPEndpoint}

\label{AnexoH}

% Appendix I - Source Code Locations

\chapter{Ubicacion de Codigo Fuente y Licencia}

\label{AnexoI} 

\section{Ubicacion de Codigo Fuente}
Se puede encontrar el codigo fuente de este trabajo de titulacion en la siguiente direccion: \url{https://gitlab.com/nishedcob/GitEDU}

\section{Licencia de Codigo}


%----------------------------------------------------------------------------------------
%	BIBLIOGRAPHY
%----------------------------------------------------------------------------------------
%\nocite{*}

\printbibliography[heading=bibintoc]

%----------------------------------------------------------------------------------------

% INDEX
\addcontentsline{toc}{chapter}{Índice alfabético}
\printindex

\end{document}  
